\section{Phân tích hệ thống}
\label{sec:analysis}

Chương này trình bày phân tích hệ thống cho trò chơi \textit{Fortress of the Fallen} ở mức logic, nhằm chuyển hoá yêu cầu (Chương~\ref{sec:requirements}) thành các mô tả vận hành có thể thiết kế và triển khai: tác nhân tham gia, các ca sử dụng theo từng phân hệ, luồng nghiệp vụ trọng yếu, dữ liệu phát sinh và các ràng buộc quan trọng. Kết quả phân tích là cơ sở trực tiếp cho thiết kế kiến trúc, thiết kế dữ liệu và thiết kế giao tiếp ở Chương~7.

% ===============================================================
\subsection{Tổng quan phân hệ và ranh giới trách nhiệm}
\label{subsec:analysis_overview}

Hệ thống được tổ chức theo mô hình client--server, trong đó:
\begin{itemize}
    \item \textbf{Client} chịu trách nhiệm hiển thị, UI/UX, thu thập input và gửi yêu cầu/ý định hành động.
    \item \textbf{Server} chịu trách nhiệm xác thực, quản lý phiên, kiểm tra luật nghiệp vụ và xác nhận các thay đổi ảnh hưởng tiến trình (reward/currency/progression/gacha/island).
\end{itemize}

Các phân hệ nghiệp vụ chính của game được phân tích trong chương này gồm:
\begin{itemize}
    \item \textbf{Identity \& Profile}: tài khoản, thiết lập người dùng, game profile và phiên hoạt động.
    \item \textbf{Character \& Progression}: nhân vật, level/EXP, phân bổ chỉ số, mở slot kỹ năng, hệ race/class và điều kiện mở khoá.
    \item \textbf{Inventory \& Economy}: quản lý inventory, currency, trang bị, nâng cấp, tham chiếu tới cấu hình item.
    \item \textbf{Combat \& Gameplay Modes}: tháp trung tâm, dungeon theo phiên (instance), arena PvP; vòng lặp chiến đấu thời gian thực.
    \item \textbf{Recruitment (Gacha)}: banner, tỉ lệ, pity, xử lý trùng lặp và lịch sử.
    \item \textbf{Personal Island \& NPC}: đảo cá nhân, công trình, timer xây dựng/nâng cấp, tài nguyên, bộ sưu tập NPC và phân công worker.
    \item \textbf{Administration}: quản trị user, trạng thái, log/audit và cấu hình sự kiện.
\end{itemize}

Ngoài ra, hệ thống có lớp \textbf{game configuration} (dữ liệu tĩnh) phục vụ định nghĩa item/building/skill/rate, được quản lý theo pipeline riêng và được nạp vào \textit{Config Manager} để truy xuất nhất quán trong runtime.

% ===============================================================
\subsection{Tác nhân của hệ thống (Actors)}
\label{subsec:analysis_actors}

Các tác nhân tương tác với hệ thống được xác định như sau.

\begin{table}[H]
\centering
\renewcommand{\arraystretch}{1.2}
\setlength{\tabcolsep}{10pt}
\begin{tabular}{|p{3.2cm}|p{11cm}|}
\hline
\textbf{Tác nhân} & \textbf{Mô tả} \\
\hline
Player & Người chơi sử dụng game client để đăng nhập, chọn profile/nhân vật, tham gia nội dung chiến đấu, quản lý tiến trình và tài sản. \\
\hline
Admin & Người quản trị hệ thống; có quyền thao tác trên user, log, mail hệ thống và cấu hình sự kiện. \\
\hline
Content Designer & Người thiết kế nội dung/thông số; quản lý bảng cấu hình (items/buildings/skills/rates) và xuất dữ liệu theo pipeline cấu hình. \\
\hline
Game Client & Ứng dụng chạy trên máy người chơi; hiển thị, gửi request HTTP, duy trì kênh real-time và hiển thị trạng thái. \\
\hline
Backend Server & Thành phần xử lý xác thực, nghiệp vụ và đồng bộ; ghi dữ liệu vào DB và cung cấp giao tiếp cho client/admin tools. \\
\hline
Database & Lưu trữ bền vững: dữ liệu tiến trình người chơi và các collections cấu hình. \\
\hline
\end{tabular}
\caption{Danh sách tác nhân và vai trò}
\label{tab:actors}
\end{table}

% ===============================================================
\subsection{Phân tích Use Case theo phân hệ}
\label{subsec:analysis_usecase}

Phần này trình bày use case theo từng phân hệ nghiệp vụ. Mỗi nhóm use case đi kèm sơ đồ tổng quan và các use case trọng yếu sẽ được đặc tả chi tiết ở Mục~\ref{subsec:analysis_usecase_specs}.

% ---------------------------
\subsubsection{Identity \& Profile}
\label{subsec:analysis_uc_auth}

\begin{figure}[H]
    \centering
    \includegraphics[width=0.92\textwidth]{Images/auth.png}
    \caption{Use case tổng quan cho phân hệ Identity \& Profile}
    \label{fig:uc-auth}
\end{figure}

Nhóm này bao gồm đăng ký/đăng nhập, quản lý thiết lập, tạo/chọn game profile và thiết lập session runtime để chuẩn bị vào phiên chơi.

% ---------------------------
\subsubsection{Character \& Progression}
\label{subsec:analysis_uc_progress}

\begin{figure}[H]
    \centering
    \includegraphics[width=0.8\textwidth]{Images/progress.png}
    \caption{Use case tổng quan cho phân hệ Character \& Progression}
    \label{fig:uc-progress}
\end{figure}

Nhóm này tập trung vào tạo/chọn nhân vật, level/EXP, phân bổ chỉ số, quản lý kỹ năng theo slot, combo skill và mở khoá/chuyển class theo điều kiện.

% ---------------------------
\subsubsection{Combat \& Gameplay Modes}
\label{subsec:analysis_uc_combat}

\begin{figure}[H]
    \centering
    \includegraphics[width=0.95\textwidth]{Images/combat.png}
    \caption{Use case tổng quan cho phân hệ Combat \& Gameplay Modes}
    \label{fig:uc-combat}
\end{figure}

Nhóm này mô tả hành trình tham gia nội dung (tower/dungeon/arena), vòng lặp chiến đấu real-time, xử lý kết phiên và ghi nhận reward/checkpoint.

% ---------------------------
\subsubsection{Recruitment (Gacha)}
\label{subsec:analysis_uc_gacha}

\begin{figure}[H]
    \centering
    \includegraphics[width=0.92\textwidth]{Images/gacha.png}
    \caption{Use case tổng quan cho phân hệ Recruitment (Gacha)}
    \label{fig:uc-gacha}
\end{figure}

Nhóm này mô tả quy trình xem banner/tỉ lệ, thực hiện summon, cập nhật pity, xử lý trùng lặp và ghi lịch sử.

% ---------------------------
\subsubsection{Personal Island \& NPC}
\label{subsec:analysis_uc_island}

\begin{figure}[H]
    \centering
    \includegraphics[width=0.95\textwidth]{Images/island.png}
    \caption{Use case tổng quan cho phân hệ Personal Island \& NPC}
    \label{fig:uc-island}
\end{figure}

Nhóm này mô tả đảo cá nhân theo mô hình base-building: xây/nâng cấp công trình theo timer, quản lý tài nguyên, và phân công NPC làm worker.

% ---------------------------
\subsubsection{Administration}
\label{subsec:analysis_uc_admin}

\begin{figure}[H]
    \centering
    \includegraphics[width=0.\textwidth]{Images/admin.png}
    \caption{Use case tổng quan cho phân hệ Administration}
    \label{fig:uc-admin}
\end{figure}

Nhóm này bao gồm các tác vụ quản trị: quản lý trạng thái user, ban/unban, gửi mail hệ thống, xem log/audit và cấu hình sự kiện.

% ===============================================================
\subsection{Đặc tả Use Case trọng yếu}
\label{subsec:analysis_usecase_specs}

Phần này đặc tả các use case có ảnh hưởng trực tiếp tới tiến trình và tính nhất quán dữ liệu. Mỗi use case gồm: mục tiêu, tác nhân, tiền điều kiện, hậu điều kiện, luồng chính và luồng ngoại lệ.

% ---------------------------
\subsubsection{UC-A01 -- Đăng ký tài khoản}
\label{subsubsec:uc-a01}

\textbf{Mục tiêu:} Player tạo tài khoản mới để sử dụng hệ thống.

\textbf{Tác nhân chính:} Player.

\textbf{Tiền điều kiện:} Player chưa đăng nhập; client có thể kết nối API.

\textbf{Hậu điều kiện:} USER được tạo trong DB; trạng thái tài khoản hợp lệ để đăng nhập.

\textbf{Luồng chính:}
\begin{enumerate}
    \item Player nhập username/email và mật khẩu.
    \item Client gửi request đăng ký.
    \item Server kiểm tra định dạng, trùng lặp và chính sách mật khẩu.
    \item Server băm mật khẩu và tạo USER (kèm settings mặc định).
    \item Server trả kết quả thành công.
\end{enumerate}

\textbf{Ngoại lệ:}
\begin{itemize}
    \item Email/username trùng $\rightarrow$ trả lỗi trùng lặp.
    \item Dữ liệu không hợp lệ $\rightarrow$ trả lỗi theo trường.
\end{itemize}

% ---------------------------
\subsubsection{UC-A02 -- Đăng nhập và tạo phiên hoạt động}
\label{subsubsec:uc-a02}

\textbf{Mục tiêu:} Xác thực Player, cấp token và thiết lập \texttt{SESSION\_RUNTIME} để duy trì phiên.

\textbf{Tác nhân chính:} Player.

\textbf{Tiền điều kiện:} USER tồn tại và không bị khoá.

\textbf{Hậu điều kiện:} Token hợp lệ được cấp; server ghi nhận \texttt{last\_login}; phiên runtime có thể tham chiếu profile/character khi Player vào game.

\textbf{Luồng chính:}
\begin{enumerate}
    \item Player nhập thông tin đăng nhập.
    \item Client gửi request login.
    \item Server xác thực thông tin, kiểm tra trạng thái tài khoản.
    \item Server cấp token (có thời hạn) và tạo hoặc cập nhật session runtime (token, socket\_id nếu có, expires\_at).
    \item Client lưu token và điều hướng sang bước chọn profile.
\end{enumerate}

\textbf{Ngoại lệ:}
\begin{itemize}
    \item Sai thông tin $\rightarrow$ trả lỗi xác thực.
    \item Token hết hạn/không hợp lệ ở các request sau $\rightarrow$ yêu cầu login lại.
\end{itemize}

% ---------------------------
\subsubsection{UC-A03 -- Tạo và chọn Game Profile}
\label{subsubsec:uc-a03}

\textbf{Mục tiêu:} Player tạo/chọn GAME\_PROFILE để tách tiến trình theo nhiều profile (server/slot) trong cùng user.

\textbf{Tác nhân chính:} Player.

\textbf{Tiền điều kiện:} Đã đăng nhập (UC-A02).

\textbf{Hậu điều kiện:} GAME\_PROFILE được tạo hoặc được chọn; session runtime cập nhật \texttt{profile\_id}.

\textbf{Luồng chính:}
\begin{enumerate}
    \item Client hiển thị danh sách profile hiện có.
    \item Player tạo profile mới hoặc chọn profile.
    \item Server ghi \texttt{last\_played} khi profile được chọn.
    \item Server cập nhật session runtime với \texttt{profile\_id}.
\end{enumerate}

\textbf{Ngoại lệ:}
\begin{itemize}
    \item Vượt giới hạn số profile (nếu áp dụng) $\rightarrow$ trả lỗi giới hạn.
\end{itemize}

% ---------------------------
\subsubsection{UC-P01 -- Tạo nhân vật và khởi tạo dữ liệu nền}
\label{subsubsec:uc-p01}

\textbf{Mục tiêu:} Player tạo CHARACTER thuộc GAME\_PROFILE, kèm dữ liệu nền: level, base stats, resources, position và appearance.

\textbf{Tác nhân chính:} Player.

\textbf{Tiền điều kiện:} Đã chọn profile (UC-A03); còn slot nhân vật (theo chính sách).

\textbf{Hậu điều kiện:} CHARACTER và INVENTORY 1:1 được tạo; có thể chọn nhân vật để vào game.

\textbf{Luồng chính:}
\begin{enumerate}
    \item Player chọn race, class khởi đầu, đặt tên và chọn appearance.
    \item Client gửi request tạo nhân vật.
    \item Server kiểm tra hợp lệ (tên, ràng buộc race/class).
    \item Server tạo CHARACTER (level=1, exp=0), gán base stats mặc định, resources mặc định và vị trí khởi tạo.
    \item Server tạo INVENTORY rỗng (max\_slots mặc định, currency mặc định).
    \item Server trả dữ liệu nhân vật để client hiển thị.
\end{enumerate}

\textbf{Ngoại lệ:}
\begin{itemize}
    \item Tên không hợp lệ/trùng $\rightarrow$ trả lỗi và yêu cầu nhập lại.
    \item Race/Class không hợp lệ $\rightarrow$ trả lỗi ràng buộc.
\end{itemize}

% ---------------------------
\subsubsection{UC-P02 -- Lên level và phân bổ điểm chỉ số}
\label{subsubsec:uc-p02}

\textbf{Mục tiêu:} Khi nhận đủ EXP, nhân vật lên level và nhận điểm phân bổ chỉ số theo quy tắc.

\textbf{Tác nhân chính:} Player (thông qua hành động gameplay).

\textbf{Tiền điều kiện:} CHARACTER tồn tại; có sự kiện nhận EXP.

\textbf{Hậu điều kiện:} Level tăng (tối đa 100); cộng thêm \textbf{+5 stat points} mỗi level; mở slot kỹ năng theo milestone khi đạt mốc.

\textbf{Luồng chính:}
\begin{enumerate}
    \item Sau khi hoàn thành nội dung, server tính EXP và cộng vào \texttt{current\_exp}.
    \item Nếu vượt ngưỡng, server tăng level và cộng \texttt{stat\_points\_available} (+5).
    \item Nếu level đạt mốc mở slot, server cập nhật trạng thái slot (normal/combo).
    \item Client hiển thị thông báo lên level và cho phép phân bổ điểm.
\end{enumerate}

\textbf{Ngoại lệ:}
\begin{itemize}
    \item Đạt level 100 $\rightarrow$ không tăng level, chỉ cộng EXP theo chính sách hoặc khoá EXP.
\end{itemize}

% ---------------------------
\subsubsection{UC-P03 -- Quản lý kỹ năng theo slot và tạo Combo Skill}
\label{subsubsec:uc-p03}

\textbf{Mục tiêu:} Player trang bị kỹ năng vào slot; combo skill chỉ hợp lệ khi có slot combo và thoả quy tắc kết hợp.

\textbf{Tác nhân chính:} Player.

\textbf{Tiền điều kiện:} Player sở hữu kỹ năng đã học; slot tương ứng đã được mở theo level.

\textbf{Hậu điều kiện:} Bộ kỹ năng trang bị được cập nhật; combo skill (nếu có) được lưu như một cấu hình trang bị hợp lệ.

\textbf{Luồng chính:}
\begin{enumerate}
    \item Player mở màn hình kỹ năng và xem danh sách kỹ năng đã học.
    \item Player kéo/thả hoặc chọn kỹ năng để gán vào normal slots.
    \item Player chọn các kỹ năng thành phần để tạo combo, sau đó gán vào combo slot.
    \item Server kiểm tra điều kiện (slot mở, điều kiện class/race/level, quy tắc combo).
    \item Server lưu trạng thái trang bị và trả kết quả.
\end{enumerate}

\textbf{Ngoại lệ:}
\begin{itemize}
    \item Chưa mở combo slot $\rightarrow$ từ chối thao tác gán combo.
    \item Combo không hợp lệ $\rightarrow$ trả lỗi quy tắc.
\end{itemize}

% ---------------------------
\subsubsection{UC-P04 -- Mở khoá Class theo điều kiện và chuyển Class}
\label{subsubsec:uc-p04}

\textbf{Mục tiêu:} Player mở khoá class mới dựa trên ngưỡng chỉ số/trait/achievement và có thể chuyển class theo chính sách.

\textbf{Tác nhân chính:} Player.

\textbf{Tiền điều kiện:} CHARACTER tồn tại; có dữ liệu chỉ số và trạng thái trait/achievement.

\textbf{Hậu điều kiện:} Class mới được đánh dấu đã mở; nếu chuyển class, server cập nhật class hiện tại và các ràng buộc liên quan (kỹ năng class, loadout).

\textbf{Luồng chính:}
\begin{enumerate}
    \item Player xem danh sách class và điều kiện mở.
    \item Player thực hiện hành động ``Unlock'' khi tin rằng đã đủ điều kiện.
    \item Server kiểm tra điều kiện (stat threshold, karma/affinity/luck/resistance, điều kiện tiến trình).
    \item Nếu đạt, server ghi nhận class đã mở và trả kết quả.
    \item (Tuỳ chính sách) Player chọn ``Switch class''; server cập nhật class hiện hành và hiệu lực kỹ năng liên quan.
\end{enumerate}

\textbf{Ngoại lệ:}
\begin{itemize}
    \item Không đủ điều kiện $\rightarrow$ trả lỗi và hiển thị phần thiếu.
    \item Class yêu cầu race đặc thù $\rightarrow$ từ chối nếu race không khớp.
\end{itemize}

% ---------------------------
\subsubsection{UC-I01 -- Nhận reward và cập nhật Inventory}
\label{subsubsec:uc-i01}

\textbf{Mục tiêu:} Ghi nhận item/currency sau khi hoàn thành nội dung; đảm bảo quy tắc stack và dung lượng inventory.

\textbf{Tác nhân chính:} Player (kết quả từ gameplay).

\textbf{Tiền điều kiện:} Nội dung kết thúc và có reward; INVENTORY tồn tại.

\textbf{Hậu điều kiện:} INVENTORY được cập nhật nhất quán; item instance tham chiếu đúng item config.

\textbf{Luồng chính:}
\begin{enumerate}
    \item Server xác định reward (item, gold, gem, \ldots).
    \item Với item: server áp dụng quy tắc stack theo \texttt{CONFIG\_ITEM.max\_stack}.
    \item Nếu thiếu slot, server xử lý theo chính sách (gửi mailbox, trả về kho tạm, hoặc từ chối nhận).
    \item Server ghi cập nhật inventory và trả danh sách thay đổi cho client.
\end{enumerate}

\textbf{Ngoại lệ:}
\begin{itemize}
    \item Inventory đầy $\rightarrow$ kích hoạt chính sách overflow (tuỳ thiết kế).
\end{itemize}

% ---------------------------
\subsubsection{UC-C01 -- Tham gia nội dung theo phiên (Tower/Dungeon/Arena)}
\label{subsubsec:uc-c01}

\textbf{Mục tiêu:} Player chọn nội dung, tham gia instance/room và bắt đầu vòng lặp combat real-time.

\textbf{Tác nhân chính:} Player.

\textbf{Tiền điều kiện:} Đã chọn nhân vật; kết nối real-time sẵn sàng.

\textbf{Hậu điều kiện:} Player được gán vào instance; trạng thái phiên được ghi vào session runtime; client nhận dữ liệu khởi tạo.

\textbf{Luồng chính:}
\begin{enumerate}
    \item Player chọn chế độ (tower/dungeon/arena) và tham số (tầng, độ khó, \ldots).
    \item Client gửi request/join event.
    \item Server kiểm tra điều kiện tham gia (level, tài nguyên, vé, \ldots).
    \item Server tạo hoặc gán vào instance, gửi thông tin spawn/room state.
    \item Client load scene và hiển thị HUD chiến đấu.
\end{enumerate}

\textbf{Ngoại lệ:}
\begin{itemize}
    \item Không đủ điều kiện $\rightarrow$ trả lỗi điều kiện và không tạo instance.
\end{itemize}

% ---------------------------
\subsubsection{UC-C02 -- Vòng lặp chiến đấu real-time và đồng bộ trạng thái}
\label{subsubsec:uc-c02}

\textbf{Mục tiêu:} Xử lý hành động combat và đồng bộ trạng thái giữa client và server trong phiên chơi.

\textbf{Tác nhân chính:} Player.

\textbf{Tiền điều kiện:} Đã ở trong instance (UC-C01) và có kênh real-time.

\textbf{Hậu điều kiện:} Trạng thái authoritative được cập nhật; client hiển thị nhất quán theo snapshot/event.

\textbf{Luồng chính:}
\begin{enumerate}
    \item Client gửi input/intent (move, attack, cast skill, dodge) theo nhịp.
    \item Server xác thực input theo luật (cooldown, tài nguyên, vị trí hợp lệ).
    \item Server cập nhật trạng thái (HP/MP/buff/debuff/position) và phát snapshot/event.
    \item Client nội suy/trình diễn (animation, VFX) theo trạng thái nhận được.
\end{enumerate}

\textbf{Ngoại lệ:}
\begin{itemize}
    \item Mất kết nối $\rightarrow$ server đóng phiên hoặc cho phép reconnect trong thời gian cho phép.
\end{itemize}

% ---------------------------
\subsubsection{UC-C03 -- Kết phiên, checkpoint và ghi nhận tiến trình}
\label{subsubsec:uc-c03}

\textbf{Mục tiêu:} Sau khi kết thúc phiên, hệ thống ghi nhận kết quả: checkpoint, EXP, reward, log nghiệp vụ.

\textbf{Tác nhân chính:} Player.

\textbf{Tiền điều kiện:} Phiên kết thúc (thắng/thua/thoát).

\textbf{Hậu điều kiện:} Dữ liệu tiến trình được cập nhật bền vững; reward được đưa vào inventory; checkpoint được ghi theo chính sách.

\textbf{Luồng chính:}
\begin{enumerate}
    \item Server xác định trạng thái kết phiên và bảng phần thưởng.
    \item Server cập nhật EXP/level (UC-P02), cập nhật inventory (UC-I01).
    \item Server cập nhật checkpoint (tower/dungeon) nếu đạt điều kiện.
    \item Server ghi log nghiệp vụ cho các giao dịch quan trọng.
    \item Client hiển thị màn hình kết quả và phần thưởng.
\end{enumerate}

% ---------------------------
\subsubsection{UC-G01 -- Gacha: Summon, pity và xử lý trùng lặp}
\label{subsubsec:uc-g01}

\textbf{Mục tiêu:} Player thực hiện summon theo banner; hệ thống trả kết quả theo tỉ lệ, cập nhật pity và xử lý duplicate.

\textbf{Tác nhân chính:} Player.

\textbf{Tiền điều kiện:} Banner đang hoạt động; Player có đủ tài nguyên summon.

\textbf{Hậu điều kiện:} Kết quả summon được ghi nhận; NPC/đơn vị mới thêm vào collection hoặc chuyển đổi theo rule; pity cập nhật nhất quán.

\textbf{Luồng chính:}
\begin{enumerate}
    \item Player chọn banner và hình thức summon (1x/10x).
    \item Server kiểm tra tài nguyên và trừ chi phí.
    \item Server thực hiện RNG theo rate của banner; áp dụng pity nếu đạt ngưỡng.
    \item Với kết quả trùng: server chuyển đổi sang shard/point theo rule.
    \item Server ghi lịch sử gacha và trả danh sách kết quả cho client.
\end{enumerate}

\textbf{Ngoại lệ:}
\begin{itemize}
    \item Không đủ tài nguyên $\rightarrow$ từ chối summon.
    \item Banner hết hạn $\rightarrow$ yêu cầu refresh danh sách banner.
\end{itemize}

% ---------------------------
\subsubsection{UC-S01 -- Xây/Nâng cấp công trình trên đảo cá nhân}
\label{subsubsec:uc-s01}

\textbf{Mục tiêu:} Player xây hoặc nâng cấp công trình; hệ thống trừ tài nguyên và ghi timer.

\textbf{Tác nhân chính:} Player.

\textbf{Tiền điều kiện:} PERSONAL\_ISLAND tồn tại; có đủ tài nguyên; building config hợp lệ.

\textbf{Hậu điều kiện:} BUILDING instance được tạo/cập nhật trạng thái; \texttt{finish\_time} được thiết lập; tài nguyên đảo được cập nhật.

\textbf{Luồng chính:}
\begin{enumerate}
    \item Player chọn loại công trình và vị trí grid.
    \item Client gửi request build/upgrade.
    \item Server kiểm tra vị trí hợp lệ, không chồng lấn, thoả điều kiện level/town hall.
    \item Server trừ tài nguyên theo \texttt{CONFIG\_BUILDING.cost} và đặt \texttt{finish\_time}.
    \item Client hiển thị trạng thái \textit{Constructing/Upgrading}.
\end{enumerate}

% ---------------------------
\subsubsection{UC-S02 -- Gán NPC làm worker cho công trình}
\label{subsubsec:uc-s02}

\textbf{Mục tiêu:} Player phân công NPC vào công trình để vận hành/đẩy tiến trình theo job.

\textbf{Tác nhân chính:} Player.

\textbf{Tiền điều kiện:} NPC\_COLLECTION tồn tại; NPC rảnh; công trình cho phép gán worker.

\textbf{Hậu điều kiện:} Building lưu \texttt{assigned\_worker\_id}; NPC cập nhật \texttt{current\_job}.

\textbf{Luồng chính:}
\begin{enumerate}
    \item Player chọn công trình và chọn NPC.
    \item Server kiểm tra ràng buộc (NPC rảnh, công trình chưa có worker hoặc cho phép thay thế).
    \item Server cập nhật liên kết hai chiều giữa building và NPC.
    \item Client hiển thị NPC đang làm việc tại công trình.
\end{enumerate}

% ---------------------------
\subsubsection{UC-AD01 -- Quản trị user: ban/unban và audit log}
\label{subsubsec:uc-ad01}

\textbf{Mục tiêu:} Admin quản lý trạng thái user, ban/unban và đảm bảo có audit log.

\textbf{Tác nhân chính:} Admin.

\textbf{Tiền điều kiện:} Admin đã xác thực và có quyền phù hợp.

\textbf{Hậu điều kiện:} Trạng thái user cập nhật; phiên liên quan có thể bị thu hồi theo chính sách; hành động được ghi log.

\textbf{Luồng chính:}
\begin{enumerate}
    \item Admin tìm kiếm user theo username/email/id.
    \item Admin thực hiện ban/unban hoặc thay đổi trạng thái.
    \item Server cập nhật USER.status và ghi audit log (actor, thời gian, hành động).
    \item (Tuỳ chính sách) Server thu hồi session runtime/token hiện hành.
\end{enumerate}

% ===============================================================
\subsection{Quy tắc nghiệp vụ và ràng buộc hệ thống}
\label{subsec:analysis_rules}

\subsubsection{Quy tắc Level và mở slot kỹ năng}
\label{subsec:analysis_rules_level}

Theo đặc tả hệ thống level:
\begin{itemize}
    \item Level tối đa: \textbf{100}.
    \item Mỗi level nhận \textbf{+5 stat points}.
    \item Level không tự động cấp kỹ năng; level chủ yếu mở slot để trang bị kỹ năng đã học.
\end{itemize}

\begin{table}[H]
\centering
\renewcommand{\arraystretch}{1.2}
\setlength{\tabcolsep}{8pt}
\begin{tabular}{|c|p{10.5cm}|}
\hline
\textbf{Mốc level} & \textbf{Phần thưởng tiến trình (slot)} \\
\hline
10 / 20 / 30 & Mở lần lượt Normal Skill Slot 1--3 \\
\hline
40 & Mở Combo Skill Slot 1 \\
\hline
50 & Mốc nội dung (boss loop) và điều chỉnh giới hạn stat cap theo thiết kế \\
\hline
60 / 80 / 100 & Mở lần lượt Combo Skill Slot 2--4; tại level 100 mở \textit{Ultimate Combo} \\
\hline
70 / 90 & Mở lần lượt Normal Skill Slot 4--5 \\
\hline
\end{tabular}
\caption{Mốc level và cơ chế mở slot kỹ năng}
\label{tab:level-milestones-analysis}
\end{table}

\subsubsection{Quy tắc Race, Stat Cap và Trait bẩm sinh}
\label{subsec:analysis_rules_race}

Race quyết định:
\begin{itemize}
    \item \textbf{Stat cap} theo từng thuộc tính.
    \item \textbf{Trait bẩm sinh} (ví dụ tăng kháng, tăng hiệu quả chế tạo, hoặc điều kiện kích hoạt theo HP).
\end{itemize}
Ràng buộc quan trọng là điểm phân bổ chỉ số không được vượt stat cap theo race, và một số class đặc thù yêu cầu race tương ứng.

\subsubsection{Quy tắc mở khoá Class}
\label{subsec:analysis_rules_class}

Class được phân tầng (Basic/Intermediate/Advanced/Legendary/Hidden). Điều kiện mở khoá có thể kết hợp:
\begin{itemize}
    \item Ngưỡng chỉ số (STR/DEX/CON/INT/WIS/CHA).
    \item Trait đặc biệt (Karma/Affinity/Luck/Resistance, \ldots).
    \item Điều kiện hành vi/tiến trình (ví dụ thắng PvP, tham gia arena, số NPC chỉ huy).
    \item Điều kiện race cho các class độc quyền.
\end{itemize}
Việc mở khoá cần đảm bảo kiểm tra trên server để tránh sai lệch do client.

\subsubsection{Quy tắc Inventory và tham chiếu cấu hình}
\label{subsec:analysis_rules_inventory}

\begin{itemize}
    \item Inventory có giới hạn slot; item áp dụng quy tắc stack theo cấu hình.
    \item Item trong inventory là \textbf{instance} (có \texttt{unique\_uid}, \texttt{enhancement\_level}, \texttt{durability}) và \textbf{tham chiếu} item definition bằng \texttt{item\_ref\_id}.
\end{itemize}

\subsubsection{Quy tắc Island và xây dựng công trình}
\label{subsec:analysis_rules_island}

\begin{itemize}
    \item Đảo cá nhân quản lý tài nguyên và danh sách công trình theo grid.
    \item Xây/nâng cấp công trình tạo timer (\texttt{finish\_time}) và trạng thái (building status).
    \item Worker assignment cần đồng bộ hai chiều giữa công trình và NPC để tránh trạng thái treo.
\end{itemize}

\subsubsection{Quy tắc Gacha: rate, pity và duplicate}
\label{subsec:analysis_rules_gacha}

\begin{itemize}
    \item Tỉ lệ theo banner là dữ liệu cấu hình; cần minh bạch hiển thị cho Player.
    \item Pity là trạng thái tiến trình, cần lưu bền vững theo banner hoặc theo nhóm banner.
    \item Kết quả trùng lặp được chuyển đổi theo rule (shard/point/\ldots) để tránh ``mất giá trị''.
\end{itemize}

% ===============================================================
\subsection{Phân tích dữ liệu ở mức logic}
\label{subsec:analysis_data}

\subsubsection{Phân lớp dữ liệu: động và tĩnh}
\label{subsec:analysis_data_types}

Dữ liệu trong hệ thống được phân thành hai nhóm:
\begin{itemize}
    \item \textbf{Dữ liệu tiến trình (dynamic)}: USER, GAME\_PROFILE, SESSION\_RUNTIME, CHARACTER, INVENTORY, PERSONAL\_ISLAND, NPC\_COLLECTION, pity state, lịch sử gacha, log nghiệp vụ.
    \item \textbf{Dữ liệu cấu hình (static)}: CONFIG\_ITEM, CONFIG\_BUILDING và các bảng cấu hình mở rộng (skills, rates, drop tables, \ldots).
\end{itemize}

\subsubsection{Các thực thể lõi theo phân hệ}
\label{subsec:analysis_entities}

\begin{itemize}
    \item \textbf{Identity}: \texttt{USER} (kèm \texttt{OBJECT\_USER\_SETTINGS}), \texttt{GAME\_PROFILE}, \texttt{SESSION\_RUNTIME}.
    \item \textbf{Character}: \texttt{CHARACTER} (base\_stats/resources/position/appearance dạng nhúng).
    \item \textbf{Inventory}: \texttt{INVENTORY} và danh sách \texttt{STRUCT\_INVENTORY\_ITEM} tham chiếu \texttt{CONFIG\_ITEM}.
    \item \textbf{Island}: \texttt{PERSONAL\_ISLAND} (resources nhúng) và \texttt{STRUCT\_BUILDING} tham chiếu \texttt{CONFIG\_BUILDING}.
    \item \textbf{NPC}: \texttt{NPC\_COLLECTION} và \texttt{STRUCT\_OWNED\_NPC} (tái sử dụng struct stats/resources).
\end{itemize}

\subsubsection{Quan hệ dữ liệu quan trọng}
\label{subsec:analysis_relationships}

Các quan hệ dữ liệu có ảnh hưởng trực tiếp tới luồng nghiệp vụ:
\begin{itemize}
    \item USER $\rightarrow$ GAME\_PROFILE là quan hệ 1:N.
    \item GAME\_PROFILE $\rightarrow$ CHARACTER là quan hệ 1:N.
    \item CHARACTER $\rightarrow$ INVENTORY là quan hệ 1:1.
    \item GAME\_PROFILE $\rightarrow$ PERSONAL\_ISLAND và NPC\_COLLECTION là quan hệ 1:1.
    \item Inventory item instance tham chiếu item config; building instance tham chiếu building config.
\end{itemize}

% ===============================================================
\subsection{Tổng kết chương}
\label{subsec:analysis_summary}

Chương này đã phân tích hệ thống theo các phân hệ nghiệp vụ của \textit{Fortress of the Fallen}: xác thực và profile, tiến trình nhân vật (level/slot/combo, race/class), inventory và kinh tế, combat theo phiên, gacha, đảo cá nhân \& NPC, và phân hệ quản trị. Đồng thời, chương đã xác định các use case trọng yếu, quy tắc nghiệp vụ và cấu trúc dữ liệu logic, làm tiền đề cho chương thiết kế hệ thống (Chương~7).
