\section{Phân tích yêu cầu}
\label{sec:requirements}

Chương này xác định yêu cầu cho hệ thống game \textit{Fortress of the Fallen} ở mức \textbf{phân tích hệ thống}. Nội dung bao gồm: phạm vi, tác nhân, nhóm yêu cầu chức năng theo phân hệ, yêu cầu phi chức năng và yêu cầu dữ liệu. Các yêu cầu ở chương này là cơ sở để xây dựng mô hình use case/luồng nghiệp vụ (Chương~6) và kiến trúc + thiết kế dữ liệu (Chương~7).

% ===============================================================
\subsection{Phạm vi và giả định}
\label{subsec:req-scope}

\subsubsection{Phạm vi giai đoạn 1}

Giai đoạn 1 tập trung vào \textbf{thiết kế hệ thống}. Do đó, các yêu cầu được mô tả theo hướng:
\begin{itemize}
    \item làm rõ \textbf{các tính năng cần tồn tại} và cách chúng liên kết với nhau;
    \item làm rõ \textbf{dữ liệu cần lưu} để đảm bảo nhất quán trạng thái người chơi;
    \item làm rõ \textbf{ranh giới trách nhiệm} giữa client và server cho các tác vụ request--response và real-time.
\end{itemize}

\subsubsection{Giả định}

\begin{itemize}
    \item Hệ thống vận hành theo mô hình client--server; server giữ vai trò quyết định cho các kết quả quan trọng (reward/currency/progression).
    \item Dữ liệu trong hệ thống được chia thành:
    \begin{itemize}
        \item \textbf{Player progression/state (động)}: thay đổi theo hành vi người chơi.
        \item \textbf{Game configuration (tĩnh)}: định nghĩa và tham số gameplay (item/building/skill/rates/\ldots).
    \end{itemize}
    \item Pipeline cấu hình \textbf{chỉ áp dụng cho game configuration}, không áp dụng cho toàn bộ dữ liệu DB.
\end{itemize}

% ===============================================================
\subsection{Tác nhân và ranh giới hệ thống}
\label{subsec:req-actors}

\subsubsection{Tác nhân}

\begin{itemize}
    \item \textbf{Player}: người chơi sử dụng client để trải nghiệm gameplay, quản lý tiến trình và tài sản.
    \item \textbf{Admin}: người quản trị hệ thống, có quyền thao tác trên user, log, mail và cấu hình sự kiện.
\end{itemize}

\subsubsection{Ranh giới hệ thống}

Hệ thống được chia theo các nhóm thành phần:
\begin{itemize}
    \item \textbf{Client}: hiển thị (render), UI, input, điều hướng màn hình, nội suy hiển thị; gửi yêu cầu/ý định hành động.
    \item \textbf{Server}: xác thực, quản lý phiên, xử lý nghiệp vụ, xác nhận kết quả, ghi DB.
    \item \textbf{Data layer}: DB cho progression/state; collections cấu hình cho configuration; cache/session runtime (định hướng).
\end{itemize}

% ===============================================================
\subsection{Yêu cầu chức năng}
\label{subsec:req-functional}

Phần này mô tả yêu cầu chức năng theo từng phân hệ. Mỗi yêu cầu được gắn mã định danh để thuận tiện đối chiếu về sau.

% ---------------------------
\subsubsection{Nhóm A: Authentication \& Profile}
\label{subsec:req-auth}

\begin{itemize}
    \item \textbf{FR-A01} -- Đăng ký tài khoản: Player có thể đăng ký bằng username/email và mật khẩu.
    \item \textbf{FR-A02} -- Đăng nhập: Player đăng nhập, nhận token phiên; server ghi nhận \texttt{last\_login}.
    \item \textbf{FR-A03} -- Khôi phục mật khẩu: cung cấp luồng khôi phục (tối thiểu ở mức thiết kế).
    \item \textbf{FR-A04} -- Quản lý thiết lập người dùng: lưu \textit{master volume}, \textit{sfx volume}, \textit{language}, \textit{notifications}.
    \item \textbf{FR-A05} -- Tạo game profile: một user có thể tạo nhiều profile (phục vụ nhiều server/nhân vật).
    \item \textbf{FR-A06} -- Chọn game profile: chọn profile để vào game; cập nhật \texttt{last\_played}.
    \item \textbf{FR-A07} -- Quản lý session runtime: hệ thống lưu trạng thái phiên hiện hành (profile/character/room\_id).
\end{itemize}

% ---------------------------
\subsubsection{Nhóm B: Character Creation \& Character Data}
\label{subsec:req-character}

\paragraph{Yêu cầu tạo nhân vật}
\begin{itemize}
    \item \textbf{FR-B01} -- Tạo nhân vật: Player tạo CHARACTER thuộc một GAME\_PROFILE.
    \item \textbf{FR-B02} -- Tuỳ chọn cơ bản: đặt tên, chọn giới tính (nếu có), chọn \textbf{race} và \textbf{class khởi đầu}.
    \item \textbf{FR-B03} -- Tuỳ chọn ngoại hình: lưu các tham số appearance (hair/skin/face) để tái hiện nhất quán.
    \item \textbf{FR-B04} -- Xoá nhân vật: Player có thể xoá CHARACTER theo chính sách (có thể yêu cầu xác nhận).
    \item \textbf{FR-B05} -- Chọn nhân vật: Player chọn nhân vật để vào phiên chơi.
\end{itemize}

\paragraph{Race system (yêu cầu mức dữ liệu và mở khoá)}
Dựa trên thiết kế chủng tộc, hệ thống cần thoả:
\begin{itemize}
    \item \textbf{FR-B06} -- Race ảnh hưởng \textbf{stat cap} và \textbf{trait bẩm sinh}.
    \item \textbf{FR-B07} -- Race hiếm có thể yêu cầu điều kiện mở khoá (ví dụ: Karma ngưỡng, achievement, hoặc điều kiện tiến trình).
    \item \textbf{FR-B08} -- Hiển thị thông tin race: client có thể xem mô tả, trait và stat cap trước khi xác nhận tạo nhân vật.
\end{itemize}

% ---------------------------
\subsubsection{Nhóm C: Stats, Leveling, Skill Slots}
\label{subsec:req-stats-level}

\paragraph{Level system}
\begin{itemize}
    \item \textbf{FR-C01} -- Level tối đa là \textbf{100}.
    \item \textbf{FR-C02} -- Mỗi lần lên level nhận \textbf{+5 stat points} để phân bổ.
    \item \textbf{FR-C03} -- Level không tự động cấp kỹ năng; kỹ năng được thu thập từ nguồn khác (loot/quest/NPC/\ldots).
\end{itemize}

\paragraph{Stat allocation}
\begin{itemize}
    \item \textbf{FR-C04} -- Player có thể phân bổ điểm vào các thuộc tính nền: STR, DEX, CON, INT, WIS, CHA.
    \item \textbf{FR-C05} -- Hệ thống tính các chỉ số suy diễn (HP/MP/ATK/DEF/CRIT/ACC/EVA/\ldots) từ thuộc tính nền theo công thức cấu hình.
    \item \textbf{FR-C06} -- Ràng buộc stat cap theo race: không cho phân bổ vượt ngưỡng cho phép.
\end{itemize}

\paragraph{Skill slots và combo slots}
Để hỗ trợ lối chơi đa dạng, hệ thống áp dụng cơ chế slot:
\begin{itemize}
    \item \textbf{FR-C07} -- Có \textbf{normal skill slots} để trang bị kỹ năng thường.
    \item \textbf{FR-C08} -- Có \textbf{combo skill slots} để trang bị kỹ năng kết hợp (combo).
    \item \textbf{FR-C09} -- Slot được mở dần theo level (milestone). Một cấu hình milestone tham chiếu:
    \begin{itemize}
        \item Normal slots: mở dần đến tối đa 5 slot.
        \item Combo slots: mở dần tại các mốc 40/60/80/100 đến tối đa 4 slot (slot cuối là \textit{ultimate combo}).
    \end{itemize}
\end{itemize}

\paragraph{Thu thập và quản lý kỹ năng}
\begin{itemize}
    \item \textbf{FR-C10} -- Player có thể sở hữu danh sách kỹ năng đã học (learned skills).
    \item \textbf{FR-C11} -- Player có thể trang bị kỹ năng vào slot; hệ thống kiểm tra điều kiện (class/race/level).
    \item \textbf{FR-C12} -- Combo skill chỉ hợp lệ nếu thoả quy tắc kết hợp (theo cấu hình).
\end{itemize}

% ---------------------------
\subsubsection{Nhóm D: Class System và mở khoá theo điều kiện}
\label{subsec:req-class}

\paragraph{Phân tầng class}
Hệ thống class được phân tầng để tạo meta-progression:
\begin{itemize}
    \item \textbf{FR-D01} -- Hỗ trợ nhiều nhánh class (Warrior/Archer/Mage/Healer/Rogue/Tactician).
    \item \textbf{FR-D02} -- Hỗ trợ nhiều tầng class: Basic, Intermediate, Advanced, Legendary và Hidden.
\end{itemize}

\paragraph{Điều kiện mở khoá class}
\begin{itemize}
    \item \textbf{FR-D03} -- Mở khoá class dựa trên điều kiện kết hợp:
    \begin{itemize}
        \item ngưỡng chỉ số (STR/DEX/CON/INT/WIS/CHA);
        \item trait đặc biệt (Karma/Affinity/Luck/Resistance/\ldots);
        \item điều kiện tiến trình/achievement (PvP, quest, sưu tầm, \ldots).
    \end{itemize}
    \item \textbf{FR-D04} -- Player có thể xem trước điều kiện và tiến độ đạt điều kiện để mở class.
    \item \textbf{FR-D05} -- Khi mở class thành công, Player có thể chuyển class theo chính sách (ví dụ: cần xác nhận, giới hạn số lần, hoặc chi phí).
\end{itemize}

% ---------------------------
\subsubsection{Nhóm E: Inventory, Equipment, Resources}
\label{subsec:req-inventory}

\begin{itemize}
    \item \textbf{FR-E01} -- Inventory là 1:1 với character, có số slot tối đa và số slot đã dùng.
    \item \textbf{FR-E02} -- Hỗ trợ currency cơ bản: gold và gem.
    \item \textbf{FR-E03} -- Item trong inventory là \textbf{item instance} (uid/quantity/slot/enhancement/durability) và tham chiếu đến \textbf{item definition} qua \texttt{item\_ref\_id}.
    \item \textbf{FR-E04} -- Stack rule: item có \texttt{max\_stack} theo cấu hình.
    \item \textbf{FR-E05} -- Equip: trang bị vũ khí/giáp theo slot (theo cấu hình), cập nhật chỉ số hiệu lực.
    \item \textbf{FR-E06} -- Enhance: nâng cấp trang bị theo chi phí và quy tắc; ghi nhận \texttt{enhancement\_level}.
    \item \textbf{FR-E07} -- Durability (tuỳ chọn theo thiết kế): giảm độ bền theo sử dụng và có cơ chế sửa/khôi phục.
\end{itemize}

% ---------------------------
\subsubsection{Nhóm F: Combat \& Gameplay Modes}
\label{subsec:req-combat}

\paragraph{Nội dung combat}
\begin{itemize}
    \item \textbf{FR-F01} -- Central Tower: chọn tầng và tham gia nội dung theo tầng.
    \item \textbf{FR-F02} -- Dungeon: tham gia nội dung PvE theo phiên (instance).
    \item \textbf{FR-F03} -- Arena: tham gia PvP arena (tối thiểu ở mức thiết kế).
\end{itemize}

\paragraph{Hành động trong combat}
\begin{itemize}
    \item \textbf{FR-F04} -- Normal attack.
    \item \textbf{FR-F05} -- Cast active skill.
    \item \textbf{FR-F06} -- Dodge/Dash.
    \item \textbf{FR-F07} -- Use consumable item.
    \item \textbf{FR-F08} -- Revive (nếu có cơ chế hồi sinh).
\end{itemize}

\paragraph{Checkpoint và reward}
\begin{itemize}
    \item \textbf{FR-F09} -- Checkpoint: lưu mốc tiến độ theo nội dung (tower/dungeon) theo chính sách.
    \item \textbf{FR-F10} -- Loot delivery: reward sau combat được ghi vào inventory; hệ thống kiểm tra capacity/stack.
    \item \textbf{FR-F11} -- Reward logging: các giao dịch reward/currency cần được ghi log nghiệp vụ để truy vết.
\end{itemize}

% ---------------------------
\subsubsection{Nhóm G: Recruitment (Gacha)}
\label{subsec:req-gacha}

\begin{itemize}
    \item \textbf{FR-G01} -- Banner: hiển thị danh sách banner đang hoạt động.
    \item \textbf{FR-G02} -- Rates: hiển thị tỉ lệ rơi theo banner (minh bạch).
    \item \textbf{FR-G03} -- Summon 1x và \textbf{FR-G04} -- Summon 10x.
    \item \textbf{FR-G05} -- Pity meter: theo dõi pity theo banner hoặc theo nhóm banner (theo cấu hình).
    \item \textbf{FR-G06} -- Exchange: đổi pity points theo rule.
    \item \textbf{FR-G07} -- Duplicate handling: xử lý trùng lặp theo rule (convert shard/point/\ldots).
    \item \textbf{FR-G08} -- Kết quả gacha được ghi vào DB và có thể truy vấn lịch sử (tối thiểu ở mức thiết kế).
\end{itemize}

% ---------------------------
\subsubsection{Nhóm H: Personal Island \& NPC Collection}
\label{subsec:req-island}

\paragraph{Personal island}
\begin{itemize}
    \item \textbf{FR-H01} -- Mỗi GAME\_PROFILE sở hữu 1 đảo cá nhân (1:1).
    \item \textbf{FR-H02} -- Quản lý tài nguyên đảo: wood/stone/food/iron (hoặc tập tương đương theo cấu hình).
    \item \textbf{FR-H03} -- Xây công trình theo grid: kiểm tra vị trí hợp lệ, không chồng lấn.
    \item \textbf{FR-H04} -- Construction timer: công trình có trạng thái và \texttt{finish\_time}.
    \item \textbf{FR-H05} -- Nâng cấp công trình: trừ tài nguyên và cập nhật level + timer.
    \item \textbf{FR-H06} -- Harvest: thu tài nguyên theo chu kỳ/điều kiện.
\end{itemize}

\paragraph{NPC collection và worker assignment}
\begin{itemize}
    \item \textbf{FR-H07} -- Mỗi GAME\_PROFILE sở hữu 1 bộ sưu tập NPC (1:1).
    \item \textbf{FR-H08} -- NPC có rarity/level/star/friendship/current\_job và có thể tái sử dụng cấu trúc stats/resources giống nhân vật.
    \item \textbf{FR-H09} -- Assign worker: gán NPC vào công trình; cập nhật liên kết hai chiều (building.assigned\_worker\_id và npc.current\_job).
    \item \textbf{FR-H10} -- NPC hoạt động tạo tài nguyên ở mức đơn giản (định hướng prototype), có thể mở rộng hành vi theo job.
\end{itemize}

% ---------------------------
\subsubsection{Nhóm I: Administration}
\label{subsec:req-admin}

\begin{itemize}
    \item \textbf{FR-I01} -- Quản lý user: xem thông tin, cập nhật trạng thái (active/banned/\ldots).
    \item \textbf{FR-I02} -- Ban/Unban: admin có thể ban/unban; thu hồi session theo chính sách.
    \item \textbf{FR-I03} -- System mail: gửi mail hệ thống theo user/profile/nhóm.
    \item \textbf{FR-I04} -- Log viewer: xem log theo thời gian/severity/actor.
    \item \textbf{FR-I05} -- Event config: tạo/sửa cấu hình sự kiện theo thời gian và phần thưởng (lưu DB).
\end{itemize}

% ===============================================================
\subsection{Yêu cầu phi chức năng}
\label{subsec:req-nonfunctional}

\subsubsection{Bảo mật và an toàn dữ liệu}

\begin{itemize}
    \item \textbf{NFR-S01} -- Mật khẩu phải lưu dạng hash; không lưu plaintext.
    \item \textbf{NFR-S02} -- Token/session có thời hạn; hỗ trợ revoke khi ban user hoặc logout.
    \item \textbf{NFR-S03} -- Phân quyền admin theo role; mọi thao tác admin cần audit log.
    \item \textbf{NFR-S04} -- Các thao tác reward/currency/progression phải chống cấp trùng (idempotent ở mức thiết kế).
\end{itemize}

\subsubsection{Hiệu năng và tính phản hồi}

\begin{itemize}
    \item \textbf{NFR-P01} -- Các API không real-time (login/profile/inventory) phải có thời gian phản hồi ổn định.
    \item \textbf{NFR-P02} -- Kênh real-time cần hỗ trợ cập nhật trạng thái thường xuyên (snapshot/event) để client nội suy mượt.
    \item \textbf{NFR-P03} -- Giảm truy vấn DB cho dữ liệu nóng bằng cache hoặc nạp cấu hình vào bộ nhớ (Config Manager).
\end{itemize}

\subsubsection{Tính mở rộng và bảo trì}

\begin{itemize}
    \item \textbf{NFR-M01} -- Hệ thống module hoá theo phân hệ, hạn chế phụ thuộc chéo.
    \item \textbf{NFR-M02} -- Dữ liệu cấu hình tĩnh tách khỏi code và hỗ trợ versioning.
    \item \textbf{NFR-M03} -- Có khả năng mở rộng theo chiều ngang ở mức kiến trúc (nhiều instance server).
\end{itemize}

\subsubsection{Khả dụng và quan sát hệ thống}

\begin{itemize}
    \item \textbf{NFR-O01} -- Log request và log nghiệp vụ cho các giao dịch quan trọng (reward, gacha, admin actions).
    \item \textbf{NFR-O02} -- Có khả năng truy vết sự cố theo user/profile/character và theo mốc thời gian.
\end{itemize}

\subsubsection{Yêu cầu UX/UI ở mức hệ thống}

\begin{itemize}
    \item \textbf{NFR-U01} -- UI phải hiển thị rõ trạng thái: loading/empty/error/success/disabled.
    \item \textbf{NFR-U02} -- Với phong cách pixel art, UI cần ưu tiên readability: khung panel rõ, icon theo lưới pixel, tương phản đủ.
    \item \textbf{NFR-U03} -- Combat HUD tối thiểu: HP/MP (hoặc tài nguyên tương đương), skill slots, cooldown, buff/debuff.
\end{itemize}

% ===============================================================
\subsection{Yêu cầu dữ liệu}
\label{subsec:req-data}

\subsubsection{Phân loại dữ liệu}

\paragraph{Dữ liệu tiến trình người chơi (dynamic)}
Bao gồm: user/profile/session, character, inventory, island state, npc collection, pity state, history logs.

\paragraph{Dữ liệu cấu hình game (static)}
Bao gồm: định nghĩa item/building/skill, bảng hệ số chỉ số, drop tables, banner rates, pity rules, event configs.

\subsubsection{Yêu cầu mô hình dữ liệu mức logic}

\begin{itemize}
    \item \textbf{DR-01} -- USER có settings nhúng (1:1).
    \item \textbf{DR-02} -- USER có nhiều GAME\_PROFILE (1:N).
    \item \textbf{DR-03} -- GAME\_PROFILE sở hữu nhiều CHARACTER (1:N).
    \item \textbf{DR-04} -- CHARACTER có INVENTORY (1:1).
    \item \textbf{DR-05} -- GAME\_PROFILE có PERSONAL\_ISLAND (1:1) và NPC\_COLLECTION (1:1).
    \item \textbf{DR-06} -- Item instance tham chiếu item config (item\_ref\_id $\rightarrow$ CONFIG\_ITEM.id).
    \item \textbf{DR-07} -- Building instance tham chiếu building config (type/id $\rightarrow$ CONFIG\_BUILDING.id).
\end{itemize}

\subsubsection{Yêu cầu pipeline quản trị game configuration}

\begin{itemize}
    \item \textbf{DR-08} -- Nguồn nhập liệu cấu hình từ Google Sheets theo template thống nhất.
    \item \textbf{DR-09} -- Export TSV và import vào DB theo cơ chế upsert theo khoá id.
    \item \textbf{DR-10} -- Có \textit{config version} để truy vết lần import và đảm bảo nhất quán runtime.
    \item \textbf{DR-11} -- Config Manager nạp cấu hình từ DB khi khởi động và cung cấp API truy xuất theo id.
\end{itemize}

% ===============================================================
\subsection{Tiêu chí chấp nhận ở mức thiết kế}
\label{subsec:req-acceptance}

Một bản thiết kế đáp ứng yêu cầu giai đoạn 1 cần thoả:
\begin{itemize}
    \item Bao phủ đầy đủ các phân hệ đã liệt kê ở \ref{subsec:req-functional}.
    \item Mô hình dữ liệu mức logic thể hiện rõ quan hệ user/profile/character/inventory/island/npc và mối liên kết với config.
    \item Có mô tả rõ về ranh giới client--server, đặc biệt với các thao tác reward/currency/gacha và luồng real-time combat.
    \item Pipeline cấu hình chỉ áp dụng cho game configuration và có cơ chế versioning + log import.
\end{itemize}
