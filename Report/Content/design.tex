\section{Thiết kế hệ thống}

\subsection{Mục tiêu thiết kế}

Mục tiêu của chương này là chuyển hoá các kết quả phân tích hệ thống ở Chương 6 thành
các quyết định thiết kế cụ thể ở mức kiến trúc và mô-đun, làm cơ sở cho việc hiện thực
hệ thống trong Chương 8.

Thiết kế hệ thống trong giai đoạn 1 hướng đến các mục tiêu chính sau:
\begin{itemize}
    \item Xây dựng kiến trúc client--server rõ ràng, tuân theo mô hình server authoritative.
    \item Tổ chức hệ thống theo hướng module hoá để thuận tiện cho mở rộng.
    \item Đảm bảo hỗ trợ đầy đủ các yêu cầu chức năng và phi chức năng đã xác định.
    \item Giữ mức độ phức tạp phù hợp với phạm vi một đồ án chuyên ngành.
\end{itemize}

Thiết kế không nhằm tối ưu hoá cho môi trường production hay tải lớn, mà tập trung
vào tính đúng đắn, rõ ràng và khả năng phát triển trong các giai đoạn tiếp theo.

%------------------------------------------------

\subsection{Kiến trúc tổng thể hệ thống}

\subsubsection{Mô hình kiến trúc client--server}

Hệ thống Fortress of the Fallen trong giai đoạn 1 được thiết kế theo mô hình
client--server nhiều lớp, bao gồm các thành phần chính sau:

\begin{itemize}
    \item \textbf{Game Client (Unity):}
    \begin{itemize}
        \item Hiển thị đồ hoạ và giao diện người dùng.
        \item Thu thập input từ người chơi.
        \item Gửi request HTTP và message WebSocket đến server.
        \item Hiển thị trạng thái nhận được từ server.
    \end{itemize}

    \item \textbf{Backend Server (NestJS/Node.js):}
    \begin{itemize}
        \item Xử lý xác thực và quản lý người dùng.
        \item Quản lý phiên chơi và kết nối real-time.
        \item Xử lý logic gameplay tối thiểu (đồng bộ di chuyển).
        \item Đóng vai trò nguồn chân lý cho trạng thái hệ thống.
    \end{itemize}

    \item \textbf{Data Layer (MongoDB, Redis):}
    \begin{itemize}
        \item MongoDB lưu trữ dữ liệu bền vững (User, Character).
        \item Redis (tuỳ mức hiện thực) hỗ trợ session hoặc cache dữ liệu truy cập thường xuyên.
    \end{itemize}
\end{itemize}

Nguyên tắc thiết kế cốt lõi của hệ thống là \textbf{server authoritative}:
mọi trạng thái có ảnh hưởng đến tính nhất quán và công bằng đều do server xác nhận.

%------------------------------------------------

\subsubsection{Phân tách theo lớp chức năng}

Ở mức logic, hệ thống được phân chia thành các lớp chức năng:

\begin{itemize}
    \item \textbf{Presentation Layer:} Game client Unity.
    \item \textbf{Application Layer:} Các module backend xử lý nghiệp vụ.
    \item \textbf{Real-time Layer:} Gateway WebSocket và logic phiên chơi.
    \item \textbf{Data Layer:} MongoDB và Redis.
\end{itemize}

Cách phân tách này giúp giảm phụ thuộc giữa các thành phần, đồng thời tạo điều kiện
mở rộng hệ thống trong các giai đoạn sau (NPC, dungeon, PvP).

%------------------------------------------------

\subsection{Thiết kế các mô-đun backend}

Backend được tổ chức theo kiến trúc module hoá của NestJS. Trong giai đoạn 1,
các mô-đun chính bao gồm:

\subsubsection{Auth Module}

\textbf{Chức năng:}
\begin{itemize}
    \item Đăng ký tài khoản.
    \item Đăng nhập và cấp token phiên.
    \item Xác thực token cho các request tiếp theo.
\end{itemize}

\textbf{Thiết kế:}
\begin{itemize}
    \item Sử dụng HTTP/REST cho đăng ký và đăng nhập.
    \item Token (ví dụ JWT) được gửi kèm trong header hoặc handshake WebSocket.
    \item Mật khẩu chỉ được lưu dưới dạng hash.
\end{itemize}

\subsubsection{User và Character Module}

\textbf{User Module} chịu trách nhiệm quản lý thông tin tài khoản người chơi và
liên kết với danh sách nhân vật.

\textbf{Character Module} đảm nhiệm:
\begin{itemize}
    \item Tạo nhân vật tối thiểu.
    \item Lưu trữ các thuộc tính nền như level, vị trí khởi tạo và chỉ số cơ bản.
    \item Cung cấp dữ liệu cho phiên chơi real-time.
\end{itemize}

Thiết kế hai module này hướng đến khả năng mở rộng cho các hệ thống class, skill
và progression trong các giai đoạn tiếp theo.

%------------------------------------------------

\subsubsection{Session và Room Module}

Module Session quản lý trạng thái phiên chơi real-time, bao gồm ánh xạ giữa
user, character và socket.

Module Room dùng để nhóm các người chơi vào các instance logic nhằm phục vụ
kiểm thử đồng bộ trạng thái trong giai đoạn 1.

Các session chủ yếu tồn tại trong bộ nhớ runtime và không yêu cầu lưu bền vững.

%------------------------------------------------

\subsubsection{Real-time Gateway}

Gateway WebSocket đảm nhiệm:
\begin{itemize}
    \item Nhận input di chuyển từ client.
    \item Phát snapshot trạng thái về client.
    \item Xử lý kết nối và ngắt kết nối.
\end{itemize}

Input từ client được xem là \emph{ý định}, không phải trạng thái cuối cùng;
server kiểm tra ràng buộc trước khi cập nhật trạng thái authoritative.

%------------------------------------------------

\subsection{Thiết kế giao tiếp client--server}

\subsubsection{Giao tiếp HTTP}

HTTP được sử dụng cho các chức năng không yêu cầu real-time:
\begin{itemize}
    \item Đăng ký, đăng nhập.
    \item Tải hồ sơ người chơi.
    \item Tạo và chọn nhân vật.
\end{itemize}

Cách tiếp cận này giúp đơn giản hoá thiết kế và thuận tiện cho kiểm thử.

\subsubsection{Giao tiếp WebSocket}

WebSocket được sử dụng cho:
\begin{itemize}
    \item Thiết lập phiên chơi.
    \item Gửi input di chuyển.
    \item Nhận cập nhật trạng thái.
\end{itemize}

Mỗi message WebSocket được thiết kế theo dạng message-based, gồm loại sự kiện
và payload dữ liệu.

%------------------------------------------------

\subsection{Thiết kế dữ liệu ở mức logic}

\subsubsection{Đối tượng User}

Đối tượng User bao gồm các thuộc tính logic chính:
\begin{itemize}
    \item userId
    \item username/email
    \item passwordHash
    \item metadata cơ bản
\end{itemize}

User đóng vai trò định danh và liên kết các dữ liệu khác.

\subsubsection{Đối tượng Character}

Đối tượng Character bao gồm:
\begin{itemize}
    \item characterId
    \item ownerUserId
    \item level
    \item baseStats
    \item position (spawn)
\end{itemize}

Thiết kế này sẵn sàng mở rộng cho class, skill và equipment.

\subsubsection{Đối tượng Session (Runtime)}

Session tồn tại trong runtime server, bao gồm:
\begin{itemize}
    \item socketId
    \item userId
    \item characterId
    \item roomId
    \item trạng thái kết nối
\end{itemize}

Session không bắt buộc lưu vào DB trong giai đoạn 1.

%------------------------------------------------

\subsection{Thiết kế luồng xử lý real-time}

Luồng xử lý di chuyển được thiết kế như sau:
\begin{enumerate}
    \item Client gửi input di chuyển theo nhịp.
    \item Server nhận input và đưa vào vòng lặp tick.
    \item Server kiểm tra ràng buộc (tốc độ, spam).
    \item Server cập nhật trạng thái authoritative.
    \item Server gửi snapshot về client.
\end{enumerate}

Ở giai đoạn 1, cơ chế snapshot được triển khai ở mức đơn giản, chưa yêu cầu
prediction và reconciliation hoàn chỉnh.

%------------------------------------------------

\subsection{Các quyết định thiết kế và giới hạn}

Một số quyết định thiết kế quan trọng trong giai đoạn 1:
\begin{itemize}
    \item Chỉ hiện thực đồng bộ di chuyển cơ bản.
    \item Một user duy trì tối đa một session active.
    \item Chưa tối ưu cho quy mô người chơi lớn.
    \item Ưu tiên tính rõ ràng và ổn định hơn tối ưu hiệu năng.
\end{itemize}

Các giới hạn này giúp đảm bảo đồ án phù hợp với nguồn lực và mục tiêu học thuật.

%------------------------------------------------

\subsection{Tổng kết chương}

Chương này đã trình bày thiết kế hệ thống Fortress of the Fallen ở mức kiến trúc
và mô-đun cho giai đoạn 1, bao gồm kiến trúc tổng thể, thiết kế backend, giao tiếp
client--server, dữ liệu logic và luồng xử lý real-time.

Các nội dung trên là cơ sở trực tiếp cho Chương 8 -- Hiện thực hệ thống.
