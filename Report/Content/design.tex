\section{Thiết kế hệ thống}
\label{sec:design}

\subsection{Mục tiêu thiết kế}
\label{subsec:design-goals}

Mục tiêu của chương này là chuyển hoá kết quả phân tích ở Chương~\ref{sec:analysis} thành các quyết định thiết kế cụ thể ở mức:
(i) kiến trúc tổng thể client--server và triển khai; 
(ii) thiết kế mô-đun backend và ranh giới trách nhiệm; 
(iii) thiết kế dữ liệu (progression/state và game configuration);
(iv) thiết kế giao tiếp (HTTP/WebSocket) và quy ước dữ liệu; 
(v) thiết kế các cơ chế đảm bảo nhất quán, bảo mật và quan sát hệ thống.

Trong giai đoạn 1, thiết kế ưu tiên:
\begin{itemize}
    \item \textbf{Tính đúng đắn và nhất quán}: các thay đổi ảnh hưởng tiến trình (reward/currency/gacha/checkpoint) phải được server xác nhận.
    \item \textbf{Tính mô-đun}: phân tách rõ theo phân hệ (auth, profile, character, inventory, combat, gacha, island, admin).
    \item \textbf{Tính data-driven cho cấu hình}: tách dữ liệu cấu hình gameplay khỏi mã nguồn; hỗ trợ quản trị và thay đổi nhanh.
    \item \textbf{Phù hợp phạm vi đồ án}: triển khai ở mức hợp lý (single deployment có thể chạy được), nhưng vẫn giữ được hướng mở rộng.
\end{itemize}

\subsection{Kiến trúc tổng thể và triển khai}
\label{subsec:design-architecture}

\subsubsection{Tổng quan triển khai}
\label{subsubsec:design-deploy-overview}

Hệ thống được tổ chức thành 3 lớp chính: \textit{Client Devices}, \textit{Backend Infrastructure} và \textit{Data \& Storage}. Hình~\ref{fig:deploy-diagram} mô tả các thành phần và đường giao tiếp chính.

\begin{figure}[H]
    \centering
    % Yêu cầu: đặt ảnh vào Report/Images/DeployDiagram.png
    \includegraphics[width=0.95\textwidth]{Images/DeployDiagram.png}
    \caption{Sơ đồ triển khai tổng quan hệ thống}
    \label{fig:deploy-diagram}
\end{figure}

Các thành phần cốt lõi:
\begin{itemize}
    \item \textbf{Game Client (Unity)}: hiển thị đồ hoạ, UI, thu thập input; giao tiếp với server qua HTTPS và WebSocket.
    \item \textbf{API Server (NestJS)}: lớp ứng dụng xử lý REST API, đồng thời cung cấp kênh real-time (WebSocket gateway) cho phiên chơi.
    \item \textbf{Job Worker}: xử lý tác vụ nền theo lịch (ví dụ: mail hệ thống, sự kiện, tổng hợp log, dọn dẹp dữ liệu runtime).
    \item \textbf{MongoDB (Primary DB)}: lưu dữ liệu tiến trình bền vững và các collections cấu hình.
    \item \textbf{Redis (Cache \& Pub/Sub)}: lưu cache/session runtime hoặc pub-sub khi mở rộng nhiều instance.
    \item \textbf{MinIO (Object Storage)}: lưu trữ tệp và asset bundle (nếu áp dụng); client tải qua presigned URL.
\end{itemize}

\subsubsection{Nguyên tắc server-authoritative}
\label{subsubsec:design-server-authoritative}

Thiết kế áp dụng nguyên tắc \textbf{server-authoritative} cho các tác vụ ảnh hưởng tiến trình và công bằng:
\begin{itemize}
    \item Client chỉ gửi \textbf{ý định} (intent), không gửi trạng thái ``đã xảy ra'' như ``đã trúng đòn'' hay ``đã nhận item''.
    \item Server kiểm tra luật (cooldown, tài nguyên, điều kiện tham gia, dung lượng inventory) và chỉ khi hợp lệ mới ghi dữ liệu.
    \item Các giao dịch reward/currency/gacha/checkpoint được coi là \textbf{giao dịch nghiệp vụ} và phải có log/audit để truy vết.
\end{itemize}
Các quyết định này giúp giảm rủi ro gian lận và tăng tính nhất quán trong môi trường online \cite{gaffer-networking,moriarty-networked}.

\subsubsection{Phân tách giao tiếp: HTTP và WebSocket}
\label{subsubsec:design-protocols}

\paragraph{HTTP/REST} được dùng cho:
\begin{itemize}
    \item đăng ký/đăng nhập, quản lý profile/nhân vật;
    \item tải dữ liệu UI (inventory, island state, NPC collection);
    \item thao tác quản trị (admin) và các tác vụ không yêu cầu độ trễ thấp.
\end{itemize}

\paragraph{WebSocket} được dùng cho:
\begin{itemize}
    \item tham gia instance/room và vòng lặp chiến đấu real-time;
    \item gửi input (move/attack/cast/dodge) theo nhịp;
    \item nhận snapshot/event từ server để hiển thị mượt.
\end{itemize}

\subsection{Thiết kế mô-đun backend}
\label{subsec:design-backend-modules}

Backend được tổ chức theo phong cách module hoá của NestJS \cite{nest-docs}. Mỗi mô-đun tương ứng một phân hệ nghiệp vụ và sở hữu ranh giới dữ liệu rõ ràng. Bảng~\ref{tab:module-map} tóm tắt các mô-đun chính.

\begin{table}[H]
\centering
\renewcommand{\arraystretch}{1.2}
\setlength{\tabcolsep}{6pt}
\begin{tabularx}{\textwidth}{|p{3.1cm}|X|p{4.2cm}|}
\hline
\textbf{Mô-đun} & \textbf{Trách nhiệm chính} & \textbf{Dữ liệu/Collection liên quan} \\
\hline
Auth Module & Đăng ký/đăng nhập; cấp và xác thực token; chính sách role & USER, SESSION\_RUNTIME \\
\hline
Profile Module & Tạo/chọn game profile; cập nhật last\_played; thiết lập profile & GAME\_PROFILE \\
\hline
Character Module & Tạo/chọn/xoá nhân vật; quản lý level/EXP; base stats, appearance & CHARACTER \\
\hline
Inventory Module & Quản lý slot, stack, currency; equip/enhance/durability; cập nhật reward & INVENTORY, CONFIG\_ITEM \\
\hline
Combat Module & Tham gia tower/dungeon/arena; quản lý instance/room; vòng lặp tick & SESSION\_RUNTIME, (runtime state) \\
\hline
Gacha Module & Banner, rate, pity, exchange; xử lý duplicate; ghi lịch sử & NPC\_COLLECTION, (gacha history), config rates \\
\hline
Island Module & Tài nguyên đảo; xây/nâng cấp công trình; timer; layout & PERSONAL\_ISLAND, CONFIG\_BUILDING \\
\hline
NPC Module & Bộ sưu tập NPC; levelling/star/friendship; gán worker & NPC\_COLLECTION \\
\hline
Admin Module & Ban/unban; system mail; log viewer; event config & USER, (mail/event/log) \\
\hline
Config Module & Nạp và cung cấp truy xuất cấu hình; versioning; cache & CONFIG\_* (static) \\
\hline
\end{tabularx}
\caption{Bản đồ mô-đun backend và dữ liệu liên quan}
\label{tab:module-map}
\end{table}

\subsubsection{Các thành phần dùng chung (cross-cutting)}
\label{subsubsec:design-crosscutting}

Để giảm trùng lặp và tăng tính nhất quán, hệ thống dùng các thành phần dùng chung:
\begin{itemize}
    \item \textbf{Validation}: kiểm tra input cho REST và WebSocket (schema/DTO) để chặn dữ liệu sai từ đầu.
    \item \textbf{Error handling}: chuẩn hoá mã lỗi và thông điệp lỗi; đảm bảo client có thể hiển thị đúng trạng thái UI (loading/error/disabled).
    \item \textbf{Logging \& audit}: log request, log nghiệp vụ (reward/gacha/admin action), kèm correlation id.
    \item \textbf{Rate limit / anti-spam}: giới hạn tần suất với endpoint nhạy cảm (login, gacha) và message real-time (input spam).
\end{itemize}

\subsection{Thiết kế dữ liệu}
\label{subsec:design-data}

\subsubsection{Nguyên tắc thiết kế dữ liệu}
\label{subsubsec:design-data-principles}

Hệ thống dữ liệu được chia thành 2 nhóm:
\begin{itemize}
    \item \textbf{Progression/State (động)}: dữ liệu người chơi thay đổi theo thời gian (profile, character, inventory, island, NPC, pity).
    \item \textbf{Game Configuration (tĩnh)}: dữ liệu định nghĩa và thông số gameplay (item/building/skill/rates/formulas).
\end{itemize}

Với MongoDB, thiết kế ưu tiên:
\begin{itemize}
    \item \textbf{Embed} khi dữ liệu có vòng đời gắn chặt với thực thể cha và truy vấn thường đi cùng nhau (ví dụ: base\_stats/resources/appearance nhúng trong CHARACTER; items nhúng trong INVENTORY; buildings nhúng trong PERSONAL\_ISLAND).
    \item \textbf{Reference} khi dữ liệu là định nghĩa tĩnh dùng chung (CONFIG\_ITEM/CONFIG\_BUILDING) hoặc cần tái sử dụng giữa nhiều thực thể.
\end{itemize}

\subsubsection{Mô hình thực thể chính (ERD mức logic)}
\label{subsubsec:design-erd}

Trong phạm vi báo cáo, ERD được trình bày ở mức logic để làm rõ quan hệ 1:1, 1:N và các liên kết tham chiếu tới cấu hình. Mã mô tả ERD (Mermaid) có thể dùng để render thành hình khi cần:

\begin{lstlisting}
erDiagram
    USER ||--|{ GAME_PROFILE : "has (1:N)"
    GAME_PROFILE ||--|{ CHARACTER : "owns (1:N)"
    CHARACTER ||--|| INVENTORY : "has (1:1)"
    GAME_PROFILE ||--|| PERSONAL_ISLAND : "owns (1:1)"
    GAME_PROFILE ||--|| NPC_COLLECTION : "owns (1:1)"
    INVENTORY ||--|{ STRUCT_INVENTORY_ITEM : "contains"
    STRUCT_INVENTORY_ITEM }|..|| CONFIG_ITEM : "ref config"
    PERSONAL_ISLAND ||--|{ STRUCT_BUILDING : "contains"
    STRUCT_BUILDING }|..|| CONFIG_BUILDING : "ref config"
\end{lstlisting}

Các thực thể trọng tâm:
\begin{itemize}
    \item \textbf{USER}: định danh tài khoản; nhúng \texttt{OBJECT\_USER\_SETTINGS}.
    \item \textbf{GAME\_PROFILE}: tách tiến trình theo profile; là ``root'' cho character/island/npc.
    \item \textbf{SESSION\_RUNTIME}: trạng thái phiên hoạt động (token, socket, profile/character đang dùng).
    \item \textbf{CHARACTER}: dữ liệu nhân vật và trạng thái cơ bản.
    \item \textbf{INVENTORY}: danh sách item instance + currency.
    \item \textbf{PERSONAL\_ISLAND}: tài nguyên đảo + công trình theo grid.
    \item \textbf{NPC\_COLLECTION}: danh sách NPC sở hữu; phục vụ tuyển dụng và worker assignment.
\end{itemize}

\subsubsection{Chỉ mục (indexes) và khoá truy vấn}
\label{subsubsec:design-indexes}

Để đảm bảo truy vấn ổn định, các chỉ mục đề xuất:
\begin{itemize}
    \item USER: unique index cho \texttt{username}, \texttt{email}.
    \item GAME\_PROFILE: index theo \texttt{user\_id}; có thể unique theo (\texttt{user\_id}, \texttt{profile\_name}).
    \item CHARACTER: index theo \texttt{profile\_id}; có thể unique theo (\texttt{profile\_id}, \texttt{name}) nếu cần chống trùng trong profile.
    \item INVENTORY: unique index theo \texttt{character\_id}.
    \item PERSONAL\_ISLAND, NPC\_COLLECTION: unique index theo \texttt{profile\_id}.
    \item CONFIG collections: unique index theo \texttt{id} (khoá cấu hình).
\end{itemize}

\subsection{Thiết kế hệ thống chỉ số và tăng trưởng nhân vật}
\label{subsec:design-stats}

\subsubsection{Thuộc tính nền và chỉ số suy diễn}
\label{subsubsec:design-primary-derived}

Nhân vật có 6 thuộc tính nền: STR, DEX, CON, INT, WIS, CHA. Từ đó hệ thống suy diễn các chỉ số chiến đấu như HP/MP/ATK/DEF/CRIT/ACC/EVA. Hình~\ref{fig:stats-system} minh hoạ mối phụ thuộc giữa các nhóm chỉ số.

\begin{figure}[H]
    \centering
    % Yêu cầu: đặt ảnh vào Report/Images/stats-system.png
    \includegraphics[width=0.78\textwidth]{Images/stats-system.png}
    \caption{Mô hình thuộc tính nền và chỉ số suy diễn}
    \label{fig:stats-system}
\end{figure}

Thiết kế tách rõ:
\begin{itemize}
    \item \textbf{Base Stats}: điểm phân bổ trực tiếp bởi người chơi theo level (+5 points/level, tối đa level 100).
    \item \textbf{Derived Stats}: chỉ số tính toán (HP/MP/ATK/\ldots) dựa trên công thức và hệ số cân bằng.
    \item \textbf{Special Traits}: Karma, Affinity, Luck, Resistance là nhóm ``đặc tính'' dùng cho điều kiện mở khoá và/hoặc hiệu ứng meta.
\end{itemize}

\subsubsection{Tính toán chỉ số theo hướng data-driven}
\label{subsubsec:design-stat-formula}

Để tránh hard-code công thức, thiết kế cho phép cấu hình hoá hệ số (ví dụ: HP nhận trọng số từ CON và STR; MP nhận trọng số từ WIS và INT; SPD từ DEX). Cách tiếp cận:
\begin{itemize}
    \item Công thức được lưu dưới dạng \textbf{config} (ví dụ bảng hệ số theo stat).
    \item Server tính toán derived stats khi:
    \begin{itemize}
        \item player phân bổ điểm;
        \item thay đổi trang bị;
        \item thay đổi class/race có modifier.
    \end{itemize}
\end{itemize}

Ở mức thiết kế, có thể biểu diễn công thức bằng dạng tuyến tính:
\[
    S_{derived} = \sum_i w_i \cdot A_i + b
\]
trong đó $A_i$ là thuộc tính nền (STR/DEX/CON/INT/WIS/CHA) và $w_i$ là hệ số cấu hình.

\subsection{Thiết kế quản trị cấu hình game (Config Pipeline)}
\label{subsec:design-config-pipeline}

Pipeline cấu hình chỉ áp dụng cho \textbf{các bảng cấu hình/thông số gameplay} (item/building/skill/rates/formulas). Dữ liệu tiến trình người chơi vẫn được ghi trực tiếp bởi server trong các collections động.

\subsubsection{Luồng dữ liệu cấu hình}
\label{subsubsec:design-config-flow}

Luồng quản trị cấu hình đề xuất:
\begin{enumerate}
    \item \textbf{Soạn thảo trên Google Sheets}: mỗi bảng là một nhóm cấu hình (CONFIG\_ITEM, CONFIG\_BUILDING, rates, formulas).
    \item \textbf{Xuất TSV}: xuất theo template thống nhất; đảm bảo kiểu dữ liệu và khoá \texttt{id}.
    \item \textbf{Import vào DB}: script import thực hiện \textit{upsert} theo \texttt{id}; tạo bản ghi version/import log.
    \item \textbf{Nạp vào Config Manager}: khi server khởi động (hoặc khi client vào game tuỳ chính sách), server tải config từ DB vào bộ nhớ để truy xuất nhanh.
\end{enumerate}

\subsubsection{Thiết kế Config Manager}
\label{subsubsec:design-config-manager}

Config Manager là lớp truy xuất cấu hình thống nhất cho toàn bộ service:
\begin{itemize}
    \item \textbf{In-memory cache}: lưu map \texttt{id $\rightarrow$ config object} cho từng nhóm cấu hình.
    \item \textbf{Versioning}: gắn \texttt{config\_version} cho lần nạp, phục vụ debug và đảm bảo nhất quán khi tính toán.
    \item \textbf{Validation khi nạp}: kiểm tra dữ liệu TSV sau import (schema, khoá trùng, tham chiếu hợp lệ).
\end{itemize}

Ở giai đoạn 1, cơ chế reload có thể ở mức đơn giản (restart server để nạp lại). Khi mở rộng, có thể bổ sung:
\begin{itemize}
    \item endpoint admin ``reload config'' có xác thực;
    \item publish sự kiện qua Redis pub/sub để các instance nạp lại đồng bộ.
\end{itemize}

\subsection{Thiết kế giao tiếp client--server}
\label{subsec:design-communication}

\subsubsection{REST API: nhóm endpoint và quy ước}
\label{subsubsec:design-rest}

REST API được nhóm theo phân hệ. Mục tiêu thiết kế API:
\begin{itemize}
    \item tài nguyên rõ ràng (resource-oriented);
    \item payload ổn định cho UI;
    \item lỗi có mã và thông điệp nhất quán.
\end{itemize}

Ví dụ nhóm endpoint (mức thiết kế):
\begin{itemize}
    \item \texttt{/auth/register}, \texttt{/auth/login}, \texttt{/auth/recover}
    \item \texttt{/profiles} (list/create/select)
    \item \texttt{/characters} (list/create/delete/select)
    \item \texttt{/inventory} (get/update equip/enhance)
    \item \texttt{/island} (get/build/upgrade/assign-worker/collect)
    \item \texttt{/gacha} (banners/rates/summon/history/exchange)
    \item \texttt{/admin} (users/ban/mail/logs/events)
\end{itemize}

\subsubsection{WebSocket: định dạng message và vòng đời phiên}
\label{subsubsec:design-ws}

Thiết kế message theo dạng envelope để dễ mở rộng:
\begin{lstlisting}[language=json]
{
  "type": "combat.input",
  "seq": 1024,
  "ts": 1730000000,
  "payload": { ... }
}
\end{lstlisting}

Quy ước chính:
\begin{itemize}
    \item \textbf{type}: loại thông điệp (join, input, snapshot, event, error).
    \item \textbf{seq}: số thứ tự để hỗ trợ kiểm soát spam, debug và (nếu cần) reconciliation.
    \item \textbf{payload}: dữ liệu tuỳ theo type.
\end{itemize}

Vòng đời phiên:
\begin{enumerate}
    \item Client đăng nhập và chọn profile/character qua REST.
    \item Client handshake WebSocket kèm token; server xác thực và gán \texttt{socket\_id}.
    \item Client gửi \texttt{join} để vào instance/room; server trả dữ liệu spawn và state ban đầu.
    \item Trong phiên: client gửi input theo nhịp; server phát snapshot/event.
    \item Kết phiên: server trả kết quả, reward delta và trạng thái checkpoint.
\end{enumerate}

\subsection{Thiết kế vòng lặp real-time cho combat/instance}
\label{subsec:design-realtime}

\subsubsection{Mô hình instance/room}
\label{subsubsec:design-room}

Hệ thống tổ chức người chơi theo \textbf{room} (đơn vị logic của một instance). Mỗi room có:
\begin{itemize}
    \item danh sách người chơi tham gia (socket id, character id);
    \item trạng thái runtime (vị trí, HP/MP, trạng thái kỹ năng, entity trong phòng);
    \item cấu hình phiên (mode: tower/dungeon/arena; floor/seed).
\end{itemize}

\subsubsection{Tick loop và phát snapshot}
\label{subsubsec:design-tick}

Thiết kế loop ở mức khái niệm:
\begin{enumerate}
    \item Thu thập input queue từ client.
    \item Xác thực input (cooldown, tài nguyên, vị trí hợp lệ, tốc độ).
    \item Cập nhật trạng thái authoritative theo tick.
    \item Phát snapshot định kỳ và phát event khi có thay đổi quan trọng (hit/loot/death).
\end{enumerate}

Trong giai đoạn 1, cơ chế snapshot có thể thiết kế theo hướng \textbf{đơn giản và rõ ràng}:
\begin{itemize}
    \item snapshot theo tick rate cố định (ví dụ 10--20 tick/s tuỳ mục tiêu kiểm thử);
    \item payload snapshot ưu tiên các giá trị tối thiểu để render (pos, hp/mp, state flags);
    \item chưa bắt buộc triển khai đầy đủ prediction/reconciliation, nhưng định dạng message chừa chỗ cho mở rộng \cite{gaffer-networking,moriarty-networked}.
\end{itemize}

\subsection{Thiết kế phân hệ Administration}
\label{subsec:design-admin}

Phân hệ admin phục vụ quản trị vận hành và kiểm soát rủi ro:
\begin{itemize}
    \item \textbf{Ban/Unban user}: cập nhật \texttt{USER.status}; thu hồi phiên đang hoạt động.
    \item \textbf{System mail}: gửi thông báo/phần thưởng theo user/profile; được xử lý như job để kiểm soát retry.
    \item \textbf{View logs}: truy vấn log theo thời gian/actor/severity; hỗ trợ điều tra sự cố.
    \item \textbf{Configure events}: tạo cấu hình sự kiện và phần thưởng; dữ liệu sự kiện tách khỏi config item cơ bản.
\end{itemize}

Các thao tác admin phải có:
\begin{itemize}
    \item phân quyền theo role;
    \item audit log đầy đủ (ai làm, làm gì, khi nào, trước/sau);
    \item cơ chế chống thao tác lặp gây cấp trùng phần thưởng (idempotent key).
\end{itemize}

\subsection{Bảo mật, nhất quán và quan sát hệ thống}
\label{subsec:design-nfr}

\subsubsection{Bảo mật}
\label{subsubsec:design-security}

Các nguyên tắc:
\begin{itemize}
    \item mật khẩu lưu dạng hash; không lưu plaintext;
    \item token có thời hạn; khi ban user cần thu hồi session;
    \item phân quyền rõ giữa player và admin (role-based);
    \item giới hạn tần suất login/gacha và kiểm tra input realtime để chống spam.
\end{itemize}

\subsubsection{Nhất quán dữ liệu và giao dịch nghiệp vụ}
\label{subsubsec:design-consistency}

Các thao tác như reward, gacha, exchange, nâng cấp công trình cần đảm bảo:
\begin{itemize}
    \item \textbf{atomic ở mức tài liệu} (MongoDB document update) khi cập nhật trong cùng thực thể (inventory/items/currency);
    \item \textbf{idempotency} cho request dễ bị gửi lại (retry do mạng);
    \item \textbf{log nghiệp vụ} để có thể đối soát và khôi phục khi lỗi.
\end{itemize}

\subsubsection{Quan sát hệ thống (observability)}
\label{subsubsec:design-observability}

Thiết kế log tối thiểu:
\begin{itemize}
    \item request log: endpoint, latency, status code, user id (nếu có);
    \item business log: gacha result, reward delivery, currency delta, admin actions;
    \item realtime log: join/leave room, disconnect, lỗi xác thực socket.
\end{itemize}

\subsection{Tổng kết chương} 
\label{subsec:design-summary}

Chương này đã mô tả thiết kế hệ thống ở mức kiến trúc, mô-đun, dữ liệu và giao tiếp cho \textit{Fortress of the Fallen}. Thiết kế nhấn mạnh ranh giới trách nhiệm client--server theo hướng server-authoritative, tổ chức backend theo mô-đun rõ ràng, tách dữ liệu tiến trình khỏi dữ liệu cấu hình, và xây dựng pipeline quản trị cấu hình theo hướng data-driven. Các quyết định thiết kế này là nền tảng để mô tả hiện thực hệ thống và kiểm thử trong các chương tiếp theo \cite{nest-docs,mongodb-guide,redis-in-action,gaffer-networking,moriarty-networked}.
