\section{Công nghệ sử dụng}

Chương này trình bày các công nghệ và nền tảng phần mềm được sử dụng hoặc định hướng sử dụng để hiện thực hoá một trò chơi Action RPG trực tuyến nhiều người chơi. Thay vì chỉ liệt kê tên công nghệ, nội dung tập trung phân tích vai trò, ưu điểm, hạn chế và lý do các công nghệ này phù hợp với bối cảnh một hệ thống game online thời gian thực\cite{gregory-engine,moriarty-networked}.

Về tổng thể, một hệ thống game online có thể được chia thành các lớp chính:
\begin{itemize}
    \item Lớp \textbf{client/game engine}: hiển thị đồ họa, nhận input và xử lý trải nghiệm người chơi.
    \item Lớp \textbf{backend}: xử lý logic game phía server, xác thực, quản lý người chơi, phiên chơi và đồng bộ trạng thái.
    \item Lớp \textbf{dữ liệu}: lưu trữ lâu dài tiến trình người chơi, cấu hình game và các dữ liệu phụ trợ.
    \item Lớp \textbf{hạ tầng và công cụ}: phục vụ phát triển, kiểm thử, triển khai và giám sát hệ thống.
\end{itemize}

% ===============================================================
\subsection{Kiến trúc tổng quát nhiều lớp}

Một trò chơi Action RPG online thường được triển khai theo kiến trúc nhiều lớp (multi-tier)\cite{gregory-engine}:

\begin{itemize}
    \item \textbf{Lớp trình bày (Presentation Layer)}: game client chạy trên máy người chơi (PC, console hoặc mobile), được xây dựng bằng game engine (ví dụ: Unity). Lớp này chịu trách nhiệm kết xuất hình ảnh, xử lý input, hiển thị giao diện và phản hồi tương tác.
    \item \textbf{Lớp dịch vụ ứng dụng (Application/Backend Layer)}: tập hợp các dịch vụ server xử lý logic game, API, đồng bộ trạng thái, match-making, quản lý phiên chơi và các nghiệp vụ liên quan.
    \item \textbf{Lớp dữ liệu (Data Layer)}: bao gồm các cơ sở dữ liệu quan hệ hoặc NoSQL, hệ thống cache, kho lưu trữ file cấu hình, log, v.v.
\end{itemize}

Việc tách bạch các lớp này giúp:
\begin{itemize}
    \item Dễ \textbf{mở rộng} (scale) từng lớp độc lập.
    \item Thuận tiện \textbf{thay thế công nghệ} ở một lớp mà không ảnh hưởng toàn hệ thống.
    \item Tăng \textbf{tính bảo trì} và \textbf{khả năng tái sử dụng} các thành phần.
\end{itemize}

% ===============================================================
\subsection{Công nghệ phía client (Game Engine)}

\subsubsection{Game engine cho Action RPG}

Để phát triển nhanh một trò chơi Action RPG có đồ họa 2D/3D, các game engine phổ biến như Unity hoặc Unreal Engine thường được sử dụng. Trong phạm vi đề tài, Unity được lựa chọn vì\cite{unity-manual}:

\begin{itemize}
    \item Cung cấp sẵn hệ thống \textbf{rendering}, animation, vật lý cơ bản và quản lý scene.
    \item Có \textbf{asset pipeline} tương đối hoàn chỉnh (nhập mô hình 3D, texture, animation từ các công cụ bên ngoài).
    \item Hỗ trợ tốt việc xây dựng \textbf{UI in-game}, hệ thống input đa nền tảng và xử lý đa phương tiện.
    \item Sở hữu \textbf{cộng đồng lớn}, tài liệu phong phú, nhiều plugin và asset sẵn có, phù hợp với điều kiện tự học và tra cứu.
\end{itemize}

Việc sử dụng Unity giúp nhóm phát triển tập trung hơn vào thiết kế gameplay, cơ chế mạng và kiến trúc hệ thống, thay vì phải tự xây dựng từ đầu các thành phần nền tảng phức tạp như renderer hay physics engine\cite{gregory-engine}.

\subsubsection{Ngôn ngữ lập trình phía client}

Với Unity, ngôn ngữ lập trình chủ đạo là \textbf{C\#}\cite{csharp-lang}. Một số đặc điểm nổi bật:

\begin{itemize}
    \item Cú pháp hiện đại, hỗ trợ tốt lập trình hướng đối tượng.
    \item Thư viện phong phú cho xử lý cấu trúc dữ liệu, toán học, I/O, networking ở mức client.
    \item Tích hợp chặt chẽ với API của Unity, giúp hiện thực logic gameplay, UI và hiệu ứng một cách trực tiếp\cite{unity-manual}.
\end{itemize}

Khi lập trình phía client, cần lưu ý:

\begin{itemize}
    \item Tách biệt \textbf{logic gameplay cục bộ} (UI, hiệu ứng hiển thị, animation) với \textbf{logic được server quyết định} (kết quả combat, trạng thái chính thức của nhân vật).
    \item Hạn chế lưu trữ và xử lý \textbf{dữ liệu nhạy cảm} phía client để giảm nguy cơ gian lận.
    \item Tổ chức mã nguồn theo hướng \textbf{component-based} hoặc \textbf{entity--component--system (ECS)} để dễ mở rộng và bảo trì\cite{gregory-engine}.
\end{itemize}

% ===============================================================
\subsection{Công nghệ phía server (Backend)}

Backend là nơi xử lý logic game phía server, đóng vai trò “nguồn chân lý” (source of truth) cho trạng thái thế giới game\cite{moriarty-networked}. Một backend cho game online thường bao gồm:

\begin{itemize}
    \item Dịch vụ xác thực và quản lý người dùng (authentication, account management).
    \item Dịch vụ quản lý phiên chơi (session, room, instance).
    \item Dịch vụ xử lý logic game (chiến đấu, nhiệm vụ, sự kiện).
    \item Dịch vụ truy xuất và cập nhật dữ liệu (progress nhân vật, cấu hình game).
\end{itemize}

\subsubsection{Lựa chọn nền tảng backend: Node.js và NestJS}

Trong phạm vi đề tài, backend được định hướng xây dựng bằng \textbf{Node.js} kết hợp với \textbf{NestJS} và \textbf{TypeScript}\cite{nodejs-docs,nest-docs}. Lý do lựa chọn:

\begin{itemize}
    \item Node.js hỗ trợ tốt \textbf{lập trình bất đồng bộ}, phù hợp với hệ thống có nhiều kết nối WebSocket đồng thời.
    \item NestJS cung cấp \textbf{kiến trúc mô-đun rõ ràng}, dựa trên decorator và dependency injection, giúp tổ chức mã nguồn backend khoa học và dễ mở rộng.
    \item TypeScript giúp mã nguồn \textbf{có kiểu tường minh}, hỗ trợ kiểm tra lỗi tại thời điểm biên dịch, giảm rủi ro trong phát triển trung và dài hạn.
    \item Hệ sinh thái phong phú: nhiều thư viện cho REST API, WebSocket, ORM, JWT, phân quyền, logging, v.v.
\end{itemize}

\subsubsection{Mô hình dịch vụ và tổ chức logic backend}

Về kiến trúc, backend có thể triển khai theo\cite{moriarty-networked}:
\begin{itemize}
    \item \textbf{Monolithic}: toàn bộ logic và API tập trung trong một ứng dụng NestJS duy nhất.
    \item \textbf{Module hoá}: bên trong ứng dụng monolithic, chia nhỏ thành các module như User, Auth, Character, Session, v.v., sẵn sàng để tách thành microservices khi hệ thống mở rộng.
\end{itemize}

% ===============================================================
\subsection{Công nghệ networking và giao tiếp thời gian thực}

\subsubsection{Giao tiếp client--server}

Game online thường sử dụng kết hợp hai cơ chế giao tiếp\cite{moriarty-networked}:

\begin{itemize}
    \item \textbf{HTTP/REST}:
    \begin{itemize}
        \item Phù hợp với các tác vụ không yêu cầu độ trễ thấp, như: đăng ký, đăng nhập, tải cấu hình, truy vấn thông tin tĩnh.
        \item Trong NestJS, có thể hiện thực nhanh bằng các controller REST tiêu chuẩn\cite{nest-docs}.
    \end{itemize}
    \item \textbf{WebSocket}:
    \begin{itemize}
        \item Phù hợp với truyền thông hai chiều thời gian thực: cập nhật vị trí, trạng thái chiến đấu, sự kiện trong thế giới game.
        \item Giảm overhead so với việc liên tục mở và đóng kết nối HTTP.
        \item NestJS cung cấp sẵn hỗ trợ WebSocket gateway, giúp tích hợp với client Unity thông qua các thư viện WebSocket tương ứng\cite{nest-docs}.
    \end{itemize}
\end{itemize}

\subsubsection{Quản lý kết nối và phiên chơi}

Một backend real-time cần\cite{moriarty-networked}:

\begin{itemize}
    \item Quản lý \textbf{phiên kết nối WebSocket} cho từng người chơi (mapping giữa user ID và socket).
    \item Ánh xạ người chơi vào \textbf{phòng/instance} tương ứng (ví dụ: khu vực chung, dungeon, map thử nghiệm).
    \item Xử lý ngắt kết nối, timeout và cơ chế reconnect khi người chơi mất kết nối tạm thời.
\end{itemize}

% ===============================================================
\subsection{Công nghệ lưu trữ và cơ sở dữ liệu}

\subsubsection{Cơ sở dữ liệu NoSQL (MongoDB)}

Trong phạm vi đề tài, cơ sở dữ liệu chính được định hướng sử dụng là \textbf{MongoDB}\cite{mongodb-guide}. Lý do:

\begin{itemize}
    \item Dữ liệu game có cấu trúc linh hoạt (nhân vật, vật phẩm, trạng thái nhiệm vụ, cấu hình NPC, cấu hình map).
    \item Schema thường thay đổi trong quá trình phát triển, NoSQL cho phép thích ứng dễ dàng hơn so với các hệ quan hệ.
    \item Hỗ trợ tốt document lồng nhau, phù hợp với các cấu trúc dữ liệu phức hợp của nhân vật và cấu hình game.
\end{itemize}

\subsubsection{Hệ thống cache (Redis)}

\textbf{Redis} được sử dụng ở vai trò\cite{redis-in-action}:

\begin{itemize}
    \item \textbf{Cache} cho dữ liệu truy cập thường xuyên (thông tin người chơi đang online, token, một phần cấu hình game).
    \item \textbf{Session store}: lưu token phiên đăng nhập và thông tin phiên làm việc.
    \item \textbf{Pub/Sub}: truyền sự kiện giữa nhiều instance backend trong trường hợp cần mở rộng theo chiều ngang.
\end{itemize}

% ===============================================================
\subsection{Công cụ hỗ trợ phát triển và vận hành}

\subsubsection{Quản lý mã nguồn và làm việc nhóm}

\begin{itemize}
    \item \textbf{Git}: hệ thống quản lý phiên bản phân tán, hỗ trợ làm việc nhóm trên cùng một codebase.
    \item \textbf{GitHub}: nền tảng lưu trữ mã nguồn, hỗ trợ issue tracking, pull request và tích hợp CI/CD\cite{github-docs}.
    \item \textbf{GitHub Projects}: hệ thống quản lý tác vụ dựa trên bảng Kanban, cho phép phân chia, gán trách nhiệm, theo dõi tiến độ và liên kết tác vụ với issue/pull request tương ứng\cite{github-projects}.
\end{itemize}

\subsubsection{Công cụ thiết kế và tài liệu}

\begin{itemize}
    \item \textbf{Công cụ vẽ UML/BPMN} (Draw.io, Mermaid): dùng để mô tả use case, luồng xử lý, kiến trúc hệ thống\cite{uml-bpmn}.
    \item \textbf{Công cụ thiết kế giao diện} (Figma, Piskel, Stable Diffusion): dùng để phác thảo giao diện và trải nghiệm người dùng trước khi hiện thực trên Unity.
    \item \textbf{Công cụ quản lý tài liệu} (TeX, Markdown): dùng để viết Game Design Document (GDD), tài liệu thiết kế hệ thống và báo cáo học thuật\cite{latex-companion}.
\end{itemize}

% ===============================================================
\subsection{Tổng kết}

Chương này đã trình bày các lớp công nghệ chính liên quan đến việc xây dựng một trò chơi Action RPG trực tuyến: game engine phía client (Unity), nền tảng backend xử lý logic (Node.js, NestJS, TypeScript), cơ sở dữ liệu NoSQL (MongoDB) và hệ thống cache (Redis), cùng các công cụ hỗ trợ phát triển và vận hành (Git, GitHub, GitHub Projects, các công cụ thiết kế và tài liệu). Việc hiểu rõ vai trò và đặc điểm của từng công nghệ giúp hình thành một kiến trúc tổng thể hợp lý, là nền tảng cho các quyết định thiết kế chi tiết ở các chương tiếp theo, đặc biệt là phân tích yêu cầu (Chương~5) và thiết kế hệ thống (Chương~7).
