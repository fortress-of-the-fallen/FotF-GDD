\section{Công nghệ sử dụng}

Chương này trình bày các công nghệ và nền tảng phần mềm được sử dụng hoặc \textbf{định hướng sử dụng} trong đồ án để xây dựng một trò chơi Action RPG trực tuyến. Nội dung tập trung vào vai trò của từng công nghệ trong kiến trúc tổng thể, các ưu điểm/\allowbreak hạn chế và lý do phù hợp với bối cảnh hệ thống game online thời gian thực\cite{gregory-engine,moriarty-networked}.

Về mặt chức năng, hệ thống có thể chia thành các lớp chính:
\begin{itemize}
    \item \textbf{Client/\allowbreak Game Engine}: hiển thị đồ hoạ, nhận input, mô phỏng cục bộ (visual/\allowbreak animation), UI và giao tiếp mạng.
    \item \textbf{Backend/\allowbreak Application Server}: xử lý nghiệp vụ game, xác thực, quản lý phiên, đồng bộ trạng thái và cung cấp API.
    \item \textbf{Data \& Storage}: lưu trữ lâu dài (progress người chơi), cache/\allowbreak pub-sub, và kho dữ liệu cấu hình/\allowbreak tài nguyên.
    \item \textbf{Tooling \& Process}: quản lý mã nguồn, quản lý tác vụ, tài liệu hoá và công cụ thiết kế.
\end{itemize}

% ===============================================================
\subsection{Công nghệ phía client}

\subsubsection{Game engine: Unity}

Unity là lựa chọn phù hợp để hiện thực client trong bối cảnh đồ án nhờ hệ sinh thái hoàn chỉnh và tài liệu phong phú\cite{unity-manual}. Các lý do chính:
\begin{itemize}
    \item \textbf{Pipeline đồ hoạ và nội dung}: hỗ trợ sprite/\allowbreak animation, quản lý scene, prefab và asset import giúp tăng tốc phát triển nội dung.
    \item \textbf{UI framework}: cung cấp hệ thống UI phục vụ HUD chiến đấu, inventory, menu và các màn hình meta-progression.
    \item \textbf{Khả năng mở rộng}: cho phép tổ chức mã theo hướng component-based; có thể mở rộng sang hướng ECS tuỳ mức độ phức tạp\cite{gregory-engine}.
\end{itemize}

Trong giai đoạn 1, Unity được xem là nền tảng client nhằm hỗ trợ \textbf{thiết kế trải nghiệm} và \textbf{định nghĩa cấu trúc giao tiếp} với backend; chưa đặt trọng tâm vào tối ưu hiệu năng production.

\subsubsection{Ngôn ngữ lập trình: C\#}

C\# là ngôn ngữ chính khi phát triển với Unity\cite{csharp-lang}. Các đặc điểm phù hợp:
\begin{itemize}
    \item Hỗ trợ tốt lập trình hướng đối tượng và tổ chức hệ thống gameplay theo mô-đun.
    \item Có thư viện tiêu chuẩn mạnh cho cấu trúc dữ liệu, toán học và xử lý bất đồng bộ ở mức client.
    \item Tích hợp trực tiếp với API Unity để điều khiển animation, UI và xử lý input\cite{unity-manual}.
\end{itemize}

\subsubsection{Nguyên tắc xử lý logic ở client}

Đối với game online, cần tách bạch rõ giữa:
\begin{itemize}
    \item \textbf{Client logic (hiển thị)}: UI, animation, hiệu ứng, nội suy chuyển động và dự đoán phản hồi để giảm cảm giác trễ.
    \item \textbf{Server logic (quyết định)}: các kết quả quan trọng như tính sát thương, xác nhận loot, cập nhật tài nguyên và thay đổi trạng thái chính thức\cite{moriarty-networked}.
\end{itemize}
Nguyên tắc này giúp giảm rủi ro gian lận và đảm bảo tính nhất quán khi nhiều người chơi tương tác.

% ===============================================================
\subsection{Công nghệ phía server}

\subsubsection{Nền tảng runtime: Node.js}

Node.js phù hợp với backend game online nhờ mô hình I/\allowbreak O bất đồng bộ, thuận lợi khi phải xử lý đồng thời nhiều kết nối mạng (đặc biệt là WebSocket)\cite{nodejs-docs}. Với đặc thù đồ án, Node.js hỗ trợ:
\begin{itemize}
    \item Xây dựng API nhanh, dễ mở rộng theo mô-đun.
    \item Tích hợp tốt với hệ sinh thái thư viện cho xác thực, logging, validation và kết nối DB.
\end{itemize}

\subsubsection{Framework: NestJS và TypeScript}

NestJS cung cấp kiến trúc rõ ràng theo hướng module + dependency injection, phù hợp khi hệ thống cần phân rã theo phân hệ (auth/\allowbreak profile, session, character, inventory, \ldots)\cite{nest-docs}. TypeScript giúp tăng độ an toàn của mã nguồn nhờ kiểu tường minh, giảm lỗi trong phát triển trung hạn.

Các lý do lựa chọn:
\begin{itemize}
    \item \textbf{Tổ chức mã nguồn}: module hoá tự nhiên, dễ quản lý ranh giới nghiệp vụ và tái sử dụng service dùng chung.
    \item \textbf{Hỗ trợ đa giao thức}: REST cho tác vụ quản trị/\allowbreak truy vấn; WebSocket gateway cho tương tác thời gian thực\cite{nest-docs}.
    \item \textbf{Khả năng mở rộng}: có thể bắt đầu theo kiểu monolithic-module để giảm độ phức tạp, và có cơ sở để tách dịch vụ khi hệ thống tăng quy mô.
\end{itemize}

\subsubsection{Tác vụ nền (background jobs)}

Trong game online, một số xử lý không cần chạy trực tiếp trên request realtime, ví dụ: đồng bộ dữ liệu định kỳ, tổng hợp log, xử lý mail hệ thống, hoặc quản lý sự kiện theo lịch. Các tác vụ này có thể được tách thành \textbf{job worker} chạy độc lập với API server để giảm tải cho luồng realtime, đồng thời dễ kiểm soát retry và lịch chạy.

% ===============================================================
\subsection{Công nghệ giao tiếp mạng}

\subsubsection{HTTP/\allowbreak REST}

HTTP phù hợp với các tác vụ không yêu cầu độ trễ thấp hoặc không cần kết nối hai chiều liên tục:
\begin{itemize}
    \item Đăng ký/\allowbreak đăng nhập, chọn profile, lấy thông tin nhân vật.
    \item Tải dữ liệu cấu hình khi khởi động (tuỳ cách triển khai Config Manager).
    \item Các thao tác quản trị (admin tools) và truy vấn lịch sử\cite{nest-docs}.
\end{itemize}

\subsubsection{WebSocket}

WebSocket phù hợp cho các tương tác realtime trong phiên chơi:
\begin{itemize}
    \item Cập nhật trạng thái chiến đấu, sự kiện tức thời trong instance.
    \item Thông báo thay đổi trạng thái (loot, revive, buff/\allowbreak debuff) và đồng bộ theo nhịp tick.
\end{itemize}
Với mô hình server authoritative, WebSocket đóng vai trò kênh truyền nhận input/\allowbreak snapshot, kết hợp các kỹ thuật giảm cảm giác trễ ở client\cite{moriarty-networked,gaffer-networking}.

% ===============================================================
\subsection{Công nghệ lưu trữ dữ liệu}

\subsubsection{MongoDB}

MongoDB phù hợp cho dữ liệu game do đặc tính document linh hoạt, dễ biểu diễn các cấu trúc lồng nhau (inventory items, island buildings, npc collection) và thích nghi khi schema thay đổi trong quá trình phát triển\cite{mongodb-guide}. Trong bối cảnh đồ án:
\begin{itemize}
    \item \textbf{Dữ liệu tiến trình (dynamic)}: user/\allowbreak profile/\allowbreak session/\allowbreak character/\allowbreak inventory/\allowbreak island/\allowbreak npc cần lưu bền vững và truy vấn theo khoá định danh.
    \item \textbf{Dữ liệu cấu hình (static)}: các collection cấu hình như item/\allowbreak building/\allowbreak skill có thể được import từ pipeline TSV và nạp vào bộ nhớ khi chạy.
\end{itemize}

\subsubsection{Redis}

Redis được định hướng sử dụng như lớp hỗ trợ hiệu năng và phối hợp realtime\cite{redis-in-action}:
\begin{itemize}
    \item \textbf{Cache}: giảm tải đọc DB cho dữ liệu truy cập thường xuyên (ví dụ session mapping, một phần config hot, trạng thái online).
    \item \textbf{Pub/\allowbreak Sub}: chia sẻ sự kiện giữa nhiều tiến trình/\allowbreak instance backend khi mở rộng theo chiều ngang.
\end{itemize}

\subsubsection{Object Storage (tuỳ chọn theo giai đoạn)}

Một số dữ liệu dạng file (asset pack, snapshot log, hoặc các tệp cấu hình đóng gói) có thể lưu trong object storage theo chuẩn S3-compatible. Trong giai đoạn 1, phần này chỉ dừng ở mức định hướng kiến trúc để sẵn sàng mở rộng; chưa đặt trọng tâm vận hành production.

% ===============================================================
\subsection{Công cụ quản trị game configuration theo hướng data-driven}

Trong đồ án, \textbf{pipeline cấu hình} chỉ áp dụng cho \textbf{các bảng cấu hình/\allowbreak thông số game} (ví dụ: định nghĩa item/\allowbreak building/\allowbreak skill và các hệ số cân bằng), không áp dụng cho toàn bộ dữ liệu tiến trình người chơi.

Định hướng quy trình:
\begin{itemize}
    \item Thiết kế bảng cấu hình bằng \textbf{Google Sheets} để thuận lợi nhập liệu và review.
    \item Export sang \textbf{TSV} nhằm đảm bảo format đơn giản, dễ parse và thân thiện khi version control.
    \item Import TSV vào các \textbf{collections cấu hình} trong DB (tách biệt với collections progress).
    \item Khi khởi động hoặc khi tải game, hệ thống nạp config từ DB vào \textbf{Config Manager} (in-memory) để truy xuất thống nhất trong runtime.
\end{itemize}

Lợi ích chính của cách tiếp cận data-driven:
\begin{itemize}
    \item Giảm phụ thuộc vào hard-code tham số, hỗ trợ điều chỉnh cân bằng nhanh.
    \item Tăng khả năng kiểm soát thay đổi: TSV có thể được quản lý theo commit để truy vết phiên bản cấu hình.
    \item Tạo tiền đề cho mở rộng nội dung và kiểm thử A/\allowbreak B cấu hình trong tương lai.
\end{itemize}

% ===============================================================
\subsection{Công cụ hỗ trợ phát triển và tài liệu hoá}

\subsubsection{Quản lý mã nguồn và phối hợp nhóm}

Git/\allowbreak GitHub được sử dụng để quản lý phiên bản mã nguồn và phối hợp theo pull request; GitHub Projects dùng để quản lý tác vụ theo bảng, gán trách nhiệm và theo dõi tiến độ\cite{github-docs,github-projects}.

\subsubsection{Công cụ mô hình hoá và tài liệu}

Các sơ đồ use case/\allowbreak luồng nghiệp vụ/\allowbreak kiến trúc có thể được xây dựng bằng công cụ UML/\allowbreak BPMN hoặc Mermaid để đảm bảo tính nhất quán khi trình bày\cite{uml-bpmn}. Báo cáo và tài liệu hệ thống được soạn thảo bằng \LaTeX\ nhằm đảm bảo định dạng học thuật và quản lý nội dung theo từng chương độc lập\cite{latex-companion}.

% ===============================================================
\subsection{Tổng kết}

Chương này đã trình bày các công nghệ chính được sử dụng hoặc định hướng sử dụng cho đồ án: Unity và C\# ở phía client\cite{unity-manual,csharp-lang}; Node.js, NestJS và TypeScript ở phía server\cite{nodejs-docs,nest-docs}; MongoDB và Redis cho lớp dữ liệu\cite{mongodb-guide,redis-in-action}; cùng các công cụ hỗ trợ phát triển và tài liệu hoá\cite{github-docs,github-projects,latex-companion}. Ngoài ra, chương cũng nêu định hướng pipeline quản trị \textbf{game configuration} theo hướng data-driven nhằm phục vụ thiết kế và mở rộng nội dung trong các giai đoạn tiếp theo.
