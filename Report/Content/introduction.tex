\section{Giới thiệu}

\subsection{Động lực}

Trong bối cảnh ngành công nghiệp game phát triển mạnh mẽ, các trò chơi nhập vai trực tuyến nhiều người chơi (Multiplayer Online RPG) ngày càng trở nên phổ biến. Người chơi không chỉ kỳ vọng vào đồ họa đẹp mắt, nội dung hấp dẫn mà còn yêu cầu trải nghiệm trực tuyến ổn định, mượt mà, có khả năng hỗ trợ nhiều người chơi tương tác với nhau trong thời gian thực.

Để đáp ứng các yêu cầu này, hệ thống phía server cần có khả năng xử lý lượng lớn kết nối đồng thời, đồng bộ trạng thái nhân vật, quản lý tài nguyên và dữ liệu trò chơi một cách hiệu quả, đồng thời vẫn đảm bảo độ trễ thấp. Việc nghiên cứu, thiết kế và hiện thực một hệ thống game online hoàn chỉnh đòi hỏi sinh viên phải kết hợp nhiều mảng kiến thức: lập trình game, mạng máy tính, kiến trúc client–server, cơ sở dữ liệu, cũng như thiết kế cơ chế gameplay.

Đề tài ``Fortress of the Fallen'' được lựa chọn với mục tiêu vừa thỏa mãn niềm yêu thích đối với thể loại Action RPG Online, vừa là cơ hội để nhóm sinh viên rèn luyện kỹ năng phân tích, thiết kế và xây dựng một hệ thống phần mềm phức tạp với nhiều thành phần tương tác.

\subsection{Mục tiêu}

Mục tiêu tổng quát của đề tài là thiết kế và phát triển hệ thống nền tảng cho game online nhập vai hành động nhiều người chơi ``Fortress of the Fallen''. Cụ thể, trong giai đoạn 1, đề tài hướng đến các mục tiêu sau:

\begin{itemize}
    \item Nghiên cứu và tổng hợp ý tưởng thiết kế game từ các tác phẩm tham khảo (tiểu thuyết và game), từ đó định hình bối cảnh, cơ chế chơi và hệ thống tính năng cốt lõi của trò chơi.
    \item Xác định phạm vi, tính khả thi kỹ thuật và mức độ phù hợp của đề tài trong khuôn khổ đồ án chuyên ngành.
    \item Thiết kế mô hình kiến trúc hệ thống, bao gồm các thành phần: client, server backend, cơ sở dữ liệu và cơ chế giao tiếp mạng.
    \item Hiện thực bộ khung ban đầu (source base) cho hệ thống, sử dụng Unity cho client và NestJS kết hợp .NET cho backend, làm nền tảng để phát triển tiếp ở các giai đoạn sau.
\end{itemize}

\subsection{Phạm vi}

Trong phạm vi giai đoạn 1 của đồ án chuyên ngành, đề tài tập trung vào việc xây dựng \textit{prototype nền tảng} hơn là hoàn thiện toàn bộ nội dung game. Cụ thể, phạm vi thực hiện bao gồm:

\begin{itemize}
    \item Thiết kế kiến trúc tổng thể cho hệ thống game online nhiều người chơi.
    \item Thiết kế và xây dựng mô hình dữ liệu cơ bản phục vụ cho quản lý tài khoản, nhân vật, đảo cá nhân và NPC.
    \item Xây dựng các chức năng cơ bản trên client:
    \begin{itemize}
        \item Giao diện đăng ký, đăng nhập.
        \item Kết nối đến server, xác thực người dùng.
        \item Vào đảo cá nhân và hiển thị nhân vật chính.
        \item Điều khiển di chuyển nhân vật trên đảo.
    \end{itemize}
    \item Xây dựng các chức năng nền tảng trên server:
    \begin{itemize}
        \item Dịch vụ xác thực (authentication) và quản lý người dùng.
        \item Quản lý phiên làm việc và kết nối real-time.
        \item Cơ chế gacha NPC cơ bản và lưu trữ NPC vào cơ sở dữ liệu.
        \item Cài đặt AI đơn giản cho NPC (ví dụ: làm việc định kỳ để tạo tài nguyên).
    \end{itemize}
\end{itemize}

Các nội dung nâng cao như hệ thống leo tháp 100 tầng, 18 tầng ngục, hệ thống class–race–skill hoàn chỉnh, PvP Arena, cân bằng game và tối ưu hiệu năng nâng cao sẽ được xem xét ở các giai đoạn tiếp theo và nằm ngoài phạm vi hiện tại.

\subsection{Ý nghĩa của đề tài}

\subsubsection{Ý nghĩa thực tiễn}

Về mặt thực tiễn, đề tài giúp:

\begin{itemize}
    \item Cung cấp một ví dụ cụ thể về quá trình xây dựng hệ thống game online nhiều người chơi từ bước ý tưởng đến kiến trúc và hiện thực.
    \item Minh họa cách kết hợp các công nghệ phổ biến trong phát triển game và backend như Unity, NestJS, .NET, MongoDB, Redis, \dots
    \item Làm nền tảng để mở rộng thành một sản phẩm game hoàn chỉnh hoặc một bộ khung tham khảo cho các đồ án, dự án nghiên cứu khác liên quan đến game online.
\end{itemize}

\subsubsection{Ý nghĩa khoa học và học thuật}

Về mặt học thuật, đề tài mang lại các giá trị sau:

\begin{itemize}
    \item Ứng dụng kiến thức đã học về cấu trúc dữ liệu và giải thuật, lập trình hướng đối tượng, cơ sở dữ liệu, mạng máy tính và kiến trúc phần mềm vào một bài toán thực tế.
    \item Tạo điều kiện cho sinh viên tiếp cận mô hình kiến trúc phân tán, xử lý kết nối đồng thời và bài toán đồng bộ trạng thái trong môi trường real-time.
    \item Góp phần xây dựng tài liệu tham khảo nội bộ về quy trình phân tích, thiết kế và hiện thực một hệ thống game online nhiều người chơi trong khuôn khổ chương trình Khoa học Máy tính.
\end{itemize}

\subsection{Cấu trúc báo cáo}

Phần còn lại của báo cáo được tổ chức như sau:

\begin{itemize}
    \item \textbf{Chương 2 – Kiến thức nền tảng:} Trình bày các khái niệm, định nghĩa và kiến thức liên quan đến game Action RPG, hệ thống multiplayer, kiến trúc client–server và các cơ chế gameplay liên quan.
    \item \textbf{Chương 3 – Công nghệ sử dụng:} Giới thiệu các công nghệ chính được sử dụng trong đề tài như Unity, NestJS, .NET, MongoDB, Redis, cùng lý do lựa chọn.
    \item \textbf{Chương 4 – Các công trình liên quan:} Phân tích các tác phẩm và trò chơi tham khảo, các nghiên cứu về game networking và kiến trúc server real-time, từ đó rút ra bài học áp dụng cho đề tài.
    \item \textbf{Chương 5 – Phân tích yêu cầu:} Xác định người dùng hệ thống, yêu cầu chức năng, phi chức năng và yêu cầu dữ liệu cho giai đoạn 1.
    \item \textbf{Chương 6 – Phân tích hệ thống:} Mô tả quy trình nghiệp vụ, đặc tả use case và luồng dữ liệu logic trong hệ thống.
    \item \textbf{Chương 7 – Thiết kế hệ thống:} Trình bày kiến trúc tổng thể, thiết kế cơ sở dữ liệu, thiết kế API và các mô-đun chính của hệ thống.
    \item \textbf{Chương 8 – Hiện thực hệ thống:} Mô tả quá trình cài đặt các thành phần client và server, cơ chế giao tiếp và tích hợp.
    \item \textbf{Chương 9 – Đánh giá hệ thống:} Đánh giá mức độ hoàn thành so với mục tiêu, kiểm thử các chức năng chính và thảo luận các hạn chế.
    \item \textbf{Chương 10 – Kết luận:} Tổng kết kết quả đạt được, nêu các hạn chế và định hướng phát triển trong các giai đoạn tiếp theo.
\end{itemize}
