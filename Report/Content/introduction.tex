\section{Giới thiệu}

\subsection{Động lực}

Trong những năm gần đây, thị trường game trực tuyến tiếp tục tăng trưởng và xu hướng người chơi chuyển dịch mạnh sang các sản phẩm có tính tương tác và cập nhật liên tục\cite{newzoo-report}. Đặc biệt, nhóm trò chơi nhập vai hành động trực tuyến (Online Action RPG) đặt ra yêu cầu cao cả về trải nghiệm thời gian thực (real-time) lẫn tiến trình dài hạn (meta-progression). Người chơi kỳ vọng hệ thống vận hành ổn định, đồng bộ trạng thái mượt mà, phản hồi nhanh và đảm bảo tính nhất quán khi tương tác với nhiều phân hệ gameplay trong cùng một tài khoản.

Để đáp ứng các kỳ vọng đó, hệ thống phía server không chỉ xử lý xác thực và lưu trữ dữ liệu, mà còn phải đảm nhiệm các bài toán đặc thù của game online như:
\begin{itemize}
    \item Quản lý phiên chơi và xử lý nhiều kết nối đồng thời.
    \item Đồng bộ trạng thái theo mô hình phù hợp để giảm sai lệch giữa client và server, đồng thời hạn chế gian lận ở mức kiến trúc\cite{moriarty-networked}.
    \item Thiết kế dữ liệu tiến trình người chơi (profile/\allowbreak character/\allowbreak inventory/\allowbreak island/\allowbreak npc) sao cho có thể mở rộng và truy vết.
    \item Tổ chức nhiều phân hệ nghiệp vụ (combat, progression, gacha, base-building, admin) trong một kiến trúc nhất quán, giảm phụ thuộc chéo và thuận lợi bảo trì\cite{gregory-engine}.
\end{itemize}

Ngoài ra, một thách thức quan trọng trong thiết kế game là quản trị \textbf{game configuration} (dữ liệu tĩnh) như định nghĩa vật phẩm, công trình, tham số cân bằng, bảng tỉ lệ và cấu hình sự kiện. Nếu cấu hình bị gắn chặt vào mã nguồn, việc điều chỉnh nội dung và cân bằng sẽ phụ thuộc vào chu kỳ build và phát hành. Do đó, hướng tiếp cận \textit{data-driven} thường được áp dụng để tách cấu hình khỏi code, giúp quy trình cập nhật linh hoạt hơn và giảm rủi ro ``hard-code'' tham số\cite{adams-game-design,schell-art}.

Trong bối cảnh đó, đề tài \textit{Fortress of the Fallen} được thực hiện với trọng tâm là phân tích và thiết kế hệ thống cho một game Online Action RPG. Báo cáo hướng đến mô tả đầy đủ các phân hệ cốt lõi ở mức đồ án chuyên ngành, làm nền tảng cho các giai đoạn hiện thực tiếp theo.

\subsection{Mục tiêu}

Mục tiêu tổng quát của đề tài là xây dựng \textbf{bản thiết kế hệ thống} cho trò chơi Online Action RPG \textit{Fortress of the Fallen}, bao gồm kiến trúc, mô hình dữ liệu, luồng nghiệp vụ và hợp đồng giao tiếp giữa các thành phần.

Trong \textbf{giai đoạn 1}, đề tài tập trung vào các mục tiêu cụ thể:
\begin{itemize}
    \item Xác định phạm vi và mô tả tập tính năng ở mức hệ thống theo các phân hệ: Authentication \& Profile, Combat \& Gameplay, Character Progression, Recruitment (Gacha), Personal Island, và Administration.
    \item Phân tích yêu cầu chức năng, phi chức năng và yêu cầu dữ liệu; làm rõ ranh giới giữa dữ liệu tiến trình người chơi và dữ liệu cấu hình tĩnh.
    \item Thiết kế kiến trúc triển khai tổng thể theo mô hình client--server, bao gồm các thành phần backend và hạ tầng dữ liệu; xác định cơ chế giao tiếp phù hợp cho tác vụ request--response và real-time\cite{moriarty-networked,gregory-engine}.
    \item Thiết kế mô hình dữ liệu mức logic cho các thực thể chính: user, profile, session, character, inventory, island, npc và quan hệ giữa chúng, phục vụ truy vấn và mở rộng tính năng về sau.
    \item Thiết kế pipeline quản trị \textbf{game configuration (tĩnh)} theo hướng data-driven:
    \begin{center}
        Google Sheets $\rightarrow$ TSV $\rightarrow$ Import vào DB (collections cấu hình) $\rightarrow$ Load vào \textit{Config Manager} khi khởi động.
    \end{center}
    Pipeline này \textbf{chỉ áp dụng cho game configuration} \\
    Ví dụ: CONFIG\_ITEM, CONFIG\_SKILL, CONFIG\_ENEMY và các bảng cấu hình mở rộng, không áp dụng cho dữ liệu tiến trình người chơi.
\end{itemize}

\subsection{Phạm vi}

\subsubsection{Phạm vi trong giai đoạn 1}

Giai đoạn 1 tập trung hoàn toàn vào \textbf{phân tích và thiết kế}, bao gồm:

\begin{itemize}
    \item \textbf{Thiết kế phân hệ và luồng nghiệp vụ:}
    mô tả actors, use case, luồng tổng quát cho các tác vụ chính (đăng nhập/\allowbreak chọn profile/\allowbreak chọn nhân vật; tham gia tower/\allowbreak dungeon/\allowbreak arena; loot và cập nhật inventory; nâng cấp/\allowbreak equip; gacha và pity; island xây dựng và worker; thao tác admin).
    \item \textbf{Thiết kế kiến trúc triển khai:}
    phân tách vai trò client và server; tổ chức backend thành các thành phần (API server, WebSocket gateway, worker) và tích hợp hạ tầng dữ liệu (DB, cache, object storage) ở mức kiến trúc.
    \item \textbf{Thiết kế kiến trúc phần mềm theo module:}
    phân rã module theo phân hệ, xác định dịch vụ dùng chung (validation, resource, RNG, inventory rules, sync/\allowbreak audit) nhằm giảm trùng lặp và đảm bảo nhất quán nghiệp vụ.
    \item \textbf{Thiết kế dữ liệu mức logic:}
    xây dựng mô hình dữ liệu cho tiến trình người chơi (dynamic) và mô hình cho cấu hình game (static), làm rõ điểm tham chiếu giữa item instance và item definition, giữa building instance và building definition.
    \item \textbf{Thiết kế hợp đồng giao tiếp:}
    định hướng nhóm REST API cho các thao tác không real-time và nhóm WebSocket events cho các hành vi trong phiên chơi.
    \item \textbf{Thiết kế pipeline cấu hình tĩnh:}
    mô tả quy trình export/\allowbreak import TSV và cơ chế nạp cấu hình vào Config Manager khi khởi động để phục vụ truy xuất thống nhất.
\end{itemize}

\subsubsection{Ngoài phạm vi giai đoạn 1}

Các nội dung sau \textbf{không thuộc phạm vi giai đoạn 1} do không phù hợp mục tiêu thiết kế ở mức đồ án:

\begin{itemize}
    \item Hiện thực gameplay hoàn chỉnh (combat runtime đầy đủ, tối ưu điều khiển, animation/\allowbreak FX pipeline).
    \item Hoàn thiện PvP ở mức sản phẩm (matchmaking nâng cao, chống gian lận toàn diện, xếp hạng theo mùa với hệ thống phần thưởng vận hành dài hạn).
    \item Live-ops và monetization thương mại; cân bằng kinh tế game ở quy mô lớn.
    \item Benchmark tải lớn, tối ưu hiệu năng production, và triển khai đa vùng (multi-region) ở mức vận hành thực tế.
\end{itemize}

\subsection{Ý nghĩa của đề tài}

\subsubsection{Ý nghĩa thực tiễn}

Đề tài cung cấp một bản thiết kế hệ thống có tính mô-đun cho game Online Action RPG, làm rõ cách tổ chức các phân hệ gameplay và cách quản trị dữ liệu nhất quán trong một sản phẩm. Đặc biệt, cơ chế quản trị \textit{game configuration} theo hướng data-driven tạo nền tảng cho việc mở rộng nội dung và cân bằng ở các giai đoạn sau mà không phụ thuộc hoàn toàn vào chỉnh sửa mã nguồn\cite{adams-game-design,gregory-engine}.

\subsubsection{Ý nghĩa khoa học và học thuật}

Đề tài tạo môi trường để vận dụng tổng hợp kiến thức về kiến trúc phần mềm, hệ thống phân tán, cơ sở dữ liệu, và nguyên lý đồng bộ trạng thái trong game online. Việc mô hình hoá nghiệp vụ và dữ liệu ở mức logic giúp tăng tính hệ thống, đồng thời cung cấp cơ sở để đánh giá tính khả thi và rủi ro khi chuyển sang giai đoạn triển khai\cite{moriarty-networked,gregory-engine}.

\subsection{Cấu trúc báo cáo}

Báo cáo được tổ chức như sau:

\begin{itemize}
    \item \textbf{Chương 2 -- Kiến thức nền tảng:} Trình bày các khái niệm nền tảng về Action RPG online, core loop/\allowbreak meta loop, nguyên tắc client--server và hướng tiếp cận data-driven cho cấu hình\cite{adams-game-design,schell-art}.
    \item \textbf{Chương 3 -- Công nghệ sử dụng:} Giới thiệu các công nghệ dự kiến sử dụng và lý do lựa chọn trong bối cảnh kiến trúc hệ thống (engine, backend, networking, database, công cụ quản lý)\cite{unity-manual,nest-docs,mongodb-guide,redis-in-action}.
    \item \textbf{Chương 4 -- Các công trình liên quan:} Tổng hợp tham khảo theo nhóm phân hệ và bảng đối chiếu để rút ra các điểm áp dụng vào thiết kế.
    \item \textbf{Chương 5 -- Phân tích yêu cầu:} Xác định yêu cầu chức năng, phi chức năng, dữ liệu và phạm vi giai đoạn 1.
    \item \textbf{Chương 6 -- Phân tích hệ thống:} Phân tích actors, use case, luồng nghiệp vụ và dữ liệu mức logic phục vụ thiết kế.
    \item \textbf{Chương 7 -- Thiết kế hệ thống:} Trình bày kiến trúc triển khai, kiến trúc module, mô hình dữ liệu và hợp đồng giao tiếp; mô tả pipeline quản trị game configuration.
    \item \textbf{Chương 8 -- Định hướng hiện thực:} Đề xuất lộ trình triển khai theo module, chiến lược tích hợp pipeline cấu hình và các điểm cần ưu tiên giảm rủi ro.
    \item \textbf{Chương 9 -- Đánh giá hệ thống:} Đánh giá mức độ bao phủ yêu cầu, tính nhất quán của thiết kế và các rủi ro còn tồn tại.
    \item \textbf{Chương 10 -- Kết luận:} Tổng kết kết quả thiết kế đạt được và định hướng phát triển cho các giai đoạn tiếp theo.
\end{itemize}
