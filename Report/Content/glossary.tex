\section*{Danh mục thuật ngữ}
\addcontentsline{toc}{section}{Danh mục thuật ngữ}

\noindent
Bảng dưới đây liệt kê các thuật ngữ và viết tắt thường được sử dụng trong báo cáo, đặc biệt liên quan tới phát triển game online, kiến trúc client--server, hệ thống AI NPC và cơ sở dữ liệu. Các thuật ngữ này sẽ được dùng xuyên suốt các chương sau.

\vspace{0.5cm}

\renewcommand{\arraystretch}{1.3}

\setlength{\LTpre}{0pt}
\setlength{\LTpost}{0pt}

% Tăng khả năng cứu ngắt dòng
\setlength{\emergencystretch}{2em}

\sloppy
\begin{longtable}{|L{3cm}|L{4cm}|L{8cm}|}
\hline
\textbf{Thuật ngữ} & \textbf{Tên đầy đủ} & \textbf{Giải thích} \\
\hline
\endfirsthead

\hline
\textbf{Thuật ngữ} & \textbf{Tên đầy đủ} & \textbf{Giải thích} \\
\hline
\endhead

\hline
\multicolumn{3}{r|}{(còn tiếp trang sau)}\\
\hline
\endfoot

\hline
\endlastfoot

% ================== NHÓM GAME & GAMEPLAY ==================
\multicolumn{3}{|l|}{\textbf{Nhóm thuật ngữ về game và gameplay}} \\ \hline

RPG & Role-Playing Game &
Game nhập vai, trong đó người chơi điều khiển một hoặc nhiều nhân vật, phát triển sức mạnh thông qua hệ thống level, chỉ số, kỹ năng, trang bị và cốt truyện. \\ \hline

Action RPG & Action Role-Playing Game &
Game nhập vai hành động, chiến đấu diễn ra theo thời gian thực (real-time), người chơi điều khiển trực tiếp nhân vật (tấn công, né, dùng kỹ năng) thay vì theo lượt (turn-based). \\ \hline

PvP & Player vs Player &
Chế độ người chơi chiến đấu trực tiếp với người chơi khác. Yêu cầu đồng bộ trạng thái nhanh, cơ chế xử lý độ trễ và chống gian lận mạnh. \\ \hline

PvE & Player vs Environment &
Chế độ người chơi đối đầu với quái, boss hoặc môi trường do hệ thống điều khiển. Trong \textit{Fortress of the Fallen}, phần lớn nội dung dungeon, tháp, đảo cá nhân ban đầu thiên về PvE. \\ \hline

Boss & Boss &
Kẻ địch đặc biệt có chỉ số cao, cơ chế tấn công phức tạp, thường là trọng tâm của một dungeon hoặc tầng tháp. \\ \hline

Dungeon & Dungeon &
Khu vực (thường là instance riêng) tập trung chiến đấu, vượt thử thách. Trong đề tài, dungeon có thể là các tầng Tinh Hà Trung Tâm hoặc các khu vực phụ. \\ \hline

Hitbox & Hitbox &
Vùng hình học trong game đại diện cho phạm vi đòn tấn công có hiệu lực. \\ \hline

Hurtbox & Hurtbox &
Vùng hình học đại diện cho khu vực trên nhân vật có thể nhận sát thương. \\ \hline

Cooldown (CD) & Cooldown &
Thời gian chờ sau khi sử dụng kỹ năng trước khi kỹ năng đó có thể được sử dụng lại. \\ \hline

Animation & Animation &
Chuỗi hình ảnh mô phỏng chuyển động trong game. \\ \hline

% ================== NHÓM MULTIPLAYER & NETWORKING ==================
\multicolumn{3}{|l|}{\textbf{Nhóm thuật ngữ về multiplayer và networking}} \\ \hline

Multiplayer & Multiplayer &
Hình thức chơi nhiều người trong cùng một thế giới game. \\ \hline

Real-time & Real-time &
Hệ thống xử lý và phản hồi gần như ngay lập tức theo input người chơi. \\ \hline

Latency & Network Latency &
Độ trễ mạng giữa client và server, thường đo bằng mili-giây (ms). \\ \hline

Tick rate & Server Tick Rate &
Số lần server cập nhật trạng thái game trong một giây. \\ \hline

Snapshot & State Snapshot &
Bản chụp trạng thái game tại một thời điểm để đồng bộ client. \\ \hline

Client-side Prediction & Client-side Prediction &
Client dự đoán kết quả hành động trước khi server phản hồi. \\ \hline

Reconciliation & Reconciliation &
Client điều chỉnh trạng thái dự đoán cho khớp với server. \\ \hline

Server Authoritative & Server Authoritative Model &
Server quyết định cuối cùng mọi logic game. \\ \hline

WebSocket & WebSocket &
Giao thức truyền thông hai chiều trên TCP cho thời gian thực. \\ \hline

Instance & Instance &
Phiên bản riêng của một khu vực game cho nhóm người chơi. \\ \hline

Zone / Shard & Zone / Shard &
Phân vùng logic thế giới game để chia tải. \\ \hline

Interest Management & Interest Management &
Chỉ gửi dữ liệu cần thiết cho mỗi client để giảm tải hệ thống. \\ \hline

\end{longtable}
\fussy
