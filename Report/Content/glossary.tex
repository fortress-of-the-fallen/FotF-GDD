\section*{Danh mục thuật ngữ}
\addcontentsline{toc}{section}{Danh mục thuật ngữ}

\noindent
Bảng dưới đây liệt kê các thuật ngữ và viết tắt thường được sử dụng trong báo cáo, đặc biệt liên quan tới phát triển game online, kiến trúc client--server, hệ thống AI NPC và cơ sở dữ liệu. Các thuật ngữ này sẽ được dùng xuyên suốt các chương sau.

\vspace{0.5cm}

\renewcommand{\arraystretch}{1.3}

% Điều chỉnh khoảng cách trên/dưới longtable nếu muốn
\setlength{\LTpre}{0pt}
\setlength{\LTpost}{0pt}

\begin{longtable}{|p{3cm}|p{4cm}|p{8cm}|}
\hline
\textbf{Thuật ngữ} & \textbf{Tên đầy đủ} & \textbf{Giải thích} \\
\hline
\endfirsthead

\hline
\textbf{Thuật ngữ} & \textbf{Tên đầy đủ} & \textbf{Giải thích} \\
\hline
\endhead

\hline
\multicolumn{3}{r|}{(còn tiếp trang sau)}\\
\hline
\endfoot

\hline
\endlastfoot

% ====== NHÓM 1: GAME & GAMEPLAY =======================================
\multicolumn{3}{|l|}{\textbf{Nhóm thuật ngữ về game và gameplay}} \\ \hline

RPG & Role-Playing Game &
Game nhập vai, trong đó người chơi điều khiển một hoặc nhiều nhân vật, phát triển sức mạnh thông qua hệ thống level, chỉ số, kỹ năng, trang bị và cốt truyện. \\ \hline

Action RPG & Action Role-Playing Game &
Game nhập vai hành động, chiến đấu diễn ra theo thời gian thực (real-time), người chơi điều khiển trực tiếp nhân vật (tấn công, né, dùng kỹ năng) thay vì theo lượt (turn-based). \\ \hline

PvP & Player vs Player &
Chế độ người chơi chiến đấu trực tiếp với người chơi khác. Yêu cầu đồng bộ trạng thái nhanh, cơ chế xử lý độ trễ và chống gian lận mạnh. \\ \hline

PvE & Player vs Environment &
Chế độ người chơi đối đầu với quái, boss hoặc môi trường do hệ thống điều khiển. Trong \textit{Fortress of the Fallen}, phần lớn nội dung dungeon, tháp, đảo cá nhân ban đầu thiên về PvE. \\ \hline

Boss & Boss &
Kẻ địch đặc biệt có chỉ số cao, cơ chế tấn công phức tạp, thường là trọng tâm của một dungeon hoặc tầng tháp. \\ \hline

Dungeon & Dungeon &
Khu vực (thường là instance riêng) tập trung chiến đấu, vượt thử thách. Trong đề tài, dungeon có thể là các tầng Tinh Hà Trung Tâm hoặc các khu vực phụ. \\ \hline

Hitbox & Hitbox &
Vùng hình học trong game đại diện cho phạm vi đòn tấn công có hiệu lực (ví dụ: vùng chém của kiếm, vùng nổ của kỹ năng). \\ \hline

Hurtbox & Hurtbox &
Vùng hình học đại diện cho khu vực trên nhân vật có thể nhận sát thương. Va chạm giữa hitbox và hurtbox thường được sử dụng để quyết định trúng đòn. \\ \hline

Cooldown (CD) & Cooldown &
Thời gian chờ sau khi sử dụng kỹ năng trước khi kỹ năng đó có thể được sử dụng lại. \\ \hline

Animation & Animation &
Chuỗi hình ảnh mô phỏng chuyển động (chạy, tấn công, né, trúng đòn...). Trong Action RPG, animation gắn chặt với gameplay. \\ \hline

% ====== NHÓM 2: MULTIPLAYER & NETWORKING ===============================
\multicolumn{3}{|l|}{\textbf{Nhóm thuật ngữ về multiplayer và networking}} \\ \hline

Multiplayer & Multiplayer &
Hình thức chơi nhiều người, cho phép nhiều người chơi cùng tồn tại và tương tác trong một thế giới game chung (ví dụ: cùng ở một đảo, dungeon, tầng tháp). \\ \hline

Real-time & Real-time &
Thời gian thực: hệ thống xử lý và phản hồi gần như ngay lập tức theo input của người chơi. \\ \hline

Latency & Network Latency &
Độ trễ mạng, là thời gian gói tin đi từ client tới server. Thường đo bằng mili-giây (ms). \\ \hline

Tick rate & Server Tick Rate &
Số lần server cập nhật trạng thái game trong một giây (ví dụ: 20~tick/s, 30~tick/s). \\ \hline

Snapshot & State Snapshot &
Bản “ảnh chụp” trạng thái game tại một thời điểm (vị trí nhân vật, HP, trạng thái kỹ năng...), được server gửi cho client để đồng bộ. \\ \hline

Client-side Prediction & Client-side Prediction &
Kỹ thuật client dự đoán trước kết quả chuyển động, hành động dựa trên input người chơi, nhằm giảm cảm giác trễ. \\ \hline

Reconciliation & Reconciliation &
Quá trình đồng bộ lại trên client: so sánh trạng thái dự đoán với trạng thái do server gửi về và điều chỉnh để khớp. \\ \hline

Server Authoritative & Server Authoritative Model &
Mô hình trong đó server nắm quyền quyết định cuối cùng về logic game (di chuyển hợp lệ, sát thương, kết quả combat). \\ \hline

WebSocket & WebSocket &
Giao thức truyền thông hai chiều trên nền TCP, cho phép client và server giữ kết nối lâu dài và gửi/nhận dữ liệu thời gian thực. \\ \hline

Instance & Instance &
Phiên bản riêng biệt của một khu vực game (ví dụ: một dungeon hoặc một tầng tháp tạo riêng cho một nhóm người chơi). \\ \hline

Zone / Shard & Zone / Shard &
Khu vực logic trong thế giới game, dùng để chia tải hoặc phân vùng nội dung. \\ \hline

Interest Management & Interest Management &
Kỹ thuật lọc và chỉ gửi dữ liệu cần thiết cho mỗi client để giảm băng thông và tải xử lý. \\ \hline

% ====== NHÓM 3, 4, 5: bạn copy các dòng còn lại từ bảng cũ vào đây, giữ đúng cấu trúc:
%  - mỗi dòng kết thúc bằng \\ \hline
%  - mỗi tiêu đề nhóm: \multicolumn{3}{|l|}{\textbf{Tên nhóm}} \\ \hline

\end{longtable}
