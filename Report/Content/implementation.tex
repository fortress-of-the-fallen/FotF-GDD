\section{Kế hoạch hiện thực prototype}
\label{sec:implementation}

\subsection{Mục tiêu và phạm vi}
\label{subsec:impl-goals}

Chương này mô tả kế hoạch hiện thực một \textbf{prototype} nhằm kiểm chứng các quyết định thiết kế đã trình bày ở Chương~\ref{sec:design}. Prototype được định hướng như một ``bản chạy được'' để xác thực:
\begin{itemize}
    \item Luồng \textbf{Identity \& Profile}: đăng ký/đăng nhập, chọn profile/nhân vật, thiết lập phiên.
    \item Luồng \textbf{real-time}: client gửi input, server xử lý authoritative và trả snapshot/event.
    \item Các mô-đun meta-progression có dữ liệu bền vững: \textbf{Inventory}, \textbf{Gacha}, \textbf{Personal Island \& NPC}.
    \item Pipeline \textbf{game configuration} theo hướng data-driven: Google Sheets $\rightarrow$ TSV $\rightarrow$ DB $\rightarrow$ Config Manager.
\end{itemize}

Trong phạm vi prototype, các nội dung (map, quái, AI phức tạp, PvP hoàn chỉnh) không phải trọng tâm; ưu tiên là \textbf{đúng luồng}, \textbf{đúng dữ liệu} và \textbf{đúng ranh giới client--server}.

%------------------------------------------------

\subsection{Kế hoạch triển khai theo mốc}
\label{subsec:impl-plan}

Kế hoạch triển khai được tổ chức theo các mốc tăng dần độ phức tạp: trước tiên dựng khung kết nối và xác thực, sau đó bổ sung hệ thống dữ liệu bền vững và cuối cùng là tích hợp các mô-đun meta.

\begin{table}[H]
\centering
\renewcommand{\arraystretch}{1.2}
\setlength{\tabcolsep}{6pt}
\begin{tabularx}{\textwidth}{|p{2.3cm}|X|p{4.2cm}|}
\hline
\textbf{Mốc} & \textbf{Mục tiêu kỹ thuật} & \textbf{Tính năng kiểm chứng} \\
\hline
M1 & Dựng backend, DB, auth và kết nối client & Register/Login, token, chọn profile \\
\hline
M2 & Dựng kênh real-time và room/instance runtime cơ bản & Join room, gửi input, nhận snapshot (di chuyển) \\
\hline
M3 & Tích hợp dữ liệu bền vững và mô-đun meta tối thiểu & Inventory/currency, gacha NPC, island resources, worker assignment \\
\hline
M4 & Hoàn thiện pipeline cấu hình và logging/audit tối thiểu & TSV import, config versioning, business logs \\
\hline
\end{tabularx}
\caption{Kế hoạch hiện thực prototype theo mốc}
\label{tab:impl-milestones}
\end{table}

%------------------------------------------------

\subsection{Hiện thực phía client (Unity)}
\label{subsec:impl-client}

\subsubsection{Tổ chức project và nguyên tắc phân lớp}
\label{subsubsec:impl-client-structure}

Client được tổ chức theo hướng tách biệt rõ ràng:
\begin{itemize}
    \item \textbf{Presentation/UI}: màn hình login, profile/character select, HUD cơ bản, panel inventory/gacha/island.
    \item \textbf{Gameplay}: điều khiển nhân vật, state machine animation, xử lý tương tác cơ bản.
    \item \textbf{Networking}: lớp HTTP client (REST) và WebSocket client (real-time), quản lý token, retry và timeout.
    \item \textbf{Data binding}: model dữ liệu để bind ra UI, tránh UI đọc trực tiếp dữ liệu thô từ network.
\end{itemize}

Nguyên tắc chính:
\begin{itemize}
    \item Client không tự ý ``xác nhận'' các thay đổi tiến trình (reward/currency/gacha).
    \item Client hiển thị trạng thái dựa trên phản hồi server; với realtime, ưu tiên snapshot/event authoritative.
\end{itemize}

\subsubsection{Luồng màn hình đề xuất}
\label{subsubsec:impl-client-flow}

Luồng điều hướng UI tối thiểu cho prototype:
\begin{enumerate}
    \item \textbf{Login/Register}: nhập thông tin, gọi API, lưu token.
    \item \textbf{Profile Select}: liệt kê profile, tạo/chọn profile.
    \item \textbf{Character Select/Create}: liệt kê nhân vật theo profile, tạo/chọn nhân vật.
    \item \textbf{Main Hub (Island)}: load scene, hiển thị HUD, mở các panel inventory/gacha/NPC/island.
    \item \textbf{Join Instance (tuỳ mốc)}: tham gia room realtime để kiểm chứng đồng bộ di chuyển/combat đơn giản.
\end{enumerate}

\subsubsection{Networking client: REST + WebSocket}
\label{subsubsec:impl-client-network}

Client duy trì 2 kênh:
\begin{itemize}
    \item \textbf{REST}: các thao tác quản lý (auth/profile/character/inventory/island/gacha).
    \item \textbf{WebSocket}: join room, gửi input, nhận snapshot/event.
\end{itemize}

Quy ước xử lý lỗi UI:
\begin{itemize}
    \item \textbf{401/403}: token hết hạn hoặc sai quyền $\rightarrow$ điều hướng về login.
    \item \textbf{5xx}: lỗi hệ thống $\rightarrow$ hiển thị trạng thái retry.
    \item \textbf{network timeout}: hiển thị offline/reconnect và không cho thao tác tiến trình.
\end{itemize}

%------------------------------------------------

\subsection{Hiện thực phía server (NestJS)}
\label{subsec:impl-server}

\subsubsection{Cấu trúc module và luồng request}
\label{subsubsec:impl-server-modules}

Backend triển khai theo module hoá để bám sát ranh giới nghiệp vụ:
\begin{itemize}
    \item \textbf{AuthModule}: register/login, issue token, guard/role.
    \item \textbf{ProfileModule}: CRUD profile, chọn profile, cập nhật last\_played.
    \item \textbf{CharacterModule}: CRUD character, load character summary.
    \item \textbf{InventoryModule}: get inventory, apply delta (reward, currency), equip/enhance (tuỳ mốc).
    \item \textbf{GachaModule}: banner/rate/summon, update pity, duplicate handling.
    \item \textbf{IslandModule}: get island, build/upgrade, resource sync.
    \item \textbf{NpcModule}: list NPC, assign worker, update job state.
    \item \textbf{RealtimeGateway}: websocket auth, join room, input handling, snapshot broadcasting.
    \item \textbf{ConfigModule}: load config collections, provide lookup service, expose config version.
\end{itemize}

Luồng xử lý chuẩn cho REST:
\begin{enumerate}
    \item Guard xác thực token và gắn context (userId/profileId/characterId nếu có).
    \item Validate DTO và kiểm tra ràng buộc nghiệp vụ.
    \item Thực hiện cập nhật DB theo hướng atomic ở mức tài liệu (document update).
    \item Ghi business log/audit log nếu là giao dịch quan trọng.
\end{enumerate}

\subsubsection{Realtime: room runtime và tick loop tối thiểu}
\label{subsubsec:impl-server-realtime}

Room runtime được lưu trong bộ nhớ server (in-memory) trong phạm vi prototype:
\begin{itemize}
    \item \texttt{roomId}, \texttt{mode} (island/tower/dungeon/arena), danh sách player trong room.
    \item Runtime state: position, HP/MP, flags trạng thái, cooldown state tối thiểu.
    \item Input queue theo socket.
\end{itemize}

Tick loop tối thiểu (không tối ưu hoá) theo các bước:
\begin{enumerate}
    \item Nhận input và đưa vào queue theo \texttt{seq}.
    \item Validate input (rate limit, trạng thái hợp lệ, không vượt tốc độ).
    \item Cập nhật state và tạo snapshot.
    \item Broadcast snapshot theo nhịp (tick rate cấu hình).
\end{enumerate}

Trong mốc đầu, snapshot chỉ cần chứa thông tin đủ để render:
\[
\{\;entityId,\;pos,\;rot,\;hp,\;mp,\;stateFlags\;\}
\]
Các kỹ thuật nâng cao (prediction/reconciliation) được xem là hướng mở rộng sau khi luồng cơ bản ổn định.

%------------------------------------------------

\subsection{Thiết kế hiện thực dữ liệu và truy xuất}
\label{subsec:impl-data}

\subsubsection{Schema collections và nguyên tắc embed/reference}
\label{subsubsec:impl-data-schema}

Các collections động (progression/state) ưu tiên embed khi dữ liệu gắn chặt vòng đời:
\begin{itemize}
    \item \textbf{CHARACTER}: nhúng base\_stats/resources/position/appearance.
    \item \textbf{INVENTORY}: nhúng danh sách item instance.
    \item \textbf{PERSONAL\_ISLAND}: nhúng island resources và danh sách buildings.
    \item \textbf{NPC\_COLLECTION}: nhúng danh sách owned NPC.
\end{itemize}

Các collections cấu hình (static) dùng reference qua khoá \texttt{id}:
\begin{itemize}
    \item \textbf{CONFIG\_ITEM}: định nghĩa item.
    \item \textbf{CONFIG\_BUILDING}: định nghĩa building và cost/time.
    \item (mở rộng) \textbf{CONFIG\_SKILL}, \textbf{CONFIG\_RATES}, \textbf{CONFIG\_FORMULAS}, \ldots
\end{itemize}

\subsubsection{Nguyên tắc cập nhật an toàn cho giao dịch tiến trình}
\label{subsubsec:impl-data-safety}

Các thao tác nhạy cảm (reward/currency/gacha/build upgrade) cần:
\begin{itemize}
    \item \textbf{Idempotency key}: mỗi request có mã định danh giao dịch để tránh cấp trùng khi retry.
    \item \textbf{Business log}: ghi lại delta trước/sau (hoặc sự kiện) để đối soát.
    \item \textbf{Atomic update}: với dữ liệu trong một document, ưu tiên update trong một thao tác DB.
\end{itemize}

%------------------------------------------------

\subsection{Hiện thực pipeline game configuration}
\label{subsec:impl-config}

\subsubsection{Định dạng TSV và quy ước}
\label{subsubsec:impl-config-tsv}

Mỗi TSV đại diện một bảng cấu hình, có cột \texttt{id} làm khoá chính. Quy ước:
\begin{itemize}
    \item Header dòng đầu tiên; encoding UTF-8.
    \item Kiểu dữ liệu rõ ràng (int/float/string/bool) theo template.
    \item Các tham chiếu chéo dùng khoá \texttt{id} (ví dụ itemRefId, buildingId).
\end{itemize}

\subsubsection{Import script: upsert và log phiên bản}
\label{subsubsec:impl-config-import}

Script import thực hiện:
\begin{enumerate}
    \item Parse TSV theo schema (validate required fields, type).
    \item Upsert theo \texttt{id} vào collection tương ứng.
    \item Tạo bản ghi \texttt{CONFIG\_IMPORT\_LOG} gồm: thời gian import, bảng, số dòng, checksum, người thực hiện.
    \item Cập nhật \texttt{config\_version} (tăng dần) để runtime biết phiên bản đang chạy.
\end{enumerate}

\subsubsection{Config Manager runtime}
\label{subsubsec:impl-config-manager}

Config Manager nạp dữ liệu cấu hình từ DB vào bộ nhớ khi server khởi động:
\begin{itemize}
    \item Lưu theo map: \texttt{(tableName, id) $\rightarrow$ object}.
    \item Cung cấp API nội bộ: \texttt{getItem(id)}, \texttt{getBuilding(id)}, \texttt{getFormula(key)}.
    \item Khi trả dữ liệu về client, có thể kèm \texttt{config\_version} để debug nhất quán.
\end{itemize}

%------------------------------------------------

\subsection{Kế hoạch kiểm thử và tiêu chí nghiệm thu}
\label{subsec:impl-testing}

\subsubsection{Kiểm thử theo use case}
\label{subsubsec:impl-uc-test}

Kiểm thử ưu tiên theo các use case trọng yếu (Chương~\ref{sec:analysis}):
\begin{itemize}
    \item Auth: register/login, token expiry.
    \item Profile/Character: create/select, load summary.
    \item Realtime: join room, move input, snapshot update, disconnect handling.
    \item Gacha: summon 1x/10x, pity update, duplicate handling, history.
    \item Island/NPC: build/upgrade timer (mock), assign worker, resource tick (mô phỏng).
\end{itemize}

\subsubsection{Kiểm thử dữ liệu và tính nhất quán}
\label{subsubsec:impl-consistency-test}

Các kịch bản cần kiểm chứng:
\begin{itemize}
    \item Retry request (gacha/reward) không làm nhân đôi kết quả.
    \item Inventory stack và slot không vượt giới hạn.
    \item Worker assignment không tạo trạng thái treo (NPC vừa ``rảnh'' vừa ``đang làm'').
    \item Import TSV sai schema bị từ chối và không làm ``bẩn'' config.
\end{itemize}

\subsubsection{Tiêu chí nghiệm thu prototype}
\label{subsubsec:impl-acceptance}

Prototype đạt yêu cầu khi thoả tối thiểu:
\begin{itemize}
    \item Có thể đăng nhập, chọn profile/nhân vật và vào scene hub.
    \item Có thể tham gia room realtime và đồng bộ di chuyển ổn định.
    \item Gacha trả kết quả hợp lệ, lưu DB, cập nhật pity và xử lý trùng lặp.
    \item Island/NPC có luồng dữ liệu bền vững (lưu và tải lại đúng).
    \item Pipeline TSV import tạo được config version và runtime truy xuất đúng theo version.
\end{itemize}

%------------------------------------------------

\subsection{Tổng kết chương}
\label{subsec:impl-summary}

Chương này đã trình bày kế hoạch hiện thực prototype nhằm kiểm chứng thiết kế hệ thống: tổ chức client và server, luồng REST/WebSocket, room realtime tối thiểu, chiến lược lưu trữ dữ liệu, pipeline game configuration theo TSV và kế hoạch kiểm thử theo use case. Kết quả của chương là cơ sở để thực hiện đánh giá/đối chiếu trong chương tiếp theo khi prototype được triển khai và chạy thử.
