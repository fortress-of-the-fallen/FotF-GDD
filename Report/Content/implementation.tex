\section{Hiện thực hệ thống}

\subsection{Mục tiêu hiện thực}

Mục tiêu của chương này là mô tả quá trình hiện thực các thành phần của hệ thống
Fortress of the Fallen trong giai đoạn 1, dựa trên thiết kế đã trình bày ở Chương 7.
Nội dung tập trung vào cách tổ chức các thành phần client và server, cơ chế tích hợp
giữa chúng, cũng như phạm vi hiện thực cụ thể của prototype.

Việc hiện thực trong giai đoạn 1 không nhằm tạo ra một sản phẩm game hoàn chỉnh,
mà đóng vai trò kiểm chứng các quyết định kiến trúc, công nghệ và luồng vận hành
của hệ thống.

%------------------------------------------------

\subsection{Môi trường và công cụ triển khai}

\subsubsection{Môi trường phát triển}

Hệ thống được hiện thực và kiểm thử trong môi trường phát triển cục bộ, bao gồm:

\begin{itemize}
    \item Máy tính cá nhân chạy hệ điều hành Windows.
    \item Unity Editor cho phía client.
    \item Node.js runtime cho phía backend.
    \item MongoDB chạy dưới dạng local service hoặc container.
\end{itemize}

Hệ thống không yêu cầu triển khai trên môi trường production trong giai đoạn này.

\subsubsection{Công cụ hỗ trợ}

Các công cụ chính được sử dụng trong quá trình hiện thực gồm:
\begin{itemize}
    \item Unity Editor: xây dựng client, scene và logic gameplay.
    \item NestJS CLI: tạo cấu trúc backend và các module dịch vụ.
    \item Git và GitHub: quản lý mã nguồn và phối hợp làm việc nhóm.
    \item Công cụ log và debug mặc định của Unity và Node.js.
\end{itemize}

%------------------------------------------------

\subsection{Hiện thực phía client}

\subsubsection{Cấu trúc project Unity}

Project Unity được tổ chức theo hướng tách biệt rõ các nhóm chức năng:

\begin{itemize}
    \item \textbf{Scenes:} các scene chính như Login, CharacterSelect và TestMap.
    \item \textbf{Scripts:} logic điều khiển nhân vật, xử lý input, quản lý kết nối mạng.
    \item \textbf{UI:} các thành phần giao diện như form đăng nhập và HUD cơ bản.
    \item \textbf{Network:} lớp giao tiếp HTTP và WebSocket với backend.
\end{itemize}

Cách tổ chức này giúp giảm phụ thuộc giữa gameplay, UI và networking.

%------------------------------------------------

\subsubsection{Hiện thực luồng đăng ký và đăng nhập}

Luồng đăng ký và đăng nhập được hiện thực theo mô hình client--server chuẩn:

\begin{itemize}
    \item Client hiển thị form đăng ký/đăng nhập và thu thập dữ liệu người dùng.
    \item Dữ liệu được gửi đến backend thông qua HTTP request.
    \item Client nhận phản hồi thành công hoặc lỗi và cập nhật giao diện tương ứng.
\end{itemize}

Token phiên nhận được sau khi đăng nhập được lưu tạm thời trên client để sử dụng
cho các request tiếp theo và cho quá trình thiết lập kết nối real-time.

%------------------------------------------------

\subsubsection{Hiện thực quản lý nhân vật}

Sau khi đăng nhập thành công, client thực hiện:

\begin{itemize}
    \item Gửi request tải hồ sơ người chơi và danh sách nhân vật.
    \item Hiển thị danh sách nhân vật hoặc giao diện tạo nhân vật mới.
    \item Lưu characterId được chọn để sử dụng trong phiên chơi.
\end{itemize}

Trong giai đoạn 1, dữ liệu nhân vật được sử dụng ở mức tối thiểu, chủ yếu phục vụ
việc xác định danh tính và vị trí khởi tạo trong phiên chơi thử nghiệm.

%------------------------------------------------

\subsubsection{Hiện thực điều khiển và hiển thị}

Client hiện thực cơ chế điều khiển nhân vật cơ bản:

\begin{itemize}
    \item Thu thập input di chuyển từ bàn phím.
    \item Gửi input dưới dạng ý định đến server thông qua WebSocket.
    \item Nhận snapshot trạng thái từ server và cập nhật vị trí nhân vật.
\end{itemize}

Ở giai đoạn 1, client ưu tiên hiển thị trực tiếp trạng thái do server gửi về, chưa
triển khai cơ chế dự đoán hay hiệu chỉnh phức tạp.

%------------------------------------------------

\subsection{Hiện thực phía server}

\subsubsection{Cấu trúc project backend}

Backend được hiện thực bằng NestJS và tổ chức theo kiến trúc module hoá:

\begin{itemize}
    \item \textbf{Auth Module:} xử lý đăng ký, đăng nhập và xác thực token.
    \item \textbf{User Module:} quản lý thông tin người dùng.
    \item \textbf{Character Module:} quản lý dữ liệu nhân vật.
    \item \textbf{Session/Room Module:} quản lý phiên chơi và room logic.
    \item \textbf{WebSocket Gateway:} xử lý giao tiếp real-time.
\end{itemize}

Cấu trúc này cho phép mở rộng thêm các module gameplay trong các giai đoạn tiếp theo.

%------------------------------------------------

\subsubsection{Hiện thực xác thực và quản lý phiên}

Backend hiện thực cơ chế xác thực theo các bước:

\begin{itemize}
    \item Nhận request đăng nhập từ client.
    \item Xác thực thông tin và tạo token phiên.
    \item Kiểm tra token cho các request HTTP và kết nối WebSocket.
\end{itemize}

Phiên real-time được quản lý bằng cách ánh xạ giữa token, userId, characterId
và socket tương ứng.

%------------------------------------------------

\subsubsection{Hiện thực WebSocket và đồng bộ trạng thái}

Gateway WebSocket chịu trách nhiệm:

\begin{itemize}
    \item Thiết lập kết nối real-time sau khi client xác thực.
    \item Nhận input di chuyển từ client.
    \item Cập nhật trạng thái nhân vật trong vòng lặp tick đơn giản.
    \item Phát snapshot trạng thái định kỳ về client.
\end{itemize}

Trong giai đoạn 1, vòng lặp tick được hiện thực ở mức cơ bản nhằm kiểm chứng
luồng dữ liệu và tính ổn định của kết nối.

%------------------------------------------------

\subsubsection{Xử lý mất kết nối}

Khi client mất kết nối WebSocket:

\begin{itemize}
    \item Server nhận sự kiện disconnect.
    \item Giải phóng session runtime liên quan.
    \item Cập nhật trạng thái nhân vật về offline (ở mức logic).
\end{itemize}

Cơ chế reconnect chỉ được xem xét ở mức định hướng và chưa hiện thực đầy đủ
trong giai đoạn này.

%------------------------------------------------

\subsection{Hiện thực lưu trữ dữ liệu}

\subsubsection{Lưu trữ dữ liệu bền vững}

MongoDB được sử dụng để lưu trữ:

\begin{itemize}
    \item Thông tin tài khoản người dùng.
    \item Thông tin nhân vật và thuộc tính cơ bản.
\end{itemize}

Dữ liệu được thiết kế linh hoạt nhằm sẵn sàng mở rộng cho các hệ thống gameplay
phức tạp hơn trong tương lai.

\subsubsection{Dữ liệu runtime}

Các dữ liệu runtime như session và trạng thái room được lưu trong bộ nhớ server.
Redis có thể được tích hợp trong các giai đoạn sau nếu cần mở rộng hoặc chia tải.

%------------------------------------------------

\subsection{Tích hợp client--server}

Quá trình tích hợp được thực hiện theo các bước:

\begin{enumerate}
    \item Client thực hiện đăng nhập và nhận token.
    \item Client thiết lập kết nối WebSocket và xác thực.
    \item Client tham gia phiên chơi thử nghiệm.
    \item Client và server trao đổi input và snapshot trạng thái.
\end{enumerate}

Việc tích hợp được kiểm thử với một số lượng nhỏ client đồng thời để đảm bảo
luồng hoạt động ổn định.

%------------------------------------------------

\subsection{Phạm vi hiện thực và giới hạn}

Trong giai đoạn 1, hệ thống chỉ hiện thực các chức năng nền tảng:

\begin{itemize}
    \item Đăng ký và đăng nhập.
    \item Quản lý nhân vật tối thiểu.
    \item Kết nối real-time và đồng bộ di chuyển.
\end{itemize}

Các hệ thống nâng cao như NPC, combat, gacha, dungeon và PvP chỉ được đề cập
ở mức thiết kế và không thuộc phạm vi hiện thực của chương này.

%------------------------------------------------

\subsection{Tổng kết chương}

Chương 8 đã trình bày quá trình hiện thực hệ thống Fortress of the Fallen trong
giai đoạn 1, bao gồm hiện thực client, backend, cơ chế lưu trữ dữ liệu và tích hợp
client--server.

Các kết quả hiện thực này là cơ sở cho Chương 9, nơi hệ thống sẽ được đánh giá
và kiểm thử dựa trên các tiêu chí đã xác định.
