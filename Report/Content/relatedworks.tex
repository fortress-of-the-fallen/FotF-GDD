\section{Các công trình liên quan}
\label{sec:relatedworks}

Chương này tổng hợp và phân tích các công trình đã tham khảo trong quá trình hình thành ý tưởng và định hướng thiết kế cho đề tài \textit{Fortress of the Fallen}. Các nguồn tham khảo được chia làm hai nhóm: (i) tác phẩm/phim/game làm nguồn cảm hứng thiết kế nội dung và hệ thống gameplay; (ii) tài liệu kỹ thuật định hướng kiến trúc hệ thống multiplayer và công nghệ triển khai.

Mục tiêu của chương là làm rõ mối liên hệ \textbf{Nguồn tham khảo $\rightarrow$ Bài học rút ra $\rightarrow$ Cách áp dụng vào đề tài}, từ đó tránh tình trạng liệt kê cảm hứng một cách chung chung.

% ===============================================================
\subsection{Khung phân tích và tiêu chí lựa chọn nguồn tham khảo}
\label{subsec:rw-method}

Nhóm lựa chọn các nguồn tham khảo dựa trên các tiêu chí:
\begin{itemize}
    \item Có cơ chế \textbf{tiến trình} (progression) và \textbf{động lực dài hạn} rõ ràng, phù hợp với Action RPG online.
    \item Có ít nhất một hệ thống đặc trưng có thể trừu tượng hoá thành mô-đun thiết kế (ví dụ: leo tháp, gacha, căn cứ/đảo cá nhân, tổ đội, raid/boss).
    \item Có đủ thông tin để phân tích theo các lớp: \textbf{vòng lặp chơi (loop)}, \textbf{hệ thống phần thưởng}, \textbf{mức độ rủi ro}, \textbf{tương tác người chơi}, \textbf{tổ chức nội dung}.
\end{itemize}

Ngoài ra, nhóm sử dụng thêm các tài liệu nền tảng về game design và kiến trúc multiplayer để đối chiếu và hợp thức hoá các quyết định thiết kế ở góc độ kỹ thuật và học thuật \cite{schell-art,adams-game-design,gaffer-networking,unity-manual,nest-docs,mongodb-docs,redis-docs}.

% ===============================================================
\subsection{Nhóm nguồn cảm hứng về cấu trúc tiến trình: Leo tháp, boss và checkpoint}

\subsubsection{Sword Art Online (Season 1): trục nội dung leo tầng và boss tầng}
\label{subsec:rw-sao}

\textit{Sword Art Online} mùa 1 mô tả thế giới game với cấu trúc \textbf{leo tầng tuyến tính}, trong đó mỗi tầng có hệ sinh thái, nhiệm vụ và đặc biệt là \textbf{boss tầng} đóng vai trò ``cổng kiểm soát tiến trình'' \cite{sao-aincrad}.  
Từ góc độ thiết kế hệ thống, SAO gợi ý ba điểm quan trọng:

\begin{itemize}
    \item \textbf{Phân lớp nội dung (content stratification):} nội dung được chia tầng giúp kiểm soát độ khó theo tiến trình, dễ thiết kế nhịp tăng trưởng sức mạnh.
    \item \textbf{Boss là mốc kiểm chứng build:} boss tầng buộc người chơi tối ưu hoá trang bị/kỹ năng, từ đó tạo động lực chuẩn bị và tương tác tổ đội.
    \item \textbf{Checkpoint rõ ràng:} mỗi lần qua tầng là một ``thành tựu'' (achievement milestone), phù hợp để gắn thưởng và mở khoá.
\end{itemize}

\textbf{Áp dụng vào đề tài:}  
Đề tài kế thừa ý tưởng \textbf{Tháp trung tâm 100 tầng} như trục nội dung dài hạn. Tuy nhiên, thay vì tuyến tính tuyệt đối, hệ thống được định hướng thiết kế theo dạng:
\begin{itemize}
    \item \textbf{Tầng chuẩn (main floors):} đi theo tiến trình chính.
    \item \textbf{Tầng thử thách (challenge/optional):} tăng độ khó, thưởng cao hơn.
    \item \textbf{Cơ chế checkpoint:} lưu tiến độ theo mốc để giảm ``cảm giác mất trắng'', phù hợp trải nghiệm game hiện đại.
\end{itemize}

Trong giai đoạn 1, nội dung leo tháp được coi là \textbf{tính năng nâng cao} (không hiện thực), nhưng được đưa vào thiết kế tổng quan để định hướng kiến trúc dữ liệu và mô-đun hoá từ sớm.

% ===============================================================
\subsection{Nhóm nguồn cảm hứng về thế giới sống: NPC, vận hành thế giới và vai trò tổ chức}

\subsubsection{Overlord: NPC có vai trò, trật tự và vận hành thế giới}
\label{subsec:rw-overlord}

\textit{Overlord} nhấn mạnh cảm giác ``thế giới vận hành độc lập'', trong đó NPC không chỉ là đối tượng đứng yên mà là thành phần có \textbf{vai trò, nhiệm vụ và quan hệ} \cite{overlord-novel}.  
Các bài học thiết kế rút ra:

\begin{itemize}
    \item \textbf{NPC như tác nhân hệ thống (system agents):} NPC có thể tham gia vận hành tài nguyên, phòng thủ, sản xuất, hoặc mở tuyến nhiệm vụ.
    \item \textbf{Tầng tổ chức tạo chiều sâu:} khi NPC gắn với chức trách, người chơi có động lực xây dựng đội hình/nhân sự thay vì chỉ tăng chỉ số.
    \item \textbf{Tính bền vững của tiến trình:} thế giới không ``reset'' theo phiên chơi; tiến trình có tính tích luỹ.
\end{itemize}

\textbf{Áp dụng vào đề tài:}  
Đề tài định hướng NPC như tài nguyên dài hạn phục vụ phát triển căn cứ/đảo cá nhân (mô hình ``nhân sự vận hành''). Tuy nhiên, theo phạm vi giai đoạn 1, phần NPC và đảo cá nhân được xếp vào \textbf{tính năng nâng cao} và chỉ dừng ở mức thiết kế/định hướng dữ liệu.

% ===============================================================
\subsection{Nhóm nguồn cảm hứng về cơ chế tuyển dụng/thu thập: Gacha và meta-progression}

\subsubsection{Pick Me Up! Infinite Gacha: gacha như hệ thống meta và động lực dài hạn}
\label{subsec:rw-pickmeup}

\textit{Pick Me Up! Infinite Gacha} cung cấp khung tham khảo cho cơ chế \textbf{tuyển dụng qua gacha} và xây dựng meta-progression dựa trên bộ sưu tập (collection) \cite{pickmeup-webtoon}.  
Các điểm đáng chú ý:

\begin{itemize}
    \item \textbf{Độ hiếm (rarity) và vai trò:} mỗi thực thể có độ hiếm và chức năng khác nhau, từ đó tạo động lực tối ưu đội hình.
    \item \textbf{Gacha tạo vòng lặp meta:} chơi để kiếm tài nguyên $\rightarrow$ quay gacha $\rightarrow$ tối ưu $\rightarrow$ chơi nội dung khó hơn.
    \item \textbf{Giảm ``dead-end'':} luôn có khả năng cải thiện thông qua hệ thống thu thập và nâng cấp.
\end{itemize}

\textbf{Áp dụng vào đề tài:}  
Đề tài sử dụng gacha như một ý tưởng cho hệ thống NPC/nhân sự (nâng cao). Về thiết kế, gacha không được dùng để ép buộc trả phí trong đồ án, mà đóng vai trò \textbf{cơ chế tạo biến thiên tiến trình} và \textbf{động lực sưu tầm}.

% ===============================================================
\subsection{Nhóm nguồn cảm hứng về trải nghiệm Action RPG: nhịp combat, phản hồi và điều khiển}

\subsubsection{Arcane Odyssey: cảm giác điều khiển và tương tác thời gian thực trong online RPG}
\label{subsec:rw-arcaneodyssey}

\textit{Arcane Odyssey} là tham chiếu về cách tổ chức trải nghiệm chiến đấu và di chuyển trong môi trường online \cite{arcane-odyssey}.  
Các bài học rút ra:

\begin{itemize}
    \item \textbf{Phản hồi tức thời (immediate feedback):} người chơi cần thấy rõ kết quả của thao tác (đòn đánh, né, hiệu ứng).
    \item \textbf{Nhấn mạnh kỹ năng người chơi:} kết quả phụ thuộc vào thao tác và vị trí (positioning) thay vì chỉ chỉ số.
    \item \textbf{Tối giản rào cản thao tác:} UI và input không được cản trở core loop.
\end{itemize}

\textbf{Áp dụng vào đề tài:}  
Đề tài định hướng combat của \textit{Fortress of the Fallen} theo hướng hack-and-slash: nhịp nhanh, có né tránh, có hit timing. Trong giai đoạn 1, hệ thống combat chưa hiện thực đầy đủ, nhưng các yêu cầu về \textbf{real-time sync} và \textbf{độ trễ thấp} được đưa vào phần phi chức năng và kiến trúc networking.

\subsubsection{Soul Knight Prequel: vòng lặp ngắn, build đơn giản và tái chơi}
\label{subsec:rw-soulknight}

\textit{Soul Knight Prequel} là tham chiếu về tổ chức vòng lặp chơi ngắn nhưng lặp lại nhiều lần \cite{soulknight-prequel}.  
Điểm rút ra:

\begin{itemize}
    \item \textbf{Session ngắn:} phù hợp người chơi phổ thông, dễ quay lại.
    \item \textbf{Reward rõ ràng theo phiên:} kết thúc phiên có thưởng để duy trì động lực.
    \item \textbf{Build dễ hiểu:} cho phép người chơi thử nghiệm mà không bị quá tải.
\end{itemize}

\textbf{Áp dụng vào đề tài:}  
Đề tài định hướng nội dung dungeon/thử thách theo dạng phiên (instance-based), dễ đóng gói thành mô-đun server về sau.

% ===============================================================
\subsection{Nhóm nguồn cảm hứng về căn cứ/đảo cá nhân: phát triển dài hạn và quản trị tài nguyên}

\subsubsection{Clash of Clans: căn cứ cá nhân và tiến trình phát triển theo tài nguyên}
\label{subsec:rw-coc}

\textit{Clash of Clans} là tham chiếu nổi bật cho cơ chế căn cứ cá nhân: người chơi đầu tư tài nguyên để nâng cấp công trình theo thời gian, tạo meta-progression bền vững \cite{clashofclans}.  
Các bài học quan trọng:

\begin{itemize}
    \item \textbf{Đảo/căn cứ như ``nhà'' của người chơi:} tạo gắn bó và mục tiêu dài hạn.
    \item \textbf{Quản trị tài nguyên:} nguồn (sources) và nơi tiêu (sinks) được thiết kế cân bằng để duy trì nhịp chơi.
    \item \textbf{Tăng trưởng theo thời gian:} cơ chế thời gian giúp tạo động lực quay lại (return loop).
\end{itemize}

\textbf{Áp dụng vào đề tài:}  
Đề tài đề xuất \textbf{đảo vệ tinh cá nhân} như hệ thống nâng cao để mở rộng meta-progression (kết hợp NPC vận hành). Tuy nhiên, ở giai đoạn 1, đảo cá nhân và NPC được xác định là \textbf{tính năng nâng cao}, chỉ đưa vào phạm vi thiết kế định hướng chứ không hiện thực.

% ===============================================================
\subsection{Tổng hợp: Bảng đối chiếu nguồn tham khảo và phần áp dụng}
\label{subsec:rw-mapping}

Bảng \ref{tab:rw-mapping} tóm tắt mối liên hệ giữa nguồn tham khảo và cách áp dụng vào đề tài nhằm đảm bảo tính minh bạch trong lập luận thiết kế.

\begin{table}[H]
\centering
\renewcommand{\arraystretch}{1.25}
\setlength{\tabcolsep}{6pt}
\begin{tabular}{|p{3.2cm}|p{5.1cm}|p{5.4cm}|}
\hline
\textbf{Nguồn tham khảo} & \textbf{Yếu tố rút ra} & \textbf{Áp dụng vào đề tài} \\
\hline
SAO S1 \cite{sao-aincrad} &
Leo tầng tuyến tính, boss tầng, checkpoint tiến trình &
Định hướng tháp 100 tầng (nâng cao), phân lớp nội dung và checkpoint \\
\hline
Overlord \cite{overlord-novel} &
NPC có vai trò, thế giới vận hành, tầng tổ chức &
Định hướng NPC vận hành hệ thống (nâng cao), thiết kế dữ liệu NPC/role \\
\hline
Pick Me Up! \cite{pickmeup-webtoon} &
Gacha tuyển dụng, rarity, meta-progression &
Định hướng gacha NPC (nâng cao), vòng lặp sưu tầm và tối ưu \\
\hline
Arcane Odyssey \cite{arcane-odyssey} &
Cảm giác điều khiển, phản hồi combat, real-time online &
Định hướng combat hack-and-slash và yêu cầu real-time sync (phi chức năng) \\
\hline
Soul Knight Prequel \cite{soulknight-prequel} &
Vòng lặp ngắn, reward rõ, build dễ hiểu &
Định hướng dungeon/instance theo phiên (nâng cao) \\
\hline
Clash of Clans \cite{clashofclans} &
Căn cứ cá nhân, quản trị tài nguyên, tiến trình theo thời gian &
Định hướng đảo cá nhân và hệ thống công trình/tài nguyên (nâng cao) \\
\hline
\end{tabular}
\caption{Đối chiếu nguồn tham khảo và hướng áp dụng vào đề tài}
\label{tab:rw-mapping}
\end{table}

% ===============================================================
\subsection{Tham khảo kỹ thuật và liên hệ với kiến trúc hệ thống}
\label{subsec:rw-technical}

Bên cạnh nguồn cảm hứng nội dung, đề tài cần nền tảng kỹ thuật để hiện thực hoá game online thời gian thực.
Do đó, nhóm tham khảo các tài liệu về:
\begin{itemize}
    \item \textbf{Game design và cách lập luận thiết kế hệ thống gameplay} \cite{schell-art,adams-game-design}.
    \item \textbf{Networking multiplayer} (server authoritative, prediction, interpolation) \cite{gaffer-networking}.
    \item \textbf{Công nghệ triển khai} ở mức đồ án: Unity cho client \cite{unity-manual}, NestJS cho backend \cite{nest-docs}, MongoDB \cite{mongodb-docs}, Redis \cite{redis-docs}.
\end{itemize}

Các tài liệu này được dùng để: (i) chuẩn hoá thuật ngữ và lập luận trong báo cáo; (ii) làm cơ sở cho các quyết định kiến trúc ở Chương 7; (iii) đảm bảo phần hiện thực giai đoạn 1 bám theo các thực hành phổ biến thay vì triển khai tuỳ hứng.

% ===============================================================
\subsection{Kết luận chương}
\label{subsec:rw-summary}

Chương này đã phân tích các công trình liên quan theo hướng ``tham khảo có kiểm soát'':
\begin{itemize}
    \item Các nguồn như SAO, Overlord, Pick Me Up, Arcane Odyssey, Soul Knight Prequel và Clash of Clans cung cấp khung ý tưởng cho hệ thống leo tháp, NPC, gacha và đảo cá nhân \cite{sao-aincrad,overlord-novel,pickmeup-webtoon,arcane-odyssey,soulknight-prequel,clashofclans}.
    \item Các tài liệu kỹ thuật cung cấp nền tảng để hiện thực kiến trúc multiplayer và lựa chọn công nghệ phù hợp phạm vi đồ án \cite{schell-art,adams-game-design,gaffer-networking,unity-manual,nest-docs,mongodb-docs,redis-docs}.
\end{itemize}

Các kết quả tổng hợp là cơ sở để xác định yêu cầu (Chương 5) và thiết kế hệ thống (Chương 7) theo hướng nhất quán, có thể mở rộng.
