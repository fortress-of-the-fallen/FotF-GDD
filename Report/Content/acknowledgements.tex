% ===================== Content/acknowledgements.tex =====================
\section*{Lời cảm ơn}
\addcontentsline{toc}{section}{Lời cảm ơn}

Trong quá trình thực hiện đồ án ``Phát triển hệ thống game online nhập vai hành động nhiều người chơi \textit{Fortress of the Fallen}'', nhóm đã nhận được sự hỗ trợ, hướng dẫn và tạo điều kiện từ nhiều cá nhân và tập thể. Với tinh thần trân trọng, nhóm xin gửi lời cảm ơn chân thành đến những người đã đồng hành và đóng góp để nhóm có thể hoàn thành đồ án cũng như báo cáo theo đúng mục tiêu và phạm vi đề ra.

Trước hết, nhóm xin bày tỏ lòng biết ơn sâu sắc đến \textbf{ThS. Vương Bá Thịnh} --- giảng viên hướng dẫn trực tiếp --- người đã dành nhiều thời gian theo sát tiến độ, định hướng phương pháp tiếp cận và liên tục góp ý trong từng giai đoạn. Từ việc chốt phạm vi bài toán, thống nhất mô hình client--server, tổ chức mô-đun backend theo hướng rõ ràng, cho đến việc hoàn thiện thiết kế dữ liệu và luồng giao tiếp REST/WebSocket, những nhận xét của thầy đã giúp nhóm tránh được các sai lệch phổ biến khi phát triển một hệ thống game online. Không chỉ hỗ trợ về mặt kỹ thuật, thầy còn góp ý cách trình bày báo cáo theo hướng chặt chẽ, có luận điểm và có dẫn chứng, giúp nhóm cải thiện đáng kể chất lượng nội dung và tính nhất quán giữa các chương.

Nhóm xin chân thành cảm ơn \textbf{quý thầy/cô trong Khoa Khoa học và Kỹ thuật Máy tính} đã giảng dạy và trang bị nền tảng kiến thức xuyên suốt quá trình học tập. Các học phần liên quan đến công nghệ phần mềm, cơ sở dữ liệu, mạng máy tính, hệ thống phân tán, cũng như các kỹ năng phân tích và thiết kế hệ thống đã tạo tiền đề quan trọng để nhóm có thể tiếp cận đề tài theo hướng có phương pháp. Bên cạnh đó, môi trường học tập của Khoa giúp nhóm rèn luyện tư duy kỷ luật, cách làm việc theo quy trình, và ý thức về chất lượng sản phẩm, đặc biệt quan trọng với những đề tài có tính tích hợp cao giữa gameplay, backend, dữ liệu và vận hành.

Nhóm cũng xin gửi lời cảm ơn đến \textbf{Hội đồng chấm bảo vệ} đã dành thời gian đọc, đánh giá và góp ý cho đồ án. Các nhận xét phản biện và gợi ý cải thiện là cơ sở để nhóm nhìn nhận rõ hơn những điểm còn hạn chế về kiến trúc, dữ liệu và trải nghiệm người dùng, đồng thời định hướng các hạng mục cần ưu tiên trong các giai đoạn phát triển tiếp theo. Nhóm xin tiếp thu nghiêm túc các góp ý và sẽ chủ động điều chỉnh để hoàn thiện sản phẩm ở mức tốt hơn.

Trong suốt quá trình triển khai, nhóm cũng nhận được nhiều hỗ trợ từ \textbf{bạn bè, các anh/chị khóa trên và các nhóm đồ án khác} thông qua việc trao đổi kinh nghiệm và chia sẻ tài liệu tham khảo. Những trao đổi về cách tổ chức mã nguồn, cách quản lý tài nguyên (asset), cách thiết kế cấu hình theo hướng data-driven, cũng như các lưu ý khi kiểm thử luồng nghiệp vụ và xử lý lỗi thực tế đã giúp nhóm rút ngắn đáng kể thời gian thử--sai. Đây là nguồn tham khảo hữu ích để nhóm cân bằng giữa mục tiêu kỹ thuật và giới hạn thời gian của một đồ án học thuật.

Cuối cùng, nhóm xin gửi lời cảm ơn sâu sắc đến \textbf{gia đình} đã luôn động viên, tạo điều kiện và hỗ trợ tinh thần để nhóm có thể tập trung thực hiện đồ án trong giai đoạn cao điểm. Sự cảm thông và đồng hành của gia đình là động lực quan trọng để nhóm duy trì kỷ luật làm việc, hoàn thành các mốc tiến độ và vượt qua áp lực về thời gian.

Nhóm ý thức rằng, dù đã nỗ lực hoàn thiện trong phạm vi cho phép, báo cáo và sản phẩm hiện thực vẫn có thể còn thiếu sót do giới hạn nguồn lực và thời lượng thực hiện. Nhóm kính mong tiếp tục nhận được sự góp ý từ quý thầy/cô để có thể hoàn thiện hơn trong các bước phát triển kế tiếp. Nhóm xin trân trọng cảm ơn.

\vspace{0.8cm}
\begin{flushright}
TP. Hồ Chí Minh, năm 2025
\end{flushright}
