\section{Đánh giá hệ thống}

\subsection{Mục tiêu đánh giá}

Mục tiêu của chương này là đánh giá mức độ đáp ứng của hệ thống Fortress of the Fallen
so với các mục tiêu và yêu cầu đã xác định ở Chương 5, đồng thời kiểm chứng tính đúng đắn
của các quyết định thiết kế và hiện thực trong giai đoạn 1.

Việc đánh giá tập trung vào:
\begin{itemize}
    \item Mức độ hoàn thành các yêu cầu chức năng cốt lõi.
    \item Khả năng vận hành ổn định của hệ thống prototype.
    \item Mức độ phù hợp giữa thiết kế và hiện thực.
    \item Các hạn chế còn tồn tại trong phạm vi đồ án.
\end{itemize}

Hệ thống không được đánh giá theo tiêu chí sản phẩm thương mại hay quy mô lớn,
mà theo góc độ một prototype học thuật.

%------------------------------------------------

\subsection{Phương pháp đánh giá}

Hệ thống được đánh giá dựa trên các phương pháp sau:

\begin{itemize}
    \item \textbf{Đối chiếu yêu cầu:} so sánh chức năng hiện thực với yêu cầu chức năng
    và phi chức năng đã nêu ở Chương 5.
    \item \textbf{Kiểm thử theo use case:} kiểm chứng các ca sử dụng đã phân tích ở Chương 6.
    \item \textbf{Quan sát vận hành:} theo dõi hành vi hệ thống khi chạy prototype với
    một số client đồng thời.
    \item \textbf{Phân tích định tính:} đánh giá mức độ rõ ràng, nhất quán và khả năng mở rộng.
\end{itemize}

Các phép đo định lượng chi tiết (benchmark hiệu năng, stress test) không được
đặt trọng tâm trong giai đoạn này.

%------------------------------------------------

\subsection{Đánh giá yêu cầu chức năng}

\subsubsection{Đăng ký và đăng nhập}

Hệ thống đã hiện thực thành công các chức năng:
\begin{itemize}
    \item Đăng ký tài khoản mới.
    \item Đăng nhập bằng thông tin hợp lệ.
    \item Từ chối đăng nhập với thông tin không hợp lệ.
\end{itemize}

Quá trình xác thực hoạt động đúng luồng phân tích, token được cấp và sử dụng
cho các bước tiếp theo. Điều này đáp ứng đầy đủ các yêu cầu chức năng UC01 và UC02.

%------------------------------------------------

\subsubsection{Quản lý nhân vật}

Chức năng quản lý nhân vật ở mức tối thiểu được hiện thực gồm:
\begin{itemize}
    \item Tải hồ sơ người chơi sau đăng nhập.
    \item Tạo nhân vật cơ bản khi chưa có dữ liệu.
    \item Chọn nhân vật để tham gia phiên chơi.
\end{itemize}

Các chức năng này đáp ứng các use case UC04, UC05 và UC06, đồng thời tạo nền tảng
cho việc mở rộng hệ thống progression trong các giai đoạn tiếp theo.

%------------------------------------------------

\subsubsection{Kết nối real-time và phiên chơi}

Hệ thống đã thiết lập được kết nối WebSocket ổn định giữa client và server:
\begin{itemize}
    \item Client xác thực kết nối real-time bằng token hợp lệ.
    \item Server quản lý phiên chơi và ánh xạ user--character--socket.
    \item Người chơi được gán vào phiên chơi thử nghiệm.
\end{itemize}

Các chức năng này đáp ứng UC07 và UC08 trong phạm vi giai đoạn 1.

%------------------------------------------------

\subsubsection{Đồng bộ trạng thái di chuyển}

Cơ chế đồng bộ di chuyển được kiểm chứng thông qua:
\begin{itemize}
    \item Client gửi input di chuyển theo nhịp.
    \item Server xử lý input và cập nhật trạng thái authoritative.
    \item Client nhận snapshot và hiển thị vị trí nhân vật.
\end{itemize}

Hệ thống hoạt động ổn định với số lượng nhỏ client đồng thời, đáp ứng UC09 và UC10.
Việc chưa triển khai prediction và reconciliation nâng cao được xem là phù hợp
với phạm vi prototype.

%------------------------------------------------

\subsubsection{Xử lý mất kết nối}

Khi client mất kết nối:
\begin{itemize}
    \item Server nhận sự kiện disconnect.
    \item Session runtime được giải phóng.
    \item Client hiển thị trạng thái mất kết nối.
\end{itemize}

Chức năng này đáp ứng UC11 ở mức cơ bản, đảm bảo hệ thống không rơi vào trạng thái
không nhất quán khi có lỗi mạng.

%------------------------------------------------

\subsection{Đánh giá yêu cầu phi chức năng}

\subsubsection{Hiệu năng}

Trong môi trường thử nghiệm cục bộ:
\begin{itemize}
    \item Độ trễ truyền dữ liệu ở mức chấp nhận được cho prototype.
    \item Hệ thống xử lý ổn định với 2--3 client đồng thời.
\end{itemize}

Hệ thống chưa được kiểm thử ở tải lớn, điều này phù hợp với phạm vi giai đoạn 1.

%------------------------------------------------

\subsubsection{Tính ổn định}

Trong quá trình kiểm thử:
\begin{itemize}
    \item Backend không bị crash khi client ngắt kết nối đột ngột.
    \item Các session được giải phóng đúng cách.
\end{itemize}

Điều này cho thấy thiết kế quản lý phiên và xử lý lỗi cơ bản hoạt động đúng.

%------------------------------------------------

\subsubsection{Bảo mật ở mức cơ bản}

Các biện pháp bảo mật tối thiểu đã được áp dụng:
\begin{itemize}
    \item Mật khẩu được mã hoá trước khi lưu trữ.
    \item Client không được phép tự quyết định trạng thái gameplay.
    \item Token được kiểm tra trước khi truy cập các chức năng quan trọng.
\end{itemize}

Các cơ chế bảo mật nâng cao chưa được triển khai trong giai đoạn này.

%------------------------------------------------

\subsection{Đánh giá mức độ phù hợp giữa thiết kế và hiện thực}

Kết quả hiện thực cho thấy:
\begin{itemize}
    \item Các mô-đun backend được tổ chức đúng theo thiết kế ở Chương 7.
    \item Luồng client--server vận hành đúng theo phân tích use case.
    \item Không phát sinh sự lệch hướng lớn giữa thiết kế và triển khai.
\end{itemize}

Điều này chứng minh thiết kế hệ thống ban đầu có tính khả thi và phù hợp
với điều kiện hiện thực của đồ án.

%------------------------------------------------

\subsection{Hạn chế và tồn tại}

Bên cạnh các kết quả đạt được, hệ thống vẫn còn một số hạn chế:

\begin{itemize}
    \item Chưa hỗ trợ các hệ thống gameplay nâng cao (combat, NPC, gacha).
    \item Chưa tối ưu cho số lượng người chơi lớn.
    \item Chưa triển khai đầy đủ các kỹ thuật networking nâng cao.
\end{itemize}

Các hạn chế này xuất phát từ việc chủ động giới hạn phạm vi để đảm bảo
tính khả thi và chất lượng học thuật của đồ án.

%------------------------------------------------

\subsection{Tổng kết chương}

Chương 9 đã đánh giá hệ thống Fortress of the Fallen trong giai đoạn 1 dựa trên
các yêu cầu và use case đã phân tích. Kết quả cho thấy hệ thống prototype đáp ứng
được các mục tiêu nền tảng về kiến trúc, kết nối real-time và quản lý phiên chơi.

Những kết quả và hạn chế được ghi nhận là cơ sở cho phần kết luận và định hướng
phát triển trong Chương 10.
