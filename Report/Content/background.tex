\section{Kiến thức nền tảng}

Chương này trình b trình bày hệ thống các kiến thức cơ sở liên quan đến phát triển trò chơi điện tử thuộc thể loại Action RPG trong môi trường trực tuyến. Nội dung bao gồm: (1) các khái niệm nền tảng về game thời gian thực và hệ thống multiplayer; (2) kiến thức nền tảng về thiết kế game (Game Design), bao gồm core loop, progression, combat, economy và UX. Những kiến thức này đóng vai trò quan trọng làm cơ sở cho phân tích yêu cầu và thiết kế hệ thống ở các chương tiếp theo.

% ===============================================================
\subsection{Kiến thức nền tảng về Action RPG và môi trường real-time}

\subsubsection{Đặc điểm của Action RPG}

Action RPG (Action Role-Playing Game) là thể loại game kết hợp cơ chế nhập vai với chiến đấu thời gian thực. Đặc điểm quan trọng bao gồm:

\begin{itemize}
    \item \textbf{Tương tác thời gian thực}: mọi hành động như tấn công, di chuyển, né tránh phải được phản hồi ngay lập tức.
    \item \textbf{Combat gắn liền với animation}: chuyển động của nhân vật ảnh hưởng trực tiếp đến logic chiến đấu.
    \item \textbf{Hitbox / Hurtbox}: hệ thống kiểm tra va chạm dựa trên hình học đơn giản như AABB, Sphere, Capsule.
    \item \textbf{Kỹ năng người chơi và chỉ số nhân vật phối hợp}: hiệu quả chiến đấu phụ thuộc vào cả thao tác người chơi và stat.
\end{itemize}

Một đòn tấn công chuẩn thường gồm ba giai đoạn:
\begin{enumerate}
    \item \textbf{Startup}: nhân vật chuẩn bị ra đòn.
    \item \textbf{Active}: hitbox có hiệu lực.
    \item \textbf{Recovery}: quay về trạng thái sẵn sàng cho hành động tiếp theo.
\end{enumerate}

Những khái niệm này giúp đảm bảo tính logic và cảm giác “mượt” khi chiến đấu.

% ===============================================================
\subsubsection{Hệ thống multiplayer trong Action RPG}

Một Action RPG nhiều người chơi yêu cầu cơ chế đồng bộ trạng thái giữa các người chơi sao cho tất cả đều thấy cùng một diễn biến. Các kiến thức nền trong hệ thống multiplayer bao gồm:

\paragraph{Mô hình client--server}

Hầu hết game online dùng mô hình server authoritative:
\begin{itemize}
    \item \textbf{Client}: thu input và hiển thị kết quả.
    \item \textbf{Server}: xử lý toàn bộ logic gameplay và quản lý trạng thái thật.
\end{itemize}

Ưu điểm:
\begin{itemize}
    \item đảm bảo tính nhất quán,
    \item chống gian lận,
    \item dễ quản lý khi mở rộng.
\end{itemize}

\paragraph{Tick-rate và vòng lặp cập nhật}

Server hoạt động theo “tick”, ví dụ 20 hoặc 30 lần mỗi giây. Mỗi tick:
\begin{verbatim}
- Nhận input từ người chơi
- Cập nhật vị trí và trạng thái
- Kiểm tra va chạm và xử lý chiến đấu
- Gửi snapshot cho client
\end{verbatim}

Tick-rate càng cao, gameplay càng mượt nhưng tốn tài nguyên.

\paragraph{Đồng bộ trạng thái}

Hai kỹ thuật phổ biến:

\begin{itemize}
    \item \textbf{Snapshot interpolation}: nội suy giữa hai snapshot để chuyển động mượt.
    \item \textbf{Client-side prediction}: client dự đoán tạm thời để tránh cảm giác điều khiển bị trễ.
\end{itemize}

Kết hợp:
\begin{itemize}
    \item \textbf{Server reconciliation}: server gửi dữ liệu chính xác, client điều chỉnh sai lệch.
\end{itemize}

Những kiến thức này giúp giảm ảnh hưởng của độ trễ mạng trong môi trường real-time.

% ===============================================================
\subsection{Kiến thức nền tảng về Game Design}

Game Design là lĩnh vực xây dựng cơ chế hoạt động của trò chơi, quy định người chơi làm gì, cách họ tương tác và động lực khiến họ tiếp tục chơi. Một trò chơi hoàn chỉnh bao gồm nhiều lớp thiết kế: core loop, hệ thống phát triển nhân vật, combat, tài nguyên, phần thưởng và trải nghiệm người dùng.

% ---------------------------------------------------------------
\subsubsection{Core Loop và Meta Loop}

\paragraph{Core Loop}

Core loop là vòng lặp hành động cốt lõi mà người chơi thực hiện liên tục trong suốt thời gian chơi. Một ví dụ điển hình cho Action RPG:

\begin{center}
\textit{Chiến đấu → Nhận phần thưởng → Tăng tiến sức mạnh → Quay lại chiến đấu}
\end{center}

Một core loop tốt phải:
\begin{itemize}
    \item đơn giản,
    \item dễ hiểu,
    \item có nhịp độ nhanh,
    \item tạo cảm giác “hài lòng” mỗi vòng lặp.
\end{itemize}

\paragraph{Meta Loop}

Meta loop là hệ thống dài hạn thúc đẩy người chơi gắn bó lâu hơn, ví dụ:
\begin{itemize}
    \item tăng cấp nhân vật,
    \item mở khóa kỹ năng mới,
    \item thu thập trang bị,
    \item khám phá khu vực mới.
\end{itemize}

Nếu core loop tạo “niềm vui ngắn hạn”, meta loop tạo “động lực dài hạn”.

% ---------------------------------------------------------------
\subsubsection{Progression System (Hệ thống phát triển nhân vật)}

Progression giúp định hình hành trình phát triển của nhân vật. Các yếu tố nền tảng:

\begin{itemize}
    \item \textbf{Leveling}: lên cấp để mở khóa chỉ số hoặc kỹ năng.
    \item \textbf{Stats}: các chỉ số như sức mạnh, nhanh nhẹn, trí lực, máu.
    \item \textbf{Equipment Progression}: vũ khí, áo giáp, vật phẩm hỗ trợ.
    \item \textbf{Skill Tree}: hệ thống kỹ năng dạng nhánh cho phép tùy biến phong cách chơi.
    \item \textbf{Unlocking System}: tiến xa hơn thì mở được khu vực/tính năng mới.
\end{itemize}

Progression phải được thiết kế có nhịp độ phù hợp để tránh:
\begin{itemize}
    \item tăng cấp quá nhanh gây nhàm,
    \item tăng quá chậm gây nản.
\end{itemize}

% ---------------------------------------------------------------
\subsubsection{Combat Design (Thiết kế chiến đấu)}

Combat là trung tâm của Action RPG. Các kiến thức nền gồm:

\paragraph{Pacing (nhịp độ)}

Combat có thể nhanh hoặc chậm tùy phong cách. Nhịp độ phải phù hợp:

\begin{itemize}
    \item nhanh → nhấn mạnh phản xạ, né tránh,
    \item chậm → thiên về chiến thuật và quan sát.
\end{itemize}

\paragraph{Frame Data}

Mỗi đòn đánh được chia thành:
\begin{itemize}
    \item \textbf{Startup}: chuẩn bị ra đòn,
    \item \textbf{Active}: hitbox có hiệu lực,
    \item \textbf{Recovery}: cooldown trước khi ra thêm đòn,
    \item \textbf{Invincibility Frame}: thời gian bất tử khi né.
\end{itemize}

\paragraph{Hit Detection}

Kiểm tra va chạm dựa trên:
\begin{itemize}
    \item hitbox của kỹ năng,
    \item hurtbox của mục tiêu,
    \item thời điểm animation.
\end{itemize}

\paragraph{Enemy Pattern}

Quái trong Action RPG thường có:
\begin{itemize}
    \item đòn đánh có báo hiệu (telegraph),
    \item nhiều phase chiến đấu,
    \item hành vi lặp theo pattern.
\end{itemize}

% ---------------------------------------------------------------
\subsubsection{World System và Level Design}

Trong game, “level design” quyết định cách người chơi di chuyển và tương tác.

Các yếu tố quan trọng:

\begin{itemize}
    \item bố cục không gian,
    \item đường di chuyển chính/phụ,
    \item phân bổ quái, vật phẩm,
    \item vùng an toàn và điểm checkpoint,
    \item nhịp độ giữa combat và khám phá.
\end{itemize}

Nếu level quá rối → người chơi lạc.  
Nếu quá đơn giản → mất hứng thú.

% ---------------------------------------------------------------
\subsubsection{Hệ thống kinh tế (Game Economy)}

Game Economy là kiến thức nền quan trọng trong mọi game RPG:

\paragraph{Nguồn tài nguyên (Resource Sources)}
\begin{itemize}
    \item phần thưởng theo nhiệm vụ,
    \item quái rơi vật phẩm,
    \item crafting,
    \item sự kiện.
\end{itemize}

\paragraph{Tiêu thụ tài nguyên (Resource Sinks)}
\begin{itemize}
    \item nâng cấp trang bị,
    \item mua vật phẩm,
    \item chi phí mở khóa tính năng.
\end{itemize}

Nếu nguồn thu lớn nhưng tiêu thụ ít → \textbf{lạm phát tài nguyên}.  
Nếu tiêu thụ nhiều nhưng khó kiếm → \textbf{người chơi bỏ game}.

% ---------------------------------------------------------------
\subsubsection{UX/UI Design trong game}

UX tốt giúp giảm tải nhận thức và tăng trải nghiệm. Các nguyên lý nền:

\begin{itemize}
    \item \textbf{Phản hồi rõ ràng (Visual Feedback)}: hiệu ứng khi nhặt vật phẩm, đánh trúng, nhận thưởng.
    \item \textbf{Giao diện nhất quán}: vị trí nút bấm và layout thống nhất.
    \item \textbf{Giảm tải thông tin}: chỉ hiển thị những gì người chơi cần.
\end{itemize}

% ---------------------------------------------------------------
\subsubsection{Balancing Fundamentals (Cân bằng game)}

Balancing là điều chỉnh sao cho:

\begin{itemize}
    \item nhân vật không quá mạnh hoặc quá yếu,
    \item progression hợp lý,
    \item phần thưởng tương xứng độ khó,
    \item thời gian hoàn thành nhiệm vụ không quá dài.
\end{itemize}

Một số khái niệm nền:
\begin{itemize}
    \item \textbf{Linear vs Exponential Growth},
    \item \textbf{Time-to-Kill (TTK)},
    \item \textbf{Risk–Reward Ratio},
    \item \textbf{Power Budget}.
\end{itemize}

% ===============================================================
\subsection{Tổng kết chương}

Chương này đã trình bày toàn bộ các nền tảng kỹ thuật và thiết kế liên quan đến Action RPG và game online. Các kiến thức về real-time gameplay, đồng bộ trạng thái, core loop, combat, progression, economy và UX sẽ được sử dụng làm cơ sở để phân tích yêu cầu và thiết kế hệ thống ở những chương tiếp theo.
